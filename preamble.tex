\usepackage{xspace}
\usepackage{tikz}
%\usepackage{morefloats,subfig,afterpage}
\usepackage{mathrsfs} % script font
\usepackage{verbatim}
\usepackage{cite}
\usepackage{graphicx}
\usepackage{caption}
\usepackage{subcaption}

%% Load special font packages here if you wish
\usepackage{lmodern}  % lmodern, mathpazo, euler

%% Using Babel allows other languages to be used and mixed-in easily
\usepackage[english]{babel}
\selectlanguage{english}

%% Use the tikz-feynman package for Feynman diagram.
\usepackage[compat=1.1.0]{tikz-feynman}

%% Tweak so maths adapts its boldness to match the context.
\makeatletter
\g@addto@macro\bfseries\boldmath
\makeatother

%% Maths
\usepackage{resources/abmath}
\DeclareRobustCommand{\mymath}[1]{\ensuremath{\maybebmsf{#1}}}
% \DeclareRobustCommand{\parenths}[1]{\mymath{\left({#1}\right)}\xspace}
% \DeclareRobustCommand{\braces}[1]{\mymath{\left\{{#1}\right\}}\xspace}
% \DeclareRobustCommand{\angles}[1]{\mymath{\left\langle{#1}\right\rangle}\xspace}
% \DeclareRobustCommand{\sqbracs}[1]{\mymath{\left[{#1}\right]}\xspace}
% \DeclareRobustCommand{\mods}[1]{\mymath{\left\lvert{#1}\right\rvert}\xspace}
% \DeclareRobustCommand{\modsq}[1]{\mymath{\mods{#1}^2}\xspace}
% \DeclareRobustCommand{\dblmods}[1]{\mymath{\left\lVert{#1}\right\rVert}\xspace}
% \DeclareRobustCommand{\expOf}[1]{\mymath{\exp{\!\parenths{#1}}}\xspace}
% \DeclareRobustCommand{\eexp}[1]{\mymath{e^{#1}}\xspace}
% \DeclareRobustCommand{\plusquad}{\mymath{\oplus}\xspace}
% \DeclareRobustCommand{\logOf}[1]{\mymath{\log\!\parenths{#1}}\xspace}
% \DeclareRobustCommand{\lnOf}[1]{\mymath{\ln\!\parenths{#1}}\xspace}
% \DeclareRobustCommand{\ofOrder}[1]{\mymath{\mathcal{O}\parenths{#1}}\xspace}
% \DeclareRobustCommand{\SOgroup}[1]{\mymath{\mathup{SO}\parenths{#1}}\xspace}
% \DeclareRobustCommand{\SUgroup}[1]{\mymath{\mathup{SU}\parenths{#1}}\xspace}
% \DeclareRobustCommand{\Ugroup}[1]{\mymath{\mathup{U}\parenths{#1}}\xspace}
% \DeclareRobustCommand{\I}[1]{\mymath{\mathrm{i}}\xspace}
% \DeclareRobustCommand{\colvector}[1]{\mymath{\begin{pmatrix}#1\end{pmatrix}}\xspace}
\DeclareRobustCommand{\Rate}{\mymath{\Gamma}\xspace}
\DeclareRobustCommand{\RateOf}[1]{\mymath{\Gamma}\parenths{#1}\xspace}

%% High-energy physics stuff
\usepackage{resources/abhep}
%\usepackage{resources/SIunits}
%\usepackage{hepnames}
%\usepackage{hepunits}

\begin{comment}
TODO: cross-section, nu and nu bar combos, proton, neutron, electron, muon, tau etc
\end{comment}

% Use \xspace at the end of the macros
% Use \text for text of units in macro
% write 5--10 for 5 to 10
% \mathrm{...} macro, i.e. $\mathrm{d}f/\mathrm{d}x$ → df /dx, not $df/dx$ → df /dx
% So put the whole of each mathematical expression in math mode (and drop out of it via \mathrm or \text if needed).
% Negative numbers should be written in math mode
% always wrap text in your equations in the \mathrm macro, which does it right

%% Personal macros
\DeclareRobustCommand{\chips}{\textsc{Chips}\xspace}
\DeclareRobustCommand{\chipsm}{\textsc{Chips-M}\xspace}
\DeclareRobustCommand{\chipsfive}{\textsc{Chips-5}\xspace}
\DeclareRobustCommand{\nova}{NOvA\xspace}
\DeclareRobustCommand{\minos}{\textsc{Minos}\xspace}
\DeclareRobustCommand{\numi}{NuMI\xspace}
\DeclareRobustCommand{\genie}{\textsc{Genie}\xspace}
\DeclareRobustCommand{\root}{\textsc{Root}\xspace}
\DeclareRobustCommand{\tensorflow}{\textsc{Tensorflow}\xspace}
\DeclareRobustCommand{\python}{\textsc{Python}\xspace}
\DeclareRobustCommand{\google}{\textsc{Google}\xspace}

\DeclareRobustCommand{\arXivCode}[1]{arXiv:#1}
\DeclareRobustCommand{\CP}{\ensuremath{\mathcal{CP}}\xspace}
\DeclareRobustCommand{\CPviolation}{\CP-violation\xspace}
\DeclareRobustCommand{\CPv}{\CPviolation}
\DeclareRobustCommand{\LHC}{LHC\xspace}
\DeclareRobustCommand{\CERN}{CERN\xspace}
\DeclareRobustCommand{\bphysics}{\Pbottom-physics\xspace}
\DeclareRobustCommand{\bhadron}{\Pbottom-hadron\xspace}
\DeclareRobustCommand{\Bmeson}{\PB-meson\xspace}
\DeclareRobustCommand{\bbaryon}{\Pbottom-baryon\xspace}
\DeclareRobustCommand{\Bdecay}{\PB-decay\xspace}
\DeclareRobustCommand{\bdecay}{\Pbottom-decay\xspace}
\DeclareRobustCommand{\BToKPi}{\HepProcess{ \PB \to \PK \Ppi }\xspace}
\DeclareRobustCommand{\BToPiPi}{\HepProcess{ \PB \to \Ppi \Ppi }\xspace}
\DeclareRobustCommand{\BToKK}{\HepProcess{ \PB \to \PK \PK }\xspace}
\DeclareRobustCommand{\BToRhoPi}{\HepProcess{ \PB \to \Prho \Ppi }\xspace}
\DeclareRobustCommand{\BToRhoRho}{\HepProcess{ \PB \to \Prho \Prho }\xspace}
\DeclareRobustCommand{\X}{\thesismath{X}\xspace}
\DeclareRobustCommand{\Xbar}{\thesismath{\overline{X}}\xspace}
\DeclareRobustCommand{\Xzero}{\HepGenParticle{X}{}{0}\xspace}
\DeclareRobustCommand{\Xzerobar}{\HepGenAntiParticle{X}{}{0}\xspace}
\DeclareRobustCommand{\epluseminus}{\Ppositron\!\Pelectron\xspace}
\DeclareRobustCommand{\protonproton}{\Pproton\APantiproton\xspace}
