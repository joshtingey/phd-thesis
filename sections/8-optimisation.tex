\chapter{Detector optimisation for CHIPS}
\label{chap:optimisation}

\begin{comment}
You have all of these possible parameters that can affect the performance (event categorisation and kinematic reconstruction) of your WC detector

Positioning of the detector (L, angle off-axis, overburden) *
Size of the detector (height and radius) *
Water quality (attenuation length, scattering vs absorption) *
Which PMT’s you use (time resolution, charge collection) *
How the PMT’s are positioned (percentage coverage, zones)
Calibration quality (position, time, charge)
Reconstruction methodology (likelihood vs CNN)
Given these restrictions

Only certain mine pits usable (fixes L and angle off-axis, use all overburden available)
12.5m radius for practical construction (fixes radius, but not height)
Want to be able to carry the planes easily (Adds limits to PMT coverage percentage)
Will just be looking at beam events, should not need much in the back
PMT’s available. Due to cost etc… we fix the PMT’s we use, can still explore this space with varying the time/charge.
Can look at

Detector height *
Water Quality *
How the PMT’s are positioned (can fix percentage coverage due to practical plane restriction)
Calibration Quality (network ensamble)
Reconstruction Methodology
Generation Parameters

Input beam flux (Use the ones I now have from Anna)
Oscillated beam flux (Use the one I now have)
Cross section splines (Use the ones I have always had, they should be good)
Generation energy range (0-15GeV, this definitely covers the full range of energies)
Fraction of H and O in water (Just keep the 95% to 5% we have always had, SuperK use)
Type of events to generate (Decide on the categories)
Simulation Parameters

Physics List (Negligable changes, therefore, use QGSP_BIC_HP as preferred for energies below 5 GeV)
PMT QE Option (Negligible changes, therefore, use stacking-only )
Absorption Length Scaling (Something to study, need default)
Reflectivity of glass cathode (0.24)
Rayleigh scattering length Scaling (Something to study, need default)
Blacksheet reflectivity Scaling (0.7)
Mie scattering length Scaling (0.0, it’s not important as Maciej found out)
Geometry definition (Something to study, need default)
PMT definition (Think they are pretty good already, use them)
Light Cone definition (Think this is pretty good already use it)
Fiducial generation distance (Set to 1m, higher density in middle and can explore edge)
PMT digitisation simulation (Decide with Paul, just use hits for now!!!)
PMT angular collection efficiency (Keep it as the one Paul suggested, mainly 100%)

TODO: Write the story of the optimisation chapter

TODO: Write the optimisation chapter section outline

TODO: Add all the sections below
\end{comment}