\chapter{Implications for CHIPS}
\label{chap:implications}

\begin{comment}
CALIBRATION SENSITIVITY
TROUBLESOME EVENTS
PMT DISTRIBUTION
TIMING RESOLUTION
WATER QUALITY
POSSIBLE IMPROVEMENTS

- You have all of these possible parameters that can affect the performance (event categorisation
and kinematic reconstruction) of your WC detector

- Positioning of the detector (L, angle off-axis, overburden) *
- Size of the detector (height and radius) *
- Water quality (attenuation length, scattering vs absorption) *
- Which PMT’s you use (time resolution, charge collection) *
- How the PMT’s are positioned (percentage coverage, zones)
- Calibration quality (position, time, charge)
- Reconstruction methodology (likelihood vs CNN)
- Given these restrictions

- Only certain mine pits usable (fixes L and angle off-axis, use all overburden available)
- 12.5m radius for practical construction (fixes radius, but not height)
- Want to be able to carry the planes easily (Adds limits to PMT coverage percentage)
- Will just be looking at beam events, should not need much in the back
- PMT’s available. Due to cost etc… we fix the PMT’s we use, can still explore this space with
varying the time/charge. Can look at
\end{comment}

- but was mainly chosen to speed up iteration of network optimisation.
- This works out at 2.5m is theta and 2m in phi approx (when viewed from the centre of the detector)