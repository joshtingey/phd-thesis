%% Contains all the frontmatter such as title page, abstract, declaration and contents etc...
\pagenumbering{arabic}

\title{Convolutional neural networks for the CHIPS neutrino detector R\&D project}
\author{Josh Chalcraft Tingey}

\titlepage[University College London]{ %%%%%%%%%%%%%%%%%%%%%%%%%%%%%%%%%%%%%%%%%%%%%%%%%%%%%%%%%%%
    Submitted to University College London in fulfilment \\
    of the requirements for the award of the \\
    degree of \textbf{Doctor of Philosophy}}
\thispagestyle{plain}

\begin{declaration} %%%%%%%%%%%%%%%%%%%%%%%%%%%%%%%%%%%%%%%%%%%%%%%%%%%%%%%%%%%%%%%%%%%%%%%%%%%%%%
    I, Josh Chalcraft Tingey confirm that the work presented in this thesis is my own. Where
    information has been derived from other sources, I confirm that this has been indicated in the
    thesis.
    \vspace*{1cm}
    \begin{flushright}
        Josh Chalcraft Tingey
    \end{flushright}
\end{declaration}

\begin{abstract} %%%%%%%%%%%%%%%%%%%%%%%%%%%%%%%%%%%%%%%%%%%%%%%%%%%%%%%%%%%%%%%%%%%%%%%%%%%%%%%%%
    The CHerenkov detectors In mine PitS (\chips) neutrino detector R\&D project aims to develop
    novel strategies and technologies for very large yet `cheap as chips' water Cherenkov neutrino
    detectors. Via deployment in a body of water, use of commercially available components, and
    instrumentation coverage optimisation for the study of exclusively accelerator beam neutrinos,
    \chips will enable megaton scale detectors to become a reality at the cost of \$200k-\$300k
    per kt of sensitive mass. During the summer of 2019 a prototype \chips detector, \chipsfive,
    was deployed into the Wentworth 2W disused mine pit in northern Minnesota, \SI{7}{\text{mrad}}
    off the \numi beam axis. A novel data acquisition system was introduced using cheap
    single-board computers and open-source software.

    This work presents a novel approach to water Cherenkov neutrino detector event reconstruction
    and classification. Three forms of a Convolutional Neural Network, a type of deep learning
    algorithm, have been trained to reject cosmic muon events, classify beam events, and estimate
    neutrino energies, all using only the raw detector event as input. When evaluated on the
    expected distribution of \chipsfive events, this new approach is shown to be robust and
    explainable as well as providing a significant performance increase over the standard
    likelihood-based reconstruction and simple neural network classification. Promisingly, the
    performance presented here is comparable to the more complex (and expensive) neutrino
    oscillation experiments within the field.
\end{abstract}

\begin{abstract}[Impact Statement] %%%%%%%%%%%%%%%%%%%%%%%%%%%%%%%%%%%%%%%%%%%%%%%%%%%%%%%%%%%%%%%
    The impact of this work inside the domain of academia is straightforward and explained in
    great detail throughout the thesis. A novel application of modern machine learning techniques
    is made to reconstruct and classify neutrino events within a water Cherenkov detector studying
    neutrino oscillations. Although other experiments within the field have conducted similar
    preliminary work, this is the first known comprehensive application of such methods.
    Hopefully, this work will promote the use of these techniques for water Cherenkov neutrino
    detectors more broadly.
    
    Outside of academia, the impact of this work is less clear. Still, it is probably most
    pronounced in the field of machine learning, which promises to bring substantial societal
    advancements. Sometimes the application of machine learning to real-world problems is seen as
    somewhat secondary to achieving marginally incremental improvements on standardised
    challenges. Hopefully, this work contributes towards a broader effort to make the
    application of machine learning methods more central and broaden the scope of tasks
    considered. Currently, the field is dominated by the commercial needs of a small number of
    large technology companies, such as Google and Facebook, any work that applies machine
    learning to a novel task can only help to broaden the scope of the field and make it more
    applicable to a wider range of real-world tasks.
\end{abstract}

\begin{acknowledgements} %%%%%%%%%%%%%%%%%%%%%%%%%%%%%%%%%%%%%%%%%%%%%%%%%%%%%%%%%%%%%%%%%%%%%%%%%
    Firstly I would like to thank my supervisor Jenny for the independence she offered me
    throughout by PhD to pursue the research path that interested me the most, and furthermore,
    for leading the whole \chips collaboration through two intense summers of construction
    activity in Hoyt Lakes.

    I would also like to thank everyone within the broader \chips collaboration. Foremost Stefano
    for his incredible support and advice throughout all the stages of my PhD, as well as the UCL
    group of Tom, Sim, Petr, and John for always being around and making the gulag bearable. I
    would also like to thank everyone else within the UCL HEP group, particularly my D14 office
    mates.

    Above all, I must thank Becca and my family for all of their endless support. 
\end{acknowledgements}

\begin{preface}
    Here I outline my contributions to the work described in this thesis on a chapter by chapter
    basis. Hopefully, this will make clear what is my own work and what is that of others. Due to
    the relatively small number of core people involved with the \chips project, I was somewhat
    unusually involved with a much broader scope of work than usual for a HEP PhD student, as were
    the majority of the \chips team.

    In \ChapterRef{chap:chips} the \chips project and the \chipsfive detector are described. I was
    personally heavily involved with developing the event generation, detector simulation, and
    event reconstruction frameworks for \chips, alongside extensive detector optimisation studies
    to inform the final configuration of the \chipsfive detector. I also contributed to the
    \chipsfive construction efforts, primarily instrumentation testing, calibration, and
    installation, amongst others.

    \ChapterRef{chap:daq} describes the data acquisition system developed and implemented for
    \chipsfive. I played a major role in the development, construction, testing, and installation
    of this system. Specifically, I made a significant contribution to the development of the
    networking solution, the high-level hardware implementation, the control and monitoring
    software, and the development of the novel low-level Madison \si{\micro}DAQ and Beaglebone
    system.

    \ChapterRef{chap:cnn} and \ChapterRef{chap:results} describe the application of Convolutional
    Neural Networks to the characterisation of neutrino events within \chipsfive. The work in both
    these chapters was solely conducted by me, representing the bulk of my efforts. I developed
    the complete pipeline, from event generation, simulation, model development and training, to
    comprehensive evaluation.
\end{preface}

\tableofcontents %%%%%%%%%%%%%%%%%%%%%%%%%%%%%%%%%%%%%%%%%%%%%%%%%%%%%%%%%%%%%%%%%%%%%%%%%%%%%%%%%