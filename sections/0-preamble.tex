\usepackage{xspace}
\usepackage{tikz}
\usepackage{mathrsfs}
\usepackage{verbatim}
\usepackage{cite}
\usepackage{graphicx}
\usepackage{caption}
\usepackage{subcaption}
\usepackage{url}
\usepackage{hyperref}

\usepackage{lmodern}  %% lmodern, mathpazo, euler
\usepackage[english]{babel}  %% Using Babel allows other languages to be used and mixed-in easily
\selectlanguage{english}

\usepackage[compat=1.1.0]{tikz-feynman}  %% Use the tikz-feynman package for Feynman diagram.

\makeatletter  %% Tweak so maths adapts its boldness to match the context.
\g@addto@macro\bfseries\boldmath
\makeatother

\usepackage{resources/abmath}
\usepackage{mathtools}
\DeclarePairedDelimiter\abs{\lvert}{\rvert}
\DeclarePairedDelimiter\norm{\lVert}{\rVert}

%% High-energy physics stuff
\usepackage{resources/abhep}
%\usepackage{resources/SIunits}
%\usepackage{hepnames}
%\usepackage{hepunits}

% Use \xspace at the end of the macros
% Use \text for text of units in macro
% write 5--10 for 5 to 10
% \mathrm{...} macro, i.e. $\mathrm{d}f/\mathrm{d}x$ → df /dx, not $df/dx$ → df /dx
% So put the whole of each mathematical expression in math mode (and drop out of it via \mathrm or \text if needed).
% Negative numbers should be written in math mode

%% Personal macros
\DeclareRobustCommand{\chips}{\textsc{Chips}\xspace}
\DeclareRobustCommand{\chipsm}{\textsc{Chips-M}\xspace}
\DeclareRobustCommand{\chipsfive}{\textsc{Chips-5}\xspace}
\DeclareRobustCommand{\nova}{NOvA\xspace}
\DeclareRobustCommand{\minos}{\textsc{Minos}\xspace}
\DeclareRobustCommand{\numi}{NuMI\xspace}
\DeclareRobustCommand{\genie}{\textsc{Genie}\xspace}
\DeclareRobustCommand{\root}{\textsc{Root}\xspace}
\DeclareRobustCommand{\tensorflow}{\textsc{Tensorflow}\xspace}
\DeclareRobustCommand{\python}{\textsc{Python}\xspace}
\DeclareRobustCommand{\google}{\textsc{Google}\xspace}

\DeclareRobustCommand{\arXivCode}[1]{arXiv:#1}
\DeclareRobustCommand{\CP}{\ensuremath{\mathcal{CP}}\xspace}
\DeclareRobustCommand{\CPviolation}{\CP-violation\xspace}
\DeclareRobustCommand{\CPv}{\CPviolation}
\DeclareRobustCommand{\LHC}{LHC\xspace}
\DeclareRobustCommand{\CERN}{CERN\xspace}
\DeclareRobustCommand{\bphysics}{\Pbottom-physics\xspace}
\DeclareRobustCommand{\bhadron}{\Pbottom-hadron\xspace}
\DeclareRobustCommand{\Bmeson}{\PB-meson\xspace}
\DeclareRobustCommand{\bbaryon}{\Pbottom-baryon\xspace}
\DeclareRobustCommand{\Bdecay}{\PB-decay\xspace}
\DeclareRobustCommand{\bdecay}{\Pbottom-decay\xspace}
\DeclareRobustCommand{\BToKPi}{\HepProcess{ \PB \to \PK \Ppi }\xspace}
\DeclareRobustCommand{\BToPiPi}{\HepProcess{ \PB \to \Ppi \Ppi }\xspace}
\DeclareRobustCommand{\BToKK}{\HepProcess{ \PB \to \PK \PK }\xspace}
\DeclareRobustCommand{\BToRhoPi}{\HepProcess{ \PB \to \Prho \Ppi }\xspace}
\DeclareRobustCommand{\BToRhoRho}{\HepProcess{ \PB \to \Prho \Prho }\xspace}
\DeclareRobustCommand{\X}{\thesismath{X}\xspace}
\DeclareRobustCommand{\Xbar}{\thesismath{\overline{X}}\xspace}
\DeclareRobustCommand{\Xzero}{\HepGenParticle{X}{}{0}\xspace}
\DeclareRobustCommand{\Xzerobar}{\HepGenAntiParticle{X}{}{0}\xspace}
\DeclareRobustCommand{\epluseminus}{\Ppositron\!\Pelectron\xspace}
\DeclareRobustCommand{\protonproton}{\Pproton\APantiproton\xspace}