\chapter{Neutrino oscillations: theoretical background and current status}
\label{chap:theory}

\begin{comment}
TODO: Write the story of the theory chapter

TODO: Write the theory chapter section outline

TODO: Add all the sections below

### Historical background (~1500 words)

Proposal of a mysterious undetector particle to explain beta decays in the 1930s through to the resolutions of a 30-year problem with the confirmation of oscillations in the early 2000s and onto the precision era.
Neutrino oscillations first discoveed in 1957 when Bruno Pornecorvo proposed a model in which neutrinos oscilate to antineutrinos and back, similar to the kain. It was actually shown that neutrinos iscilate from one flavour to another.
Wolfgang Pauli first pusoluated the neutrino in 1931 to account for the non-conservation of energy/angular momentum in beta decays. A simple two-body decay at rest should produce an electron carrying away a fixed kinetci energy corresponding to the difference in binding energies of the nuclei, but a continuos spectrum was observed. Therefore, he postulated the extistnse of a unseen second particle, alwso emmited in the decay which was undetected. This must be a neutral spin-1/2 fermion, and he named it the 'neutron'
Was named neutrino after the neutron was discovered.
The antineutrino was discovered by Reines and Cowan in 1956, using inverse beta decay in a 200l tank of water, detecting neutrinos from the Los Alamos nuclear reactor.
Muon neutrino discovered by the 'long track' from the decays of pions from a reactor in 1988, got a nobel prize.
DONUT finally found the tau neutrino in 2000 using 800GeV protons from the Tevatron.
Hence, all flavours of neutrino had been found, but the strongest constraint is due to the width of the Z^0 resonance. THis indicates the number of active neutrino states can only be 1.984+-0.008. Therefore, any as yet undiscovered neutrinos must be sterile, in that they do not couple to the weak interaction.
Cosmology can also constrain the number of active neutrino with a value about 3.3+-0.27, the planck satellite did this.

- History of neutrins physics starts in 2930 when Wolfgang Pauli prposoed their existance to solve the problem of energy and angular momentum conservation in beta decay.
- The paradox could be solves if a third particle, electrically neutral with low mass and spin 1/2 was also created in the decay. This is now known as the electron neutrino.
- Fermi soon developed the first theory of weak inteactions and in 2934, Bethe and Peierls estimated the cross-sections for the inverse beta decay process in which the neutrino could be observed. With such a low result, it showed the enormous channenge that experimental neutrinp hysics would face, required very large intensitied and detector volumes to observed significatn event rates.
- Cowan andReines 20 years later found the electron antineutrino, produced in the core of a nuclea reactor and Savannah river experiment. 200 litres of cadmium doped water as the target and 1400 litres of liquid scintillator with 100 PMTs, showed the neutrinos detector from inverse beta decay.
- In 1962 at the alternating gradient synchrotron at Brookhaven, neutrinos created from pion decays together with muons were observed t produce only muons not electrons, this then confirmed the existence of the muon neutrino.
- In 1973, the garagamelle experiment at cern using an accelerator produced neutrino beam and a big bubble chamber discivered the existence of wek neutral current interactions Proving the existance of the neutral z-boson and confirming the Glashow-salam-weingberg theory of electroweak interactions.
- Experiments at the LEP e+e- collider in the 1990s made precision measurmnets of the z decay width, from a fir to the data in showed there are exactly three active generations of neutrinos.
- In 2000 the DONUT experiment at the tevatron collider in fermilab performned a direct detection of the tau neutrino completing the three flavour picture.

- In the solar neutrino sector there is the "solar anomaly" noting a deficit of electron neutrino compared to predications made by the standard solar model (SSM)
- First observed at the Homestack experiment, neutrinos ineracted with the chlorine creating radioactive argon atoms, beause it is a noble gas it does not bing to the perchloroethylene and it can be extracted by purging the liquid with gaseous helium and then extracted from the helium with a cooled carbon trap.
- Gallium was also used by other experiments and kamiokande also observed the deficit.
- Also the fluxes measured where not consistent, depending on the energy range probed. Hinting at oscillations dependent on energy,
- SNO finally answered the question when it was able to measure three channels with different relation between te flyx or electron neutrinos and the other neutrinos. SNO could prove that the electron neutrinos are changing flavour. WHile the total flux of all neutrinos remains constant and in agrrement with the SSM.


### Theory

- Proposal of a mysterious undetector particle to explain beta decays in the 1930s through to the resolutions of a 30-year problem with the confirmation of oscillations in the early 2000s and onto the precision era.
- Neutrino oscillations first discoveed in 1957 when Bruno Pornecorvo proposed a model in which neutrinos oscilate to antineutrinos and back, similar to the kain. It was actually shown that neutrinos iscilate from one flavour to another.
- Wolfgang Pauli first pusoluated the neutrino in 1931 to account for the non-conservation of energy/angular momentum in beta decays. A simple two-body decay at rest should produce an electron carrying away a fixed kinetci energy corresponding to the difference in binding energies of the nuclei, but a continuos spectrum was observed. Therefore, he postulated the extistnse of a unseen second particle, alwso emmited in the decay which was undetected. This must be a neutral spin-1/2 fermion, and he named it the 'neutron'
- Was named neutrino after the neutron was discovered.
- The antineutrino was discovered by Reines and Cowan in 1956, using inverse beta decay in a 200l tank of water, detecting neutrinos from the Los Alamos nuclear reactor.
- Muon neutrino discovered by the 'long track' from the decays of pions from a reactor in 1988, got a nobel prize.
- DONUT finally found the tau neutrino in 2000 using 800GeV protons from the Tevatron.
- Hence, all flavours of neutrino had been found, but the strongest constraint is due to the width of the Z^0 resonance. THis indicates the number of active neutrino states can only be 1.984+-0.008. Therefore, any as yet undiscovered neutrinos must be sterile, in that they do not couple to the weak interaction.
- Cosmology can also constrain the number of active neutrino with a value about 3.3+-0.27, the planck satellite did this.

- As the standard model of particle physics was developed, neutrinos were presumed to be massless and occur only in the three flavour eigenstates.
- Various hints that this was not the case kept appearing, leading to neutrino oscillations, by witch one neutrino can oscillate to another flavour and the non-zero masses that follow as a direct consiquence from this.
- The solar neutrino problem:
- The sun produces a rich source of electron neutrinos
- Various processes cobine to form the PP-chain, which overall fuse four protons to produce a helium necleus.
- 4p+ 2e−→4He + 2νe+ 26.73MeV−Eν
- whereEνis the energy of the two neutrinos, and (26.73 MeV -Eν) is the energyemitted as photons
- In the 1960's at the Homestake mine, a 100000 gallon tank of perchloroethylene, dry-cleaning fluid, detected neutrinos from the boron-8 decay via the CC neutrino capture reaction νe+37Cl→37Ar +e−(threshold 814 keV) with the tank periodically flushed with helium to remove and measure the amount of argon produced and hence the number of neutrinos detected.
- A discrepancy was seen between the SNU (solar neutrino unit), number of interactions per cl37 per second and that predicted by theory.
- Inital measurements was of 3.0 SNU compared with predctions ranging from 4.4 o 21 SNU
- Final homestake measurement in the 1990s showd 2.56+_0.16(stat)+_0.16(syst) compared with 6.4-9.3 depending on model paramters from theory.
- The kamiokande-11 water cherenkov detector also observed this deficit. Which was also capable to inferring the neutrinos directions and showing that they originated from the sun. Measuring a value of 0.46 that of the SSM prediction.
- Both SAGE and GALLEX also reported the deficit.
- Neutrino oscillations were one wasy of explaining this deficit. In the electron neutrinos were convertd to muon or tau neutrinos, the electrons in the detector medium would not be able to capture them, additionally solar neutrinos has insufficienct energy to produce muons or taus which would allow their detection in a water cherenkov detector.
- The SNO experiment finally demonstrated that this was the case in 2001-2002.
- A 1kton tank of heavy water, containing h^2 isotopes, was able to detect solar neutrinos via three different channels νe+d→p+p+e−, νx+d→p+n+νx and νx+e−→νx+e− where d in the heavy deuterium nucelus. The first channel is a CC interaction only sensitive to electron neutrinos. The second is an NC interactions and is sensitive to neutrinos of any flavour, detected via the gamma rays produced when the dislodged neutron thermalises and is captured by another nucelous. The final channel is elastic scattering of the neutrino from an electron and is again sensitive to all three neutrino flavours.
- By comparing the fluxes measured with the three channels and deividing them into the component attributed to electron neutrinos. SNO demonstrated a 5.3 sigma singal of oscillations from electrons neutrinos to the other flavours.
- Oscillations in the atmospheric sector:
-

- Neutrino was first proposed to explain the continuous energy spectrum of the electrons observed in beta-decay.
- Fermi proposed a more complete model of beta decay in 1934, also proposing that the mass could be measured by looking at the end point of the beta decay spectrum. The first experiment set an upper limit of 500ev which was imporved to 250 ev, with the idea that hthe neutrino was massless gaining traction.
- The first directr obervation of neutrino was in 1956, and in 1962 proof that the electron and muon neutrinos are distinct particle. Found by observing that is was far more likely that the neutrinos produced in the decay of pions would interact to create muons and opposed to electrons. Meaning there had to be two flavour of neutrinos as if there was a single flavour it should produce equal number sof electron and muons.
- The homstake experiment observed the solar neutrino problem

## From eV to EeV: Neutrino Cross-Sections Across Energy Scales

- Neutrino originally postulated by Wolfgang Pauli in 1930, and has played a prominent role in understanding of nuclear and particle physics.
- The revalation that neutrinos can no longer be massless is perhaps the first significant alteration to the standard model.
- A nice plot of neutrino energy regimes from Big Bang through accelerator to Extra-galactic vs their cross-sections. I could definitely use this and cite it to the paper
- Contains a good description of fundamental electroweak scattering if we need a reference for that (page 4->)
- First anumu + e- => anumu + e- scattering made by CERN bubble cchamber experiment Gargamelle
- This and DIS NC observations confirmed the weak neutral currents and helped solidify the standard model.
- Maybe include a bubble chamber image of the first candidate neutrino interaction.
- At intermediate energy scales (CHIPS range) interactions fall into three main catageories. Elastic and quasi-elastic scattering, resonance production and DIS.
- Include an actual data cross-section plot for both CC and NC showing contributions from different experiments
- Show the tau-neutrino cross section compared to the muon/electron in our energy range to show it doesn't matter. All the interactions and arguments that go along with them are the same for nuel/numu as with nutau, except for one key difference; the energy threshold. The nutau interaction CC cross section is severely altered because of the large tau lepton mass. Then show the plot.
- Bellow 2Gev it's mainly quasi-elastic with the neutrino scattering of the entire nucelon, rather than its consituent partons.
- Modern experiments MiniBooNE and NOMAD see higher absolute cross-sections than expected. It is currently believed that nucelear effects beyond the impulse approximation are desponsible for the discrepancy. Such as nucleon-nucleon correlations and two-body exchange currents must be included to get it righ. THIS IS MEC!
- NC QEL lots of people call NC Elastic Scattering, the ratio of NC Elastic/CC QE is ~0.11 from measurements by a few experiments.
- Single pion production is when the neutrino excites the struck nucleon producing a baryon resonance, which then quickly decays most often into a nucleon and a single pion final state. There are seven possible single pion channels, 3CC and 4NC, which we see from the GENIE events.
- Show all the interaction equations for these νμp→μ−pπ+ etc...
- NC pi-zero production is often the largest numu-induced backgrond in experiments searching for numu->nuel oscillations. And CC pi production can present a non-negligable complication in the determination of neutrino energy in experiments. Therefore measuring and modelling nuclear effects in pion production has become paramount.
- These resonances can also decay into photons with a small branchng fraction, yes, but, like NC pi-zero production they still pose a non-negliable source of background to the CHIPS main search.
- Neutrinos can also coherently produce single pion final state. In this case the neutrino coherently scatters from the entire nucleus transferring negligable energy to the target. Hence, you produce a ditinctly forward-scattered pion with no nuclear recoil. This process is relatively small however.
- The resonances can also decay to multi-pion final state, along with DIS this contributes a copious source of multi-pion final states. However, due to the inherant complexity of reconstructing multiple pion final states, not many experiments look at these cross-sections.
- You can also get kaon production but they have small cross-sections due to the kain mass and because kaon channels are not enhanced by any dominant resonance.
- You then get DIS where the neutrino scatters of a quark in the nucleon via the exchange of a virtual W or Z boson producing a lepton and a hadronic system in the final state.
- To isolate DOS events experiments typically apply kinematic cuts to remove QE scattering and resonance-mediated contributions from their data.


## Visualizing Data using t-SNE

- There have been plently of attempts at visualising high-dimensional data on a 2/3 dimensional map, including Sammon mapping, Isomap, Locally Linear Embedding, Stochastic Neighbour Embedding.
- Older implementations tended to cluster all data points towards the centre of the map and proved difficult to optimise.
- You basically set a summed probability between all points in the low-dimensional space to a summed probability between all points in the high-dimensional space.
- t-SNE uses the student-t distribution (with a heavy tail) in the low-dimensional space to calculate the probability. This alleviates both the crowding problem and is easier to optimise.
- Optimisation uses a simple momentum term, plus two new ideas. "Early compression" which forces map points to stay close to each other at the start of optimisation, it is then easier for clusters to move through each other and explore all possible global organisations of the data, this is implemented as an L2-penalty term proportional to the sum of squared distances from the origin, this is then removed at an iteration given as input. Secondly, "Early exaggeration" which creates tight widely seperated clusters.

\end{comment}

%%%%%%%%%%%%%%%%%%%%%%%%%%%%%%%%%%%%%%%%%%%%%%%%%%%%%%%%%%%%%%%%%%%%%%%%%%%%%%%%%%%%%%%%%%%%%%%%%%%%%%%%%%%%%%%%%%
%                                                 INTRODUCTION                                                   %
%%%%%%%%%%%%%%%%%%%%%%%%%%%%%%%%%%%%%%%%%%%%%%%%%%%%%%%%%%%%%%%%%%%%%%%%%%%%%%%%%%%%%%%%%%%%%%%%%%%%%%%%%%%%%%%%%%
\section{Introduction}
\label{sec:theory_intro}
Consider a simple two body decay
Neutrino physics covers the widest possible range of
Proposal of a mysterious undetector particle to explain beta decays in the 1930s through to the resolutions of a 30-year problem with the confirmation of oscillations in the early 2000s and onto the precision era.
Neutrino oscillations first discoveed in 1957 when Bruno Pornecorvo proposed a model in which neutrinos oscilate to antineutrinos and back, similar to the kain. It was actually shown that neutrinos iscilate from one flavour to another.
The field of neutrino physics is ever expanding with a new generation of experiments planned for the coming years.
This chapter aims to provide an introduction to neutrino

%%%%%%%%%%%%%%%%%%%%%%%%%%%%%%%%%%%%%%%%%%%%%%%%%%%%%%%%%%%%%%%%%%%%%%%%%%%%%%%%%%%%%%%%%%%%%%%%%%%%%%%%%%%%%%%%%%
%                                                   HISTORY                                                      %
%%%%%%%%%%%%%%%%%%%%%%%%%%%%%%%%%%%%%%%%%%%%%%%%%%%%%%%%%%%%%%%%%%%%%%%%%%%%%%%%%%%%%%%%%%%%%%%%%%%%%%%%%%%%%%%%%%
\section{A history of neutrino oscillations}
\label{sec:theory_history}

DIAGRAM: Maybe add a few diagrams of the key plots throughout history

In the early 20th century, beta decays were assumed to follow the simple two-body process, $A \rightarrow B + e$, where a
nuclei spontaneously an electron, and only an electron. To conserve both energy and angular momentum the ejected electron
must have a discrete kinetic energy defined by the difference in binding energies between the initial and final nuclei.
However, in 1914, J. Chadwick instead measured a continuous energy spectrum for the electron ~\cite{chadwick1914},
placing this theory in doubt.

W. Pauli finally proposed a 'desperate solution' to this paradox in 1930 ~\cite{pauli1930}. If a light, neutrally charged,
spin $1/2$ particle was also produced in the interaction, the continuous energy distribution could be explained. Initially
this mysterious new particle was named the 'neutron'. But, to avoid confusion with the heavy baryon of the same name discovered
in 1932, E. Fermi renamed it the 'neutrino' when he formalised beta decay in 1934 ~\cite{fermi1934}.

The following month, H. Bethe and R. Peierls used Fermi's work to estimate the cross-section of the inverse beta decay process
${\nu} + p^{+} \rightarrow n + e^{+}$ ~\cite{bethe1934}. They calculated a value of less than the very small $10^{-44} cm^2$
and declared 'there is no practically possible way of observing the neutrino.' Although extensive neutrino detection has
proved possible, it hinted at the huge difficulties experimentalists would face hunting down the neutrino and measuring
it's properties in the years to come.

After an initial tentative identification if 1953, F. Reines and C. Cowan made the first confirmed observation of the neutrino
in 1956 ~\cite{cowan1956}. Electron antineutrinos produced in the Savannah River Plan nuclear reactor were detected via the
inverse decay process outlined in the previous paragraph. A 'club-sandwich' detector of three 1500 litre liquid scintillator tanks
and two 200 litre cadmium doped water target tanks, was constructed in an underground room of the reactor building. A total of
330 photomultiplier tubes were then able to measure the prompt positron annihilation signal followed by the gamma ray burst from
the neutron capture in cadmium, the signature identification for the interaction.

Brookhaven two kinds of neutrinos in Ref.~\cite{danby1962}
- Muon neutrino discovered by the 'long track' from the decays of pions from a reactor in 1988, got a nobel prize.
- In 1962 at the alternating gradient synchrotron at Brookhaven, neutrinos created from pion decays together with muons were observed t produce only muons not electrons, this then confirmed the existence of the muon neutrino.
- Neutrinos originating from pion decays primary produce muons, not electron. Detected as single long tracks in a spark chamber.
- Got the 1988 nobel prize for this discovery of the second neutrino.
- Distinct from the previously known electron neutrino
- The neutrinos produced by the pions decay from a accelerator beam, were not the same as the neutrinos observed in beta decay
- Did so by observing that it was far more likely that the neutrinos pro-duced in the decay of pions would interact to create muons, as opposed to electron
- First experiment to construct and use an artificial neutrino beam
%- νμ+p+→μ++n and ̄νμ+p+→e++n with a single neutrino would be expected to happen at the same rate, only muons were produced.
- 34 identified muon events in total no electrons.

Discovery of the tau lepton ~\cite{perl1975}
Also precise z-resonance measurement at lep in the 1990's ~\cite{electroweak2006}
Finally measured by DONUT in 2000 ~\cite{Kodama2001}
- evidence for the third neutrino, finally discovered at DONUT in 2000
- After the discovery of the tau lepton in 1975 ~\cite{perl1975}, this suggested the existence of the third neutrino which DONUT found in 2000.
- DONUT finally found the tau neutrino in 2000 using 800GeV protons from the Tevatron.
- In 2000 the DONUT experiment at the tevatron collider in fermilab performned a direct detection of the tau neutrino completing the three flavour picture.
- Experiments at the LEP e+e- collider in the 1990s made precision measurmnets of the z decay width, from a fir to the data in showed there are exactly three active generations of neutrinos.
- This indicates the number of active neutrino states can only be 1.984+-0.008. Therefore, any as yet undiscovered neutrinos must be sterile, in that they do not couple to the weak interaction.
%- Hence, all flavours of neutrino had been found, but the strongest constraint is due to the width of the Z^0 resonance. 

Homestake deficit observation in Ref.~\cite{davis1968}
first SSU predictions used to compare against homestake in Ref.~\cite{bahcall1968}
Kamiokande II deficit in Ref.~\cite{hirata1989}
SAGE experiment deficit in Ref.~\cite{abazov1991}
GALLEX experiment deficit in Ref.~\cite{anselmann1994}
SSM Prediction for Ga in Ref.~\cite{bahcall1988}
- As the standard model of particle physics was developed, neutrinos were presumed to be massless and occur only in the three flavour eigenstates.
- Various hints that this was not the case kept appearing, leading to neutrino oscillations, by witch one neutrino can oscillate to another flavour and the non-zero masses that follow as a direct consiquence from this.
- In the solar neutrino sector there is the "solar anomaly" noting a deficit of electron neutrino compared to predications made by the standard solar model (SSM)
- First observed at the Homestack experiment, neutrinos ineracted with the chlorine creating radioactive argon atoms, beause it is a noble gas it does not bing to the perchloroethylene and it can be extracted by purging the liquid with gaseous helium and then extracted from the helium with a cooled carbon trap.
- Gallium was also used by other experiments and kamiokande also observed the deficit.
- Also the fluxes measured where not consistent, depending on the energy range probed. Hinting at oscillations dependent on energy,
%- 4p+ 2e−→4He + 2νe+ 26.73MeV−Eν pp-chain in the sun whereEνis the energy of the two neutrinos, and (26.73 MeV -Eν) is the energyemitted as photons
%- νe+37Cl→37Ar +e−(threshold 814 keV) in homestake, periodically flushed with helium
- 400 000 litresof perchloroethylene (a dry-cleaning fluid), containing520 tof chlorine, placed in the Homestake Mine,1.5 kmunderground [24].
%- chlorine solution capable of neutrino capture(νe+37Cl→37Ar + e−).  The atoms of argon were counted and used as a measure of the neutrinoflux.
- he reported experimental rate was about two thirds less than what was expected from theStandard Solar Model (SSM). This large discrepancy, known as thesolar  neutrino  problem, wasinitially believed to be an experimental flaw.
% ν+Ga71→Ge71+e− for sage and gallex
- This is where the future DUNE detector will be housed, nice full circle
- This is in the solar sector

SNO oscillation measurement in Ref.~\cite{ahmad2002}
- neutrino oscillations were one way of explaining this deficit if some of the electron neutrinos converted flavour in flight.
- SNO finally answered the question when it was able to measure three channels with different relation between te flyx or electron neutrinos and the other neutrinos. SNO could prove that the electron neutrinos are changing flavour. WHile the total flux of all neutrinos remains constant and in agrrement with the SSM.
- 1kton tank of heavy (D2O deuterium) water, able to detect three different channels of neutrino interaction
%- νe+d→p+p+e− νx+d→p+n+νx νx+e−→νx+e−
- Cherenkov experiment, with 9500 8inch photomultiplier tubes detectro the light from neutrino interactions.
- Since each of the rates for the three channels has a different relation between the flux of electron neutrinos and the others, SNO could confirm electron neutrinos are changing flavour, with the total flux being constant and in agreement with the SSM.
- electron neutrino CC, NC and elastic scattering also.
- total rate was consistent but less electron neutrinos than expected as they had oscillated.
- However, only electron neutrinos canundergo CC interactions, as solar neutrinos do not have enough energy to produce muonor tau leptons.
- %e+d→p+p+e−(CC) νx+d→p+n+νx(NC) νx+e−→νx+e−(ES) The CC channel is sensitive exclusively to electron neutrinos, whilst the other two are acces-sible by neutrinos of any flavour. 

Atmospheric kamiokande deficit in Ref.~\cite{hirata1988}
IMD detector atmospheric deficit in Ref.~\cite{becker1992}
Superkamiokande direction atmospheric neutrinos in Ref.~\cite{becker1992}
- This is in the atmospheric sector

%%%%%%%%%%%%%%%%%%%%%%%%%%%%%%%%%%%%%%%%%%%%%%%%%%%%%%%%%%%%%%%%%%%%%%%%%%%%%%%%%%%%%%%%%%%%%%%%%%%%%%%%%%%%%%%%%%
%                                            OSCILLATION THEORY                                                  %
%%%%%%%%%%%%%%%%%%%%%%%%%%%%%%%%%%%%%%%%%%%%%%%%%%%%%%%%%%%%%%%%%%%%%%%%%%%%%%%%%%%%%%%%%%%%%%%%%%%%%%%%%%%%%%%%%%
\section{Neutrino oscillation theory}
\label{sec:theory_theory}

EQUATION: Basic neutrino oscillation equations
EQUATION: Neutrino oscillations in matter equations
EQUATION: Matter effect equations
- We look at the oscillation structure across a range of energies in our detectors to measure the oscilaltion parameters.

%%%%%%%%%%%%%%%%%%%%%%%%%%%%%%%%%%%%%%%%%%%%%%%%%%%%%%%%%%%%%%%%%%%%%%%%%%%%%%%%%%%%%%%%%%%%%%%%%%%%%%%%%%%%%%%%%%
%                                       CURRENT STATUS AND THE FUTURE                                            %
%%%%%%%%%%%%%%%%%%%%%%%%%%%%%%%%%%%%%%%%%%%%%%%%%%%%%%%%%%%%%%%%%%%%%%%%%%%%%%%%%%%%%%%%%%%%%%%%%%%%%%%%%%%%%%%%%%
\section{Current status and the future}
\label{sec:theory_status}

\subsection{Atmospheric}

\subsection{Accelerator experiments}

\subsection{Reactor experiments}

\subsection{The future}