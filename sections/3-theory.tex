\chapter{Neutrino oscillations: theoretical background and current status}
\label{chap:theory}

\chapterquote{Dear Radioactive Ladies and Gentlemen.}
{Wolfgang Pauli, 1930}


TO MAKE
    - cross-section protection from new line
    - all nu and nu bar combos
    - proton, neutron, electron, muon, tau etc
    - 

Consider a simple two body decay 

Neutrino physics covers the widest possible range of 

Proposal of a mysterious undetector particle to explain beta decays in the 1930s through to the resolutions of a 30-year problem with the confirmation of oscillations in the early 2000s and onto the precision era.

Neutrino oscillations first discoveed in 1957 when Bruno Pornecorvo proposed a model in which neutrinos oscilate to antineutrinos and back, similar to the kain. It was actually shown that neutrinos iscilate from one flavour to another.


The field of neutrino physics is ever expanding with a new generation of experiments planned for the coming years. 

This chapter aims to provide an introduction to neutrino 

blah blah blah

\section{A history of neutrino oscillations}
\label{sec:theoryhistory}

In the early 20th century, beta decays were assumed to follow the simple two-body process, $A \rightarrow B + e$, where a
nuclei spontaneously emits an electron. To conserve both energy and angular momentum the ejected electron should have a discrete 
kinetic energy related to the difference in mass between the initial and resultant nuclei. However, in 1914, J. Chadwick 
instead measured a continuos energy spectrum for the electron ~\cite{chadwick1914}, placing this theory in doubt.

W. Pauli finally proposed a 'desperate solution' in 1930 ~\cite{pauli1930}. If a light, neutrally charged, spin $1/2$ particle
was also produced in the decay, the continuos energy spectrum could be explained. Initially this mysterious new particle was
named the 'neutron'. But, to avoid confusion with the heavy baryon of the same name discovered in 1932, E. Fermi renamed it 
the 'neutrino' when he formalised beta decay in 1934 ~\cite{fermi1934}.

Using Fermi's work H. Bethe and R. Peierls estimated the cross-section of the inverse beta decay process
${\nu} + p^{+} \rightarrow n + e^{+}$ ~\cite{bethe1934}. They calculated a value of less than the very small $10^{-44} cm^2$ and
declared 'there is no practically possible way of observing the neutrino.' Although extensive neutrino detection has proved
possible it hinted at the huge difficulties experimentalists would face hunting down the neutrino and measuring it's properties
in the years to come.

The first detection of the neutrino came in 1956 by F. Reines and C. Cowan ~\cite{cowan1956}. A sandwich of 


The antineutrino was discovered by Reines and Cowan in 1956, using inverse beta decay in a 200l tank of water, detecting neutrinos from the Los Alamos nuclear reactor.

Muon neutrino discovered by the 'long track' from the decays of pions from a reactor in 1988, got a nobel prize.

DONUT finally found the tau neutrino in 2000 using 800GeV protons from the Tevatron.

%Hence, all flavours of neutrino had been found, but the strongest constraint is due to the width of the Z^0 resonance. 
THis indicates the number of active neutrino states can only be 1.984+-0.008. Therefore, any as yet undiscovered neutrinos must be sterile, in that they do not couple to the weak interaction.

Cosmology can also constrain the number of active neutrino with a value about 3.3+-0.27, the planck satellite did this.

As the standard model of particle physics was developed, neutrinos were presumed to be massless and occur only in the three flavour eigenstates.

Various hints that this was not the case kept appearing, leading to neutrino oscillations, by witch one neutrino can oscillate to another flavour and the non-zero masses that follow as a direct consiquence from this.



Cowan andReines 20 years later found the electron antineutrino, produced in the core of a nuclea reactor and Savannah river experiment. 200 litres of cadmium doped water as the target and 1400 litres of liquid scintillator with 100 PMTs, showed the neutrinos detector from inverse beta decay.

In 1962 at the alternating gradient synchrotron at Brookhaven, neutrinos created from pion decays together with muons were observed t produce only muons not electrons, this then confirmed the existence of the muon neutrino.

In 1973, the garagamelle experiment at cern using an accelerator produced neutrino beam and a big bubble chamber discivered the existence of wek neutral current interactions Proving the existance of the neutral z-boson and confirming the Glashow-salam-weingberg theory of electroweak interactions.

Experiments at the LEP e+e- collider in the 1990s made precision measurmnets of the z decay width, from a fir to the data in showed there are exactly three active generations of neutrinos.

In 2000 the DONUT experiment at the tevatron collider in fermilab performned a direct detection of the tau neutrino completing the three flavour picture.

In the solar neutrino sector there is the "solar anomaly" noting a deficit of electron neutrino compared to predications made by the standard solar model (SSM)

First observed at the Homestack experiment, neutrinos ineracted with the chlorine creating radioactive argon atoms, beause it is a noble gas it does not bing to the perchloroethylene and it can be extracted by purging the liquid with gaseous helium and then extracted from the helium with a cooled carbon trap.

Gallium was also used by other experiments and kamiokande also observed the deficit.

Also the fluxes measured where not consistent, depending on the energy range probed. Hinting at oscillations dependent on energy,

SNO finally answered the question when it was able to measure three channels with different relation between te flyx or electron neutrinos and the other neutrinos. SNO could prove that the electron neutrinos are changing flavour. WHile the total flux of all neutrinos remains constant and in agrrement with the SSM.


Brookhaven two kinds of neutrinos in Ref.~\cite{danby1962}

Gargamelle neutral-currents in Ref.~\cite{hasert1974}

Observation of the Tau neutrino in Ref.~\cite{Kodama2001}

Precise measurement of z-resonance for three neutrino in Ref.~\cite{electroweak2006}

Homestake deficit observation in Ref.~\cite{davis1968}

first SSU predictions used to compare against homestake in Ref.~\cite{bahcall1968}

Kamiokande II deficit in Ref.~\cite{hirata1989}

SAGE experiment deficit in Ref.~\cite{abazov1991}

GALLEX experiment deficit in Ref.~\cite{anselmann1994}

SSM Prediction for Ga in Ref.~\cite{bahcall1988}

SNO oscillation measurement in Ref.~\cite{ahmad2002}

Atmospheric kamiokande deficit in Ref.~\cite{hirata1988}

IMD detector atmospheric deficit in Ref.~\cite{becker1992}

Superkamiokande direction atmospheric neutrinos in Ref.~\cite{becker1992}

\section{Neutrino oscillation theory}
\label{sec:theorytheory}

blah blah blah

\section{Current status and the future}
\label{sec:theorystatus}

blah blah blah