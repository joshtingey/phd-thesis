\chapter{Neutrino oscillations: theoretical background and current status}
\label{chap:theory}

Consider a simple two body decay
Neutrino physics covers the widest possible range of
Proposal of a mysterious undetector particle to explain beta decays in the 1930s through to the resolutions of a 30-year problem with the confirmation of oscillations in the early 2000s and onto the precision era.
Neutrino oscillations first discoveed in 1957 when Bruno Pornecorvo proposed a model in which neutrinos oscilate to antineutrinos and back, similar to the kain. It was actually shown that neutrinos iscilate from one flavour to another.
The field of neutrino physics is ever expanding with a new generation of experiments planned for the coming years.
This chapter aims to provide an introduction to neutrino

\section{A history of neutrino oscillations}
\label{sec:theoryhistory}

\begin{comment}
TODO: Maybe add a few diagrams of the key plots throughout history
\end{comment}

In the early 20th century, beta decays were assumed to follow the simple two-body process, $A \rightarrow B + e$, where a
nuclei spontaneously an electron, and only an electron. To conserve both energy and angular momentum the ejected electron
must have a discrete kinetic energy defined by the difference in binding energies between the initial and final nuclei.
However, in 1914, J. Chadwick instead measured a continuous energy spectrum for the electron ~\cite{chadwick1914},
placing this theory in doubt.

W. Pauli finally proposed a 'desperate solution' to this paradox in 1930 ~\cite{pauli1930}. If a light, neutrally charged,
spin $1/2$ particle was also produced in the interaction, the continuous energy distribution could be explained. Initially
this mysterious new particle was named the 'neutron'. But, to avoid confusion with the heavy baryon of the same name discovered
in 1932, E. Fermi renamed it the 'neutrino' when he formalised beta decay in 1934 ~\cite{fermi1934}.

The following month, H. Bethe and R. Peierls used Fermi's work to estimate the cross-section of the inverse beta decay process
${\nu} + p^{+} \rightarrow n + e^{+}$ ~\cite{bethe1934}. They calculated a value of less than the very small $10^{-44} cm^2$
and declared 'there is no practically possible way of observing the neutrino.' Although extensive neutrino detection has
proved possible, it hinted at the huge difficulties experimentalists would face hunting down the neutrino and measuring
it's properties in the years to come.

After an initial tentative identification if 1953, F. Reines and C. Cowan made the first confirmed observation of the neutrino
in 1956 ~\cite{cowan1956}. Electron antineutrinos produced in the Savannah River Plan nuclear reactor were detected via the
inverse decay process outlined in the previous paragraph. A 'club-sandwich' detector of three 1500 litre liquid scintillator tanks
and two 200 litre cadmium doped water target tanks, was constructed in an underground room of the reactor building. A total of
330 photomultiplier tubes were then able to measure the prompt positron annihilation signal followed by the gamma ray burst from
the neutron capture in cadmium, the signature identification for the interaction.


Brookhaven two kinds of neutrinos in Ref.~\cite{danby1962}
- Muon neutrino discovered by the 'long track' from the decays of pions from a reactor in 1988, got a nobel prize.
- In 1962 at the alternating gradient synchrotron at Brookhaven, neutrinos created from pion decays together with muons were observed t produce only muons not electrons, this then confirmed the existence of the muon neutrino.
- Neutrinos originating from pion decays primary produce muons, not electron. Detected as single long tracks in a spark chamber.
- Got the 1988 nobel prize for this discovery of the second neutrino.
- Distinct from the previously known electron neutrino
- The neutrinos produced by the pions decay from a accelerator beam, were not the same as the neutrinos observed in beta decay
- Did so by observing that it was far more likely that the neutrinos pro-duced in the decay of pions would interact to create muons, as opposed to electron
- First experiment to construct and use an artificial neutrino beam
%- νμ+p+→μ++n and ̄νμ+p+→e++n with a single neutrino would be expected to happen at the same rate, only muons were produced.
- 34 identified muon events in total no electrons.


Discovery of the tau lepton ~\cite{perl1975}
Also precise z-resonance measurement at lep in the 1990's ~\cite{electroweak2006}
Finally measured by DONUT in 2000 ~\cite{Kodama2001}
- evidence for the third neutrino, finally discovered at DONUT in 2000
- After the discovery of the tau lepton in 1975 ~\cite{perl1975}, this suggested the existence of the third neutrino which DONUT found in 2000.
- DONUT finally found the tau neutrino in 2000 using 800GeV protons from the Tevatron.
- In 2000 the DONUT experiment at the tevatron collider in fermilab performned a direct detection of the tau neutrino completing the three flavour picture.
- Experiments at the LEP e+e- collider in the 1990s made precision measurmnets of the z decay width, from a fir to the data in showed there are exactly three active generations of neutrinos.
- This indicates the number of active neutrino states can only be 1.984+-0.008. Therefore, any as yet undiscovered neutrinos must be sterile, in that they do not couple to the weak interaction.
%- Hence, all flavours of neutrino had been found, but the strongest constraint is due to the width of the Z^0 resonance. 


Homestake deficit observation in Ref.~\cite{davis1968}
first SSU predictions used to compare against homestake in Ref.~\cite{bahcall1968}
Kamiokande II deficit in Ref.~\cite{hirata1989}
SAGE experiment deficit in Ref.~\cite{abazov1991}
GALLEX experiment deficit in Ref.~\cite{anselmann1994}
SSM Prediction for Ga in Ref.~\cite{bahcall1988}
- As the standard model of particle physics was developed, neutrinos were presumed to be massless and occur only in the three flavour eigenstates.
- Various hints that this was not the case kept appearing, leading to neutrino oscillations, by witch one neutrino can oscillate to another flavour and the non-zero masses that follow as a direct consiquence from this.
- In the solar neutrino sector there is the "solar anomaly" noting a deficit of electron neutrino compared to predications made by the standard solar model (SSM)
- First observed at the Homestack experiment, neutrinos ineracted with the chlorine creating radioactive argon atoms, beause it is a noble gas it does not bing to the perchloroethylene and it can be extracted by purging the liquid with gaseous helium and then extracted from the helium with a cooled carbon trap.
- Gallium was also used by other experiments and kamiokande also observed the deficit.
- Also the fluxes measured where not consistent, depending on the energy range probed. Hinting at oscillations dependent on energy,
%- 4p+ 2e−→4He + 2νe+ 26.73MeV−Eν pp-chain in the sun whereEνis the energy of the two neutrinos, and (26.73 MeV -Eν) is the energyemitted as photons
%- νe+37Cl→37Ar +e−(threshold 814 keV) in homestake, periodically flushed with helium
- 400 000 litresof perchloroethylene (a dry-cleaning fluid), containing520 tof chlorine, placed in the Homestake Mine,1.5 kmunderground [24].
%- chlorine solution capable of neutrino capture(νe+37Cl→37Ar + e−).  The atoms of argon were counted and used as a measure of the neutrinoflux.
- he reported experimental rate was about two thirds less than what was expected from theStandard Solar Model (SSM). This large discrepancy, known as thesolar  neutrino  problem, wasinitially believed to be an experimental flaw.
% ν+Ga71→Ge71+e− for sage and gallex
- This is where the future DUNE detector will be housed, nice full circle
- This is in the solar sector

SNO oscillation measurement in Ref.~\cite{ahmad2002}
- neutrino oscillations were one way of explaining this deficit if some of the electron neutrinos converted flavour in flight.
- SNO finally answered the question when it was able to measure three channels with different relation between te flyx or electron neutrinos and the other neutrinos. SNO could prove that the electron neutrinos are changing flavour. WHile the total flux of all neutrinos remains constant and in agrrement with the SSM.
- 1kton tank of heavy (D2O deuterium) water, able to detect three different channels of neutrino interaction
%- νe+d→p+p+e− νx+d→p+n+νx νx+e−→νx+e−
- Cherenkov experiment, with 9500 8inch photomultiplier tubes detectro the light from neutrino interactions.
- Since each of the rates for the three channels has a different relation between the flux of electron neutrinos and the others, SNO could confirm electron neutrinos are changing flavour, with the total flux being constant and in agreement with the SSM.
- electron neutrino CC, NC and elastic scattering also.
- total rate was consistent but less electron neutrinos than expected as they had oscillated.
- However, only electron neutrinos canundergo CC interactions, as solar neutrinos do not have enough energy to produce muonor tau leptons.
- %e+d→p+p+e−(CC) νx+d→p+n+νx(NC) νx+e−→νx+e−(ES) The CC channel is sensitive exclusively to electron neutrinos, whilst the other two are acces-sible by neutrinos of any flavour. 

Atmospheric kamiokande deficit in Ref.~\cite{hirata1988}
IMD detector atmospheric deficit in Ref.~\cite{becker1992}
Superkamiokande direction atmospheric neutrinos in Ref.~\cite{becker1992}
- This is in the atmospheric sector

\section{Neutrino oscillation theory}
\label{sec:theorytheory}

blah blah blah

\section{Current status and the future}
\label{sec:theorystatus}

\subsection{Atmospheric}

\subsection{Accelerator experiments}

\subsection{Reactor experiments}

\subsection{The future}


blah blah blah