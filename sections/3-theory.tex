\chapter{Neutrino oscillations: theoretical background and current status}
\label{chap:theory}

\chapterquote{Laws were made to be broken.}%
{Christopher North, 1785--1854}%: Blackwood's Magazine May 1830

\feynmandiagram [horizontal=a to b] {
i1 [particle=\(e^{-}\)] -- [fermion] a -- [fermion] i2 [particle=\(e^{+}\)],
a -- [photon, edge label=\(\gamma\), momentum'=\(k\)] b,
f1 [particle=\(\mu^{+}\)] -- [fermion] b -- [fermion] f2 [particle=\(\mu^{-}\)],
};

Symmetries, either intact or broken, have proved to be at the heart
of how matter interacts. The Standard Model of fundamental interactions
(SM) is composed of three independent continuous symmetry groups denoted
$\SUgroup{3} \times \SUgroup{2} \times \Ugroup{1}$, representing the
strong force, weak isospin and hypercharge
respectively~\cite{Phys.Rev.Lett.19.1264, Phys.Rev.D2.1285,hep-ph/0410370}.

\section{Neutral meson mixing}
\label{sec:neutralmixing}
We can go a long way with an effective Hamiltonian approach in
canonical single-particle quantum mechanics. To do this we construct
a wavefunction from a combination of a generic neutral meson state
$\ket{\Xzero}$ and its anti-state $\ket{\Xzerobar}$:
%
\begin{equation}
    \ket{\psi(t)} = a(t)\ket{\Xzero} + b(t)\ket{\Xzerobar}
\end{equation}
%
which is governed by a time-dependent matrix differential equation,
%
\begin{equation}
    \I \pdByd{}{t} \colvector{a \\ b}
    =
    \underbrace{%
        \twomatrix{ M_{11}-\frac{\I}{2}\Gamma_{11}
            & M_{12}-\frac{\I}{2}\Gamma_{12} }
        { M_{12}^\ast-\frac{\I}{2}\Gamma_{12}^\ast
            & M_{22}-\frac{\I}{2}\Gamma_{22} }
    }_{\boldmatrix{H}}
    \colvector{a \\ b}
    .
\end{equation}
