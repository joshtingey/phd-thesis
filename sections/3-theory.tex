\chapter{Neutrino oscillations: theoretical background and current status}
\label{chap:theory}

%%%%%%%%%%%%%%%%%%%%%%%%%%%%%%%%%%%%%%%%%%%%%%%%%%%%%%%%%%%%%%%%%%%%%%%%%%%%%%%%%%%%%%%%%%%%%%%%%%
%                                              PLAN                                              %
%%%%%%%%%%%%%%%%%%%%%%%%%%%%%%%%%%%%%%%%%%%%%%%%%%%%%%%%%%%%%%%%%%%%%%%%%%%%%%%%%%%%%%%%%%%%%%%%%%
\begin{comment}
Welcome to neutrino physics, it all began at the beggining of the century with Wolfgang puali etc
etc... Lots of postulating from various people and some really cool experiments with lots of
exciting details later, it was found that there was a neutrino corresponding to each of the
leptons, electon, muon and tau. In the original standard model these are expected to be massless,
charge-0, spin 1/2 particles that are very very waekly interacting with tiny cross sections,
making them an increadibly difficult challenge for experimentalists in the years to come.

It was then noticed by a succesion of different experiments that a few different things did not
add up, the number of neutrinos seen with the expected flavour was wrong. Lots of people then
invested loads of time into tyring to figure this out and it was finally found by SNO which used
channels sensative to the three flavours in different ways, that some neutrinos had turned into
other neutrino flavours, shock horror.

This is explained by supposing that the neutrino flavour states (the ones that interact with the
weak force) are infact a superposition of neutrino mass states. The mixing between these two basis
is described by the PMNS matrix, which contains 3 mixing angles and a phase. An additional two
phases are included if the particles are majorana. What this means is that as the energy states
porpogate through matter differently, he superposition of them leads to neutrino oscillations as
the probability of detecting a certain flavour state changes as the neutrino travels.

The actual oscillation probability derivation is quite complex, and is affected by the matter
through which it travels due to matter having lots of electrons in it. But it generally leads to
sinosoidal probabilities with a phase dependent on the neutrino mass splitting and L/E and an
amplitude dependent on the mixing angles. The single phase, also governs if this oscillation
probability is different between neutrinos and anti-neutrinos and it hence known as the CP
violating phase.

So far we have been able to measure all the phases but we still don't know what delta-cp is, the
octant of theta23 or the mass hierarchy which is the ordering of the masses of the nuutrinos as we
only really measure the mass-spllittings. Splitting the PMNS matrix into the three common areas,
atmospheric, solar and reactor and investigating these three classic experimantal measuring
regions allows us to make conclusions about different parameters and gives us sensitivity to
different neutrino oscillation paramters.

Current experiments such as Nova and T2K are looking at delta-cp and the hierarchy, but it is
likely that we will need ther VERY VERY expensive DUNE experiment to finally figure things out. As
neutrinos very rarely interact it is always a size game with experiments, as all you really are
playing is a counting game. The larger the detection volume, the more nuutrinos you can detect and
as long as you energy resolution is good enough to resolve the oscillations in you particular L/E
regime you have a better experiment than everyone else. VOLUME VOLUME VOLUME, CHEAP CHEAP CHEAP!!!

INTRODUCTION
- Just a brief outline of what I am going to talk about in the theory chapter
- Tell them what you are going to tell them

THE HISTORY OF NEUTRINOS
- How where neutrinos theorised and then discovered
- What where the big problems that led to neutrino oscillations being suggested and then discovered?

THE THEORY OF NEUTRINO OSCILLATION
- Describe basic neutrino oscillation theory
- derivation of probability in vacuum
- derivation of probability in matter

NEUTRINO INTERACTIONS
- Feynman diagrams of all the main interaction types
- Famous diagrams of cross-section regions for them all
- Descriptions of which types are easy to detect, what are the main NC backgrounds usually etc...

CURRENT EXPERIMENTAL STATUS AND THE FUTURE
- What is the status of each of the three neutrino regimes?
- What are the open questions
- What future experiments could help solve these problems
- CHIPS CHIPS CHIPS
\end{comment}

%%%%%%%%%%%%%%%%%%%%%%%%%%%%%%%%%%%%%%%%%%%%%%%%%%%%%%%%%%%%%%%%%%%%%%%%%%%%%%%%%%%%%%%%%%%%%%%%%%
%                                          INTRODUCTION                                          %
%%%%%%%%%%%%%%%%%%%%%%%%%%%%%%%%%%%%%%%%%%%%%%%%%%%%%%%%%%%%%%%%%%%%%%%%%%%%%%%%%%%%%%%%%%%%%%%%%%
\begin{comment}
-Neutrino physics covers the widest possible range of
-Proposal of a mysterious undetector particle to explain beta decays in the 1930s through to the
resolutions of a 30-year problem with the confirmation of oscillations in the early 2000s and onto
the precision era.
-Neutrino oscillations first discoveed in 1957 when Bruno Pornecorvo proposed a model in which
neutrinos oscilate to antineutrinos and back, similar to the kain. It was actually shown that
neutrinos iscilate from one flavour to another.
-The field of neutrino physics is ever expanding with a new generation of experiments planned for
the coming years.
This chapter aims to provide an introduction to neutrino...
\end{comment}

%%%%%%%%%%%%%%%%%%%%%%%%%%%%%%%%%%%%%%%%%%%%%%%%%%%%%%%%%%%%%%%%%%%%%%%%%%%%%%%%%%%%%%%%%%%%%%%%%%
%                                            HISTORY                                             %
%%%%%%%%%%%%%%%%%%%%%%%%%%%%%%%%%%%%%%%%%%%%%%%%%%%%%%%%%%%%%%%%%%%%%%%%%%%%%%%%%%%%%%%%%%%%%%%%%%
\section{A history of neutrino oscillations}
\label{sec:theory_history}

\subsection{Discovery of the neutrinos} %%%%%%%%%%%%%%%%%%%%%%%%%%%%%%%%%%%%%%%%%%%%%%%%%%%%%%%%%%
\label{sec:theory_history_neutrinos}

In the early 20th century, beta decays were assumed to follow the simple two-body process,
$A \rightarrow B + e$, where nuclei spontaneously emit a single electron. To conserve both energy
and angular momentum the ejected electron must have discrete kinetic energy defined by the
difference in binding energies between the initial and final nuclei states. However, in 1914,
J. Chadwick instead measured a continuous electron energy spectrum~\cite{chadwick1914}, placing
this theory in doubt.

W. Pauli finally proposed a 'desperate solution' to this paradox in 1930~\cite{pauli1930}. If a
light, neutrally charged, spin $1/2$ particle was also produced in the interaction, the continuous
energy distribution could be explained. Initially, this mysterious new particle was named the
'neutron'. However, to avoid confusion with the heavy baryon of the same name discovered in 1932,
E. Fermi renamed it the 'neutrino' when he formalised beta decay in 1934~\cite{fermi1934}.

The following month, H. Bethe and R. Peierls~\cite{bethe1934} used Fermi's work to estimate the
cross-section of the inverse beta decay process:
\begin{equation} % INVERSE BETA DECAY EQUATION %%%%%%%%%%%%%%%%%%%%%%%%%%%%%%%%%%%%%%%%%%%%%%%%%%%
    \bar{\nu} + p^{+} \rightarrow n + e^{+}.
\end{equation} %%%%%%%%%%%%%%%%%%%%%%%%%%%%%%%%%%%%%%%%%%%%%%%%%%%%%%%%%%%%%%%%%%%%%%%%%%%%%%%%%%%
An upper limit of $10^{-44} cm^2$ was calculated, an incredibly small value, leading them to
declare 'there is no practically possible way of observing the neutrino.' This hinted at the vast
difficulties experimentalists would face hunting down and measuring the neutrino in the years to
come.

After an initial tentative identification if 1953, F. Reines and C. Cowan made the first confirmed
observation of the neutrino in 1956~\cite{cowan1956}. Electron antineutrinos produced within the
Savannah River Plan nuclear reactor were detected via the inverse beta decay process outlined
previously. In an underground room of the reactor building, A 'club-sandwich' detector of three
1500 litre liquid scintillator tanks and two 200 litre cadmium doped water target tanks, was
constructed. A total of 330 photomultiplier tubes then measured the two successive characteristic
signals for the interaction allowing for the rejection of the large cosmic ray background.
Firstly, a prompt positron annihilation, followed shortly afterwards by a gamma-ray burst from the
neutron being captured by the cadmium.

A second distinct neutrino, the muon neutrino was discovered in 1962 at the Alternating Gradient
Synchrotron (AGS) at Brookhaven~\cite{danby1962}. Protons from the AGS beam incident upon a fixed
target produced charged pions. These would then decay into a beam of muons and neutrinos. After
passing through steel and lead absorbers to remove the muons, neutrino interactions could then be
detected in a series of spark chambers. If only a single neutrino existed, both interactions
\begin{equation} % BROOKHAVEN DECAY EQUATIONS %%%%%%%%%%%%%%%%%%%%%%%%%%%%%%%%%%%%%%%%%%%%%%%%%%%%
    \nu+p^{+}\rightarrow\mu^{+}+n \\
    \nu+p^{+}\rightarrow e^{+}+n
\end{equation} %%%%%%%%%%%%%%%%%%%%%%%%%%%%%%%%%%%%%%%%%%%%%%%%%%%%%%%%%%%%%%%%%%%%%%%%%%%%%%%%%%%
would be expected to occur at the same rate.
However, only muons, identified by a single long track were detected, confirming the existance of
the muon neutrino. Not only was this experiment the first to construct and use an artificial
neutrino beam, but it also later went onto win the 1988 nobel prize for its discovery.

The $Z^{0}$ and $W^{+}$ bosons, the force carriers for the weak force, through which neutrinos
interact, were discovered at the Super Proton Synchrotron at CERN in 1983
~\cite{arnison1983_z,arnison1983_w}. Crucially, as $Z^{0}$ bosons are expected to decay to
neutrinos, measurements made to the decay width can strongly constrain the number of neutrino
flavours. The ALEPH experiment and others at the LEP $e^{+}e^{-}$ collider in the 1990's made
such precise measurements, indicating that the number of light active neutrino flavours to be
$2.984\pm0.008$~\cite{electroweak2006}.

This combined with the discovery of the tau lepton in 1975~\cite{perl1975}, suggested that there
was a third tau neutrino. The DONUT experiment at Fermilab finally discovered this particle in
2001~\cite{Kodama2001} using 800\GeV Protons from the Tevatron, completing the trio of neutrinos
we know of today. Any additional neutrinos must either be "sterile", in that they do not couple
to the weak force, or have a mass greater than $0.5m_{Z}$.

\subsection{Discovery of neutrino oscillations} %%%%%%%%%%%%%%%%%%%%%%%%%%%%%%%%%%%%%%%%%%%%%%%%%%
\label{sec:theory_history_neutrinos}

As the standard model evolved throughout the 20th century, neutrinos were thought to be massless
and only exist in the three flavour states we observe through interactions. However, various
experiments began to see discrepencies, countering these assumptions. This finally lead to the
discovery of neutrino oscillations whose measurement still dominates the field of neutrino
physics to this day.

At the Homestake mine\footnote{The Homestake mine will also be used for the future DUNE experiment
    discussed later} in the 1960's a large tank filled with 400000 litres of perchloroethylene
\footnote{Perchloroethylene is commonly used as a dry cleaning fluid} ($C_{4}Cl_{8}$) was placed
1.5km underground to reduce the cosmic ray background. It's goal was to measure the solar electron
neutrino flux incident upon the earth. The interaction:
\begin{equation} % HOMESTAKE CHLORINE EQUATION %%%%%%%%%%%%%%%%%%%%%%%%%%%%%%%%%%%%%%%%%%%%%%%%%%%
    {}^{37}Cl+\nu_{e}\rightarrow{}^{37}Ar+e^{-}
\end{equation} %%%%%%%%%%%%%%%%%%%%%%%%%%%%%%%%%%%%%%%%%%%%%%%%%%%%%%%%%%%%%%%%%%%%%%%%%%%%%%%%%%%
allows the neutrino flux to convert the chlorine contained within the tank to the noble gas argon.
Every few weeks the tank was purged with gaseous helium and via a cooled carbon trap the amount of
argon generated measured.

After analysis, the number of electron neutrino interactions per ${}^{37}Cl$ per second was found
to be no greater than 3~\cite{davis1968}. This was compared to predictions made by the Standard
Solar Model (SSM) ranging between 4.4 and 22~\cite{bahcall1968}. The observed deficit was dubbed
the 'solar neutrino problem', but was initially belived to be due to an experimental flaw.
However, the Kamiokande II water Cherenkov experiment~\cite{hirata1989} and both the SAGE and
GALLEX galium based capture experiments~\cite{abazov1991, anselmann1994}, also observed this
discrepancy.

A neutrino deficit was also observed indirectly in the atmospheric sector. The Kamiokande and IMD
experiments, both designed to measure proton decay, had atmospheric neutrinos as an important
background to be understood. Both observed a deficit of muon neutrinos compared to electon
neutrinos when studying this background~\cite{hirata1988, becker1992}. The successor to the
Kamiokande experiments Super-Kamiokande also measured a deficit on atmospheric neutrinos in 1999
~\cite{kajita1999}.

Neutrino oscillations were put forward as a solution to these observed deficits. If neutrinos
could change flavour as they propogated it would be expected to find a reduced number in all the
experiments. The SNO experiment finally resolved the descrepancy with the conclusion of neutrino
oscillations in 2001~\cite{ahmad2002}.

\begin{align} % SNO INTERACTIONS EQUATIONS %%%%%%%%%%%%%%%%%%%%%%%%%%%%%%%%%%%%%%%%%%%%%%%%%%%%%%%
    \nu_{e}+d     & \rightarrow p+p+e^{-}     \\
    \nu_{i}+d     & \rightarrow p+n+\nu_{i}   \\
    \nu_{i}+e^{-} & \rightarrow \nu_{i}+e^{-}
\end{align} %%%%%%%%%%%%%%%%%%%%%%%%%%%%%%%%%%%%%%%%%%%%%%%%%%%%%%%%%%%%%%%%%%%%%%%%%%%%%%%%%%%%%%

~\cite{ahmad2002}

- SNO finally answered the question when it was able to measure three channels with different relation between te flyx or
electron neutrinos and the other neutrinos. SNO could prove that the electron neutrinos are changing flavour.
WHile the total flux of all neutrinos remains constant and in agrrement with the SSM.
- Neutrino oscillations were one wasy of explaining this deficit. In the electron neutrinos were convertd to
muon or tau neutrinos, the electrons in the detector medium would not be able to capture them, additionally
solar neutrinos has insufficienct energy to produce muons or taus which would allow their detection in a water cherenkov detector.
- The SNO experiment finally demonstrated that this was the case in 2001-2002.
- %A 1kton tank of heavy water, containing h^2 isotopes, was able to detect solar neutrinos via three different 
%channels νe+d→p+p+e−, νx+d→p+n+νx and νx+e−→νx+e− where d in the heavy deuterium nucelus. The first channel is a
CC interaction only sensitive to electron neutrinos. The second is an NC interactions and is sensitive to neutrinos
of any flavour, detected via the gamma rays produced when the dislodged neutron thermalises and is captured by another nucelous.
The final channel is elastic scattering of the neutrino from an electron and is again sensitive to all three neutrino flavours.
- By comparing the fluxes measured with the three channels and deividing them into the component attributed to electron neutrinos.
SNO demonstrated a 5.3 sigma singal of oscillations from electrons neutrinos to the other flavours.
- neutrino oscillations were one way of explaining this deficit if some of the electron neutrinos converted flavour in flight.
- SNO finally answered the question when it was able to measure three channels with different relation between te flyx or electron neutrinos and the other neutrinos. SNO could prove that the electron neutrinos are changing flavour. WHile the total flux of all neutrinos remains constant and in agrrement with the SSM.
- 1kton tank of heavy (D2O deuterium) water, able to detect three different channels of neutrino interaction
%- νe+d→p+p+e− νx+d→p+n+νx νx+e−→νx+e−
- Cherenkov experiment, with 9500 8inch photomultiplier tubes detectro the light from neutrino interactions.
- Since each of the rates for the three channels has a different relation between the flux of electron neutrinos and the others, SNO could confirm electron neutrinos are changing flavour, with the total flux being constant and in agreement with the SSM.
- electron neutrino CC, NC and elastic scattering also.
- total rate was consistent but less electron neutrinos than expected as they had oscillated.
- However, only electron neutrinos canundergo CC interactions, as solar neutrinos do not have enough energy to produce muonor tau leptons.
- %e+d→p+p+e−(CC) νx+d→p+n+νx(NC) νx+e−→νx+e−(ES

%%%%%%%%%%%%%%%%%%%%%%%%%%%%%%%%%%%%%%%%%%%%%%%%%%%%%%%%%%%%%%%%%%%%%%%%%%%%%%%%%%%%%%%%%%%%%%%%%%
%                                       OSCILLATION THEORY                                       %
%%%%%%%%%%%%%%%%%%%%%%%%%%%%%%%%%%%%%%%%%%%%%%%%%%%%%%%%%%%%%%%%%%%%%%%%%%%%%%%%%%%%%%%%%%%%%%%%%%
\section{Neutrino oscillation theory}
\label{sec:theory_theory}

\begin{comment}
- B. Pontecorvo first described neutrino oscillation, by nautrino and anti nuetrinos in 1957, later Maki, Nakagawa and Sakata in 1962 included electon and muon neutrino mixing
- The original standard model had the neutrinos as massless. But given that neutrinos oscillations have been observed this is now not valid.
- It is possible to modify the standard model to allow for this fact without new physics, but it is generally considered to be a significant break in the SM.
- Can be explained by the quantum pheneomentum of interferance.
- With the flavour and mass eigenstates not being evuivalent, this mixing is described by the rotation PMNS matrix, analogous to the CKM mixing describing mixing in the quark sector.
- The energy state propogation is well defined, causing the neutrino flavour to change with time in an oscillatory fashion. Such that a muon neutrino after travelling a distance can be detected as
a electon of tau neutrino but after another distance be detector as a muon neutrino again.

- The neutrino has no electric charge, virtually no mass, and can travel undetectably for vast distances.
- How does the standard model allow for neutrino masses?
- Fermi vs majorana mass terms, which one is more likely etc
DIAGRAM: For a given fixed baseline, show the oscillation probability as a function of energy, do this for the chips location.
- Talk about the full neutrino oscillation equations, but also the simplified ones and then how they affect the shape of the probability curves.
EQUATION: Basic neutrino oscillation equations
EQUATION: Neutrino oscillations in matter equations
EQUATION: Matter effect equations
- We look at the oscillation structure across a range of energies in our detectors to measure the oscilaltion parameters.
INFO: How the shape of the neutrino flux at the far detector affects the parameters, reduction fraction, position of the dip etc... maybe a little diagram

- The changing state of the neutrino flavour as it propogates can be explained by having the flavour eigenstates through which the neutrinos
couple to the weak force be different from their energy eigenstates for the matter through which the neutrinos are travelling.
- With the flavour eigenstates being expresses as a superposition of the energy states, neutrino oscillations arise through interference due tof
the energy states evolving differently with time.
- Free particles travelling in a vacuum can be trated as plane waves $E=\sqrt{p^2+m^2}$ therefore, the energy eigenstates have a well-defined mass,
this then implies than there must be three mass states for the neutrinos with different quantum numbers, atleast two must be non-zero.
- To understand neutrinos you need, a flavour basis, an orthonormal energy basis with three energy esigenstates for the medium, eigenvalues of those energy states,
and a unitary matrix to convert between the basis.
- Using the simplified approximation of treating the neutrino states as plane waves, we can simply derive the oscilaltion probabilities.
- Main flaw with this assumption is that plane waves are not localised and can not describe the localised interactions of neutrinos.
- Additionally neutrinos have differing energy and momentum not a common one approximation as c.
- These can be solved when the neutrino states are described as wave packets, but this derivation is much simpler and gets to the same equally correct result.

- In a vacuum, a unitary mixing matrix describes the mixing of mass states to produce the flavour states. and vice-versa
- In the case of three neutrino flavours the unitary mixing matrix is called the Pontecorvo, Maki, Nakata and Sakawa matrix, PMNS.
- Analagous to the CKM (Cabibbo-Kobayashi-Maskawa) matrix.
- $\mathrm{U}_{\mathrm{PMNS}}$ is a unitary, complex, $3\times3$ matrix. Which can be desribed by $n^{2}$ parameters, with $n(n-1)/2$ angles
and $n(n+1)/2$ phases, 3 mixing angles $\theta_{12}$, $\theta_{23}$ and $\theta_{13}$ and six complex phases.
- Some of these phases can be removed as it desribes the mixing between particle fields, hence no physical processes are affected.
- They are absorbed into the neutrino fields.
- If Dirac particles then just one pahse is left $\delta_{CP}$ if Majorana you get an additional two phases $\alpha_{12}$ and $\alpha_{31}$ which lie on the diagonal and hence have no effect on neutrino oscillations.

\end{comment}

\begin{align} % NEUTRINO MIXING EQUATION %%%%%%%%%%%%%%%%%%%%%%%%%%%%%%%%%%%%%%%%%%%%%%%%%%%%%%%%%
    \ket{\nu_{\alpha}} & = \sum_{k}U_{\alpha k} \ket{\nu_{k}}          \\
    \ket{\nu_{k}}      & = \sum_{k}U_{\alpha k}^{*} \ket{\nu_{\alpha}}
\end{align} %%%%%%%%%%%%%%%%%%%%%%%%%%%%%%%%%%%%%%%%%%%%%%%%%%%%%%%%%%%%%%%%%%%%%%%%%%%%%%%%%%%%%%

\begin{gather} % PMNS MATRIX SIMPLE %%%%%%%%%%%%%%%%%%%%%%%%%%%%%%%%%%%%%%%%%%%%%%%%%%%%%%%%%%%%%%
    \begin{pmatrix}
        \ket{\nu_{e}}   \\
        \ket{\nu_{\mu}} \\
        \ket{\nu_{\tau}}
    \end{pmatrix}
    =
    \begin{pmatrix}
        U_{e1}     & U_{e2}     & U_{e3}     \\
        U_{\mu 1}  & U_{\mu 2}  & U_{\mu 3}  \\
        U_{\tau 1} & U_{\tau 2} & U_{\tau 3}
    \end{pmatrix}
    \begin{pmatrix}
        \ket{\nu_{1}} \\
        \ket{\nu_{2}} \\
        \ket{\nu_{3}}
    \end{pmatrix}
\end{gather} %%%%%%%%%%%%%%%%%%%%%%%%%%%%%%%%%%%%%%%%%%%%%%%%%%%%%%%%%%%%%%%%%%%%%%%%%%%%%%%%%%%%%

\begin{align} % DIRAC PMNS MATRIX FULL %%%%%%%%%%%%%%%%%%%%%%%%%%%%%%%%%%%%%%%%%%%%%%%%%%%%%%%%%%%
    \mathrm{U}_{\mathrm{PMNS}} & =
    \begin{pmatrix}
        1 & 0       & 0      \\
        0 & c_{23}  & s_{23} \\
        0 & -s_{23} & c_{23}
    \end{pmatrix}
    \begin{pmatrix}
        c_{13}                   & 0 & s_{13}e^{-i\delta_{CP}} \\
        0                        & 1 & 0                       \\
        -s_{13}e^{-i\delta_{CP}} & 0 & c_{13}
    \end{pmatrix}
    \begin{pmatrix}
        c_{12}  & s_{12} & 0 \\
        -s_{12} & c_{12} & 0 \\
        0       & 0      & 1
    \end{pmatrix}
    \\
                               & =
    \begin{pmatrix}
        c_{12}c_{13}
         & s_{12}c_{13}
         & s_{13}e^{-i\delta_{CP}}                          \\
        -s_{12}c_{23}-c_{12}s_{23}s_{13}e^{i\delta_{CP}}
         & c_{12}c_{23}-s_{12}s_{23}s_{13}e^{i\delta_{CP}}
         & s_{23}c_{13}                                     \\
        s_{12}s_{23}-c_{12}c_{23}s_{13}e^{i\delta_{CP}}
         & -c_{12}s_{23}-s_{12}c_{23}s_{13}e^{i\delta_{CP}}
         & c_{23}c_{13}
    \end{pmatrix}
\end{align} %%%%%%%%%%%%%%%%%%%%%%%%%%%%%%%%%%%%%%%%%%%%%%%%%%%%%%%%%%%%%%%%%%%%%%%%%%%%%%%%%%%%%%


with $s_{ij}=\sin \theta_{ij}$ and $c_{ij}=\cos \theta_{ij}$

\begin{equation} % PLANE WAVE EQUATION %%%%%%%%%%%%%%%%%%%%%%%%%%%%%%%%%%%%%%%%%%%%%%%%%%%%%%%%%%%
    \psi(x,t=0)=\sum_{i=1}^{3}U_{\alpha k}e^{ip_{\alpha}x}
\end{equation} %%%%%%%%%%%%%%%%%%%%%%%%%%%%%%%%%%%%%%%%%%%%%%%%%%%%%%%%%%%%%%%%%%%%%%%%%%%%%%%%%%%

%%%%%%%%%%%%%%%%%%%%%%%%%%%%%%%%%%%%%%%%%%%%%%%%%%%%%%%%%%%%%%%%%%%%%%%%%%%%%%%%%%%%%%%%%%%%%%%%%%
%                                     NEUTRINO INTERACTIONS                                      %
%%%%%%%%%%%%%%%%%%%%%%%%%%%%%%%%%%%%%%%%%%%%%%%%%%%%%%%%%%%%%%%%%%%%%%%%%%%%%%%%%%%%%%%%%%%%%%%%%%
\section{Neutrino interactions}
\label{sec:theory_interactions}

\begin{comment}
- From eV to EeV: Neutrino Cross-Sections Across Energy Scales
- Neutrino originally postulated by Wolfgang Pauli in 1930, and has played a prominent role in understanding of nuclear and particle physics.
- The revalation that neutrinos can no longer be massless is perhaps the first significant alteration to the standard model.
- A nice plot of neutrino energy regimes from Big Bang through accelerator to Extra-galactic vs their cross-sections. I could definitely use this and cite it to the paper
- Contains a good description of fundamental electroweak scattering if we need a reference for that (page 4->)
- First anumu + e- => anumu + e- scattering made by CERN bubble cchamber experiment Gargamelle
- This and DIS NC observations confirmed the weak neutral currents and helped solidify the standard model.
- Maybe include a bubble chamber image of the first candidate neutrino interaction.
- At intermediate energy scales (CHIPS range) interactions fall into three main catageories. Elastic and quasi-elastic scattering, resonance production and DIS.
- Include an actual data cross-section plot for both CC and NC showing contributions from different experiments
- Show the tau-neutrino cross section compared to the muon/electron in our energy range to show it doesn't matter. All the interactions and arguments that go along with them are the same for nuel/numu as with nutau, except for one key difference; the energy threshold. The nutau interaction CC cross section is severely altered because of the large tau lepton mass. Then show the plot.
- Bellow 2Gev it's mainly quasi-elastic with the neutrino scattering of the entire nucelon, rather than its consituent partons.
- Modern experiments MiniBooNE and NOMAD see higher absolute cross-sections than expected. It is currently believed that nucelear effects beyond the impulse approximation are desponsible for the discrepancy. Such as nucleon-nucleon correlations and two-body exchange currents must be included to get it righ. THIS IS MEC!
- NC QEL lots of people call NC Elastic Scattering, the ratio of NC Elastic/CC QE is ~0.11 from measurements by a few experiments.
- Single pion production is when the neutrino excites the struck nucleon producing a baryon resonance, which then quickly decays most often into a nucleon and a single pion final state. There are seven possible single pion channels, 3CC and 4NC, which we see from the GENIE events.
- Show all the interaction equations for these %νμp→μ−pπ+ etc...
- NC pi-zero production is often the largest numu-induced backgrond in experiments searching for numu->nuel oscillations. And CC pi production can present a non-negligable complication in the determination of neutrino energy in experiments. Therefore measuring and modelling nuclear effects in pion production has become paramount.
- These resonances can also decay into photons with a small branchng fraction, yes, but, like NC pi-zero production they still pose a non-negliable source of background to the CHIPS main search.
- Neutrinos can also coherently produce single pion final state. In this case the neutrino coherently scatters from the entire nucleus transferring negligable energy to the target. Hence, you produce a ditinctly forward-scattered pion with no nuclear recoil. This process is relatively small however.
- The resonances can also decay to multi-pion final state, along with DIS this contributes a copious source of multi-pion final states. However, due to the inherant complexity of reconstructing multiple pion final states, not many experiments look at these cross-sections.
- You can also get kaon production but they have small cross-sections due to the kain mass and because kaon channels are not enhanced by any dominant resonance.
- You then get DIS where the neutrino scatters of a quark in the nucleon via the exchange of a virtual W or Z boson producing a lepton and a hadronic system in the final state.
- To isolate DOS events experiments typically apply kinematic cuts to remove QE scattering and resonance-mediated contributions from their data.
\end{comment}

%%%%%%%%%%%%%%%%%%%%%%%%%%%%%%%%%%%%%%%%%%%%%%%%%%%%%%%%%%%%%%%%%%%%%%%%%%%%%%%%%%%%%%%%%%%%%%%%%%
%                                 CURRENT STATUS AND THE FUTURE                                  %
%%%%%%%%%%%%%%%%%%%%%%%%%%%%%%%%%%%%%%%%%%%%%%%%%%%%%%%%%%%%%%%%%%%%%%%%%%%%%%%%%%%%%%%%%%%%%%%%%%
\section{Current status and the future}
\label{sec:theory_status}

\begin{comment}
- Over that last 20 years neutrino oscillations have become well-established and we are now moving into the precision measurement era.
- DUNE is a next-generation neutrino oscillation experiment with a primary scientific goal of observation of CP-violation in the neutrino sector.
- In DUNE a muon neutrino(anti-neutrino) beam will be produced by the Long-Baseline Neutrino Facility (LBNF)
- There will be a near detector at Fermilab before the neutrinos travel the 1285km to the Sanford Underground Research Facility (SURF) in South Dakota.
- The far detector will consist of four 10kt (fiducial) liquid argon time projections chamber (LArTPC) detectors.
- Neutrino oscillation probabilities can then be infered by comparisons of the observed neutrino spectra and the near and far detectors.
- Recent obsevation of a large theta13 have focuseed the next genetation of long baseline experiments towards the mass hierarchy, octant of theta23 and measureing delta-CP
- Nova and T2k will not be able to measure the remaining unknows (check this)
- Dune will hopefully solve these problems but will be increadibly expensive

- Symmetries under charge-conjugation and parity inversion are both macimally violated by the waek interaction.
- Their combined operation has been shown to be violated, to a small degress, by quark mixing processes.
- If sin(delta-cp) is non zero then vacuum oscillation properties of nu and anti nu will be different.
- DUNE (I assume CHIPS) is sensative to four oscillation paramters, delta31, theta23, theta13 and delta-cp.
- These can be measured using four data sample, two for neutrino and two for antineutrinos.
- These sample are produced by "Forward Horn current" FHC and "Reverse Horn current" RHC, producing predominetetly neutrinos and anti-neutrinos respectively.
- Dissapearence channels sensitive to %abs(delta31^2), and sin^2(2theta23).
- Apperence channels sensitive to all four parameters including sign of %delta32^2.
- The "signal" in all cases are CC interactions, therefore selection of nuel, anuel, numu and anumu CC is the goal.
- Main background in CC numu selections are NC with charged pions.
- Main background in CC nuel selections is pi-zero NC, which can mimic the chracteristic EM shower, due to its near certain decay into two photons.
- You get a small number of nuel intrinsic to the beam, they are just a background as they are indistinguishabe from the nuel appreaence neutrinos.
- Once you have collected samples in all four cases, a fit is performed to the reconstructed neutrino energy distributions to extract the four neutrino oscillation parameters.
- How is the absolute nuetrino energy constrained\dots
- KATRIN upper limit of 1.1eV (90percent confidence level) on the absolute mass of the neutrinos.
KATRIN mass in in Ref.~\cite{aker2019}
\end{comment}