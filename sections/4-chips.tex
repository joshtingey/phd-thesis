\chapter{The \chips R\&D Project} %%%%%%%%%%%%%%%%%%%%%%%%%%%%%%%%%%%%%%%%%%%%%%%%%%%%%%%%%%%%%%%%
\label{chap:chips}

In pursuit of answers to the open questions outlined in the previous chapter, neutrino experiments
are becoming increasingly, and possibly prohibitively, expensive and impractical. This is
particularly true of the next generation of long baseline experiments, DUNE and Hyper-Kamiokande,
with cost estimates reaching billions of dollars and construction times of greater than half a
decade. It is also telling that just these two future projects receive  the vast majority of
global research effort, such is their complexity, cost, and lead time.

It is clear that for detectors to remain practical and affordable into the future, a novel
approach to neutrino detector design is highly desirable. This is especially the case if megaton
scale detectors are ever to become a reality. While instrumentation will continue to improve with
time, the statistics of low event counts will always limit neutrino experiments until vastly
larger detectors can be built. Therefore, R\&D efforts must focus on such detectors now, whilst
also attempting to complement the current and upcoming generation of experiments.

The \chips R\&D project~\cite{adamson2013} aims to develop novel strategies and technologies for
very large yet `cheap as chips' water Cherenkov detectors. Primarily aimed for deployment in long
baseline accelerator beam scenarios, \chips aims to lower the construction cost per kt of fiducial
mass to between \$200k-\$300k. For comparison, the Super-Kamiokande detector reached a cost of
approximately \$4 million/kt to build. As physics sensitivity depends on more than just fiducial
volume, this comparison is not entirely rigorous; however, it highlights the scale of cost savings
possible.

This chapter aims to describe the fundamental aspects of the \chips R\&D project in detail.
Firstly, the \chips concept will be outlined along with neutrino beam and Cherenkov detector
physics for context. The design, construction, deployment, and status of the \chipsfive prototype
detector will then follow. Finally, the Monte Carlo event generation and detector simulation
framework will be introduced to aid the discussion of work presented in later chapters of this
thesis.

\section{\chips concept} %%%%%%%%%%%%%%%%%%%%%%%%%%%%%%%%%%%%%%%%%%%%%%%%%%%%%%%%%%%%%%%%%%%%%%%%%
\label{sec:chips_concept} %%%%%%%%%%%%%%%%%%%%%%%%%%%%%%%%%%%%%%%%%%%%%%%%%%%%%%%%%%%%%%%%%%%%%%%%

The \chips concept is to deploy cylindrical water Cherenkov detector modules into deep bodies of
water on the Earth's surface such as lakes, reservoirs and flooded mine pits. Initially
constructed on land \chips detectors can be floated into position before being sunk. The water
above the sunken detector provides a modest overburden from cosmic rays, while the surrounding
water provides support for a lightweight detector structure. By removing the need for underground
excavation and expensive structural support, the cost of construction can be dramatically reduced.

Additionally, the common practice of building majority bespoke components is replaced by using
modern commercially available components wherever possible. The number of expensive elements, such
as photomultiplier tubes is also reduced by only considering multi \GeV accelerator beam neutrino
events, such that full coverage high-density detector instrumentation is not required.

Furthermore, \chips detectors are not only designed to be cheap but practical. Easy to build,
quick to deploy, and upgradable once operational, multiple detector modules can be flexibly
combined depending on available resources. When compared to DUNE and Hyper-Kamiokande both
requiring a large upfront budget and many years to construct, cheap \chips detector modules can be
deployed as needed in under a year by a relatively small team.

To date, \chips R\&D efforts have been based in the USA to exploit the NuMI beam before the end of
its lifetime. Future plans are focused on the scaling of \chips detectors for the deployment of
multiple modules within the LBNF beam once operational. Collaborators from primarily University
College London, The University of Wisconsin Madison and Nikhef are focused on multiple R\&D
efforts, outlined below, each aiming to prove the viability of a crucial component of the \chips
concept.

\begin{itemize}
    \item \textbf{Detector construction:} This effort aims to prove that the construction and
          deployment of \chips concept detectors are possible. Two prototype detectors have so far
          been deployed. Firstly, the small (\unit{3.3}{\mathrm{m}} high) \chipsm detector,
          deployed into a flooded mine pit located in northern Minnesota during the summer of
          2014, as shown in Fig.~\ref{fig:chips_m}~\cite{perch2015, pfutznerProto2017,
              pfutzner2017}. Secondly, the much larger \unit{5}{\mathrm{kt}} \chipsfive detector,
          deployed into the same pit during the summer of 2019 and detailed in
          Section.~\ref{sec:chips_detector}.

    \item \textbf{Water filtration:} This effort aims to prove that adequate water purity can be
          achieved using cheap, commercially available filtration. Extensive studies have proven
          that by filtering water directly from bodies of water on the Earth's surface (including
          flooded mine pits), adequate photon attenuation lengths, greater than
          \unit{50}{\mathrm{m}} are achievable~\cite{amat2017, campbell2020}.

    \item \textbf{Physics sensitivity:} This effort aims to prove that \chips concept detectors
          (even the prototypes) can provide significant physics contributions alone or alongside
          the current and next generation of experiments. Single modules in the current NuMI beam
          (discussed in Section.~\ref{sec:chips_concept_beam}) and multiple modules in the future
          LBNF beam have been considered in multiple studies~\cite{pfutzner2017, adde2016,
              lang2015}.

    \item \textbf{Data acquisition:} This effort aims to prove that a cheap data acquisition (DAQ)
          system using commercially available components and software is viable~\cite{eijk2018}.
          Outlined in Chapter.~\ref{chap:daq}, \chips implements a novel use of cheap single-board
          computers to collect photomultiplier tube data.

    \item \textbf{Event reconstruction and classification:} This effort aims to prove that modern
          machine learning techniques can be successfully applied to large water Cherenkov
          concepts such as \chips. Detailed as the primary contribution of this thesis in
          Chapter.~\ref{chap:cvn}, this work feeds directly into both the physics sensitivity work
          mentioned above and detailed detector module optimisation studies.
\end{itemize}

\begin{figure} % CHIPS-M DIAGRAM %
    \includegraphics[width=0.6\textwidth]{diagrams/4-chips/chips_m.png}
    \caption[Picture of the \chipsm detector.]
    {Picture of the \chipsm detector just before deployment. The temporary floatation bags are
        visibly attached to the top rim of the detector. Additionally, the umbilical cord carrying
        data, power, and filtered water is attached to the bottom of the detector.}
    \label{fig:chips_m}
\end{figure}

\subsection{The neutrino beam and off-axis alignment} %%%%%%%%%%%%%%%%%%%%%%%%%%%%%%%%%%%%%%%%%%%%
\label{sec:chips_concept_beam} %%%%%%%%%%%%%%%%%%%%%%%%%%%%%%%%%%%%%%%%%%%%%%%%%%%%%%%%%%%%%%%%%%%

\chips detectors will primarily study the appearance of $\nu_{e}$ oscillating from $\nu_{\mu}$
over a long baseline. To generate a sufficient number of GeV scale $\nu_{\mu}$, a high-intensity
accelerator beam is required. Currently, only two such beams exist, the J-PARC based beam in Japan
used by the T2K experiment and the \numi beam in the USA used by \nova. Here we describe the
\numi~\cite{adamson2016} (Neutrinos at the Main Injection) beam as it is directly relevant to
current \chips efforts. However, it is essential to note that \chips detectors are designed to be
used in any neutrino beam, including the future \numi replacement, LBNF.

The \numi beam is an accelerator muon neutrino beam produced at Fermilab near Chicago in the USA.
Beginning operation in 2005 for the MINOS experiment, \numi was upgraded in 2013 to provide a
higher intensity and energy, principally to achieve peak energy near the $\sim$\unit{1.5}{\GeV}
$\nu_{\mu}\rightarrow\nu_{e}$ oscillation maximum for \nova. Currently the \numi beam achieves an
intensity above \unit{700}{\mathrm{kW}} (\unit{740}{\mathrm{kW}} peak) making it the most powerful
such beam in the world. A schematic of the \numi beamline configuration is shown in
Fig.~\ref{fig:numi_beam}.

\begin{figure} % NUMI BEAM DIAGRAM %
    \includegraphics[width=\textwidth]{diagrams/4-chips/numi_beam.png}
    \caption[Schematic of the main components of the \numi beam.]
    {Schematic of the main components of the \numi beam (not to scale) shown with dimensions.
        Figure taken from Ref.~\cite{adamson2016}. The MINOS and \nova near detectors and the
        MINERvA experiment are located just to the right of what is shown.}
    \label{fig:numi_beam}
\end{figure}

Every \unit{1.33}{\mathrm{seconds}} a \unit{10}{\mu\mathrm{s}} long spill of protons accelerated
to \unit{120}{\GeV} by the Main Injector are directed towards a stationary graphite target. The
resulting interactions within the target create a shower of secondary hadrons containing
predominantly pions and kaons. The hadrons are passed through a focusing system of two magnetic
horns tuned principally to focus charged pions along the beamline while rejecting other particles.
After focusing, the surviving hadrons are allowed to decay in flight to a beam of muon neutrinos
in a \unit{675}{\mathrm{m}} long decay pipe via the processes,
\begin{align} % NUMI DECAY EQUATIONS %
    \pi^{+} & \rightarrow\mu^{+}+\nu_{\mu}, \\
    K^{+}   & \rightarrow\mu^{+}+\nu_{\mu}. \\
    \label{eq:numi_decays}
\end{align}
The resulting muons also decay such that $\mu^{+}\rightarrow e^{+}+\nu_{e}+\bar{\nu}_{\mu}$
producing an intrinsic $\nu_{e}$ background (alongside other decays) as well as wrong sign
$\nu_{\mu}$ contamination. Alternatively, the polarity of the horns can be used to switch the
dominant sign of the hadrons focused, allowing \numi to operate as either a neutrino or
antineutrino beam. These two modes of operation are called \emph{forward horn current} and
\emph{reverse horn current} for primarily a neutrino or antineutrino beam composition
respectively. Any remaining hadrons, alongside electrons, muons and surviving primary protons are
absorbed by rock downstream of the decay pipe, leaving just the neutrino components of the beam.

Long baseline neutrino experiments typically consist of a \emph{near detector} to measure the beam
neutrino composition at source and a much larger \emph{far detector} to measure the oscillated
composition after a long distance. The \numi beamline contains three detectors just downstream of
\unit{300}{\mathrm{m}} of rock: The MINERvA spectrometer~\cite{mcfarland2006}, the near detector
for MINOS (now used by MINERvA), and the near detector for \nova. \chips prototypes within the
\numi beam will not have a dedicated near detector; therefore, data from the above detectors will
be crucial for physics analysis in order to constrain the beam composition and flux. However, any
future near detector for \chips, possibly within the LBNF beam should also be a water Cherenkov
detector, to reduce detector induced systematic uncertainties.

The \numi neutrino beam passes through the Earth's crust until it finally emerges in northern
Minnesota. This is where the MINOS, \nova, and prototype \chips far detectors are located (used to
be located in the MINOS case), as shown in Fig.~\ref{fig:numi_map}. The Minnesota state nickname
`land of 10,000 lakes' is not an overstatement, there are a vast number of potential lakes for
\chips detector deployment in this region. Additionally, intense iron ore mining on the `Iron
Range' provides many suitable disused (and now flooded) mine pits. The exact \chipsfive prototype
detector location is discussed in greater detail in Section.~\ref{sec:chips_detector}.

\begin{figure} % CHIPS LOCATION IN NUMI DIAGRAM %
    \includegraphics[width=0.7\textwidth]{diagrams/4-chips/numi_map.png}
    \caption[Map of detector locations in the \numi beam.]
    {Map of the MINOS, \nova and \chips locations in the \numi beam as it surfaces in northern
        Minnesota. Shown is the expected neutrino event rate assuming no oscillations, with lines
        of constant L/E indicated by contours. The western extent of Lake Superior can be seen in
        the lower right of the map. Image taken from Ref.\cite{adamson2013}.}
    \label{fig:numi_map}
\end{figure}

Due to the kinematics of pion decay, whether the far detector is placed directly on the beam axis
or not can have a significant impact on the observed energy spectrum of beam neutrinos as shown in
Fig.~\ref{fig:numi_axis}. For neutrinos directly on the beam axis, there is a strong energy
dependence on the parent pion energy. However, as the off-axis angle increases, the neutrino
energy is less dependent on the parent pion energy and becomes restricted to a narrowing range of
decreasing energies. This effect known as the \emph{off-axis effect} is used by both \nova and T2K
to create a narrower energy spectrum focused on the $\nu_{\mu}\rightarrow\nu_{e}$ oscillation
maximum. Additionally, by reducing the tail of higher-energy neutrinos, the number of background
NC events can be greatly reduced, as is the case for the \unit{7}{\mathrm{mrad}} off-axis
\chipsfive detector.

\begin{figure} % OFF-AXIS FLUX DIAGRAM %
    \includegraphics[width=0.6\textwidth]{diagrams/4-chips/numi_axis.png}
    \caption[Muon neutrino flux for different \numi detectors at different off-axis angles.]
    {Muon neutrino flux for different \numi detectors at different off-axis angles. Shown are the
        neutrino energy spectrum for MINOS (on-axis), \nova (\unit{14}{\mathrm{mrad}} off-axis),
        and \chipsfive (\unit{7}{\mathrm{mrad}} off-axis). Figure taken from
        Ref.~\cite{adamson2013}.}
    \label{fig:numi_axis}
\end{figure}

\subsection{Water Cherenkov detectors} %%%%%%%%%%%%%%%%%%%%%%%%%%%%%%%%%%%%%%%%%%%%%%%%%%%%%%%%%%%
\label{sec:chips_concept_cherenkov} %%%%%%%%%%%%%%%%%%%%%%%%%%%%%%%%%%%%%%%%%%%%%%%%%%%%%%%%%%%%%%

The \chips detector concept is based upon the water Cherenkov technique for neutrino detection,
which has successfully been used multiple times by neutrino experiments. A large body of target
water is instrumented with photomultiplier tubes (PMTs) to record the Cherenkov radiation produced
by sufficiently relativistic charged particles resulting from neutrino interactions. By using
readily available water as the target material and only instrumenting the volume surface, water
Cherenkov detectors provide the best detection methodology for maximising volume and reducing
cost.

Cherenkov radiation is emitted by all electrically charged particles travelling faster than the
local phase velocity of light in a dielectric medium. Similar to the sonic boom created by a
supersonic aircraft, Cherenkov radiation forms a shock wave of coherent light, as shown in
Fig.~\ref{fig:cherenkov}, typically with wavelengths in the ultraviolet and optical range. When
projected onto the detector wall, the resulting cone of radiation generates a distinctive ring
shape. The cone opening angle (the angle at which light is emitted) $\theta_{c}$ is given by:
\begin{equation}
    \cos\theta_{c} = \frac{1}{\beta n(\lambda)},
    \label{eq:cherenkov_angle}
\end{equation}
where $\beta=v/c$ and $n$ is the refractive index of the medium~\cite{particle2020}. Note that $n$
is a function of the wavelength of emission $\lambda$, and so is the opening angle. As the
refractive index of water is $\sim 1.33$ for typical wavelengths of emission, and using the
ultrarelativistic limit $\beta\sim 1$ the opening angle is found to be $\sim41^{\circ}$.

\begin{figure} % CHERENKOV EFFECT DIAGRAM DIAGRAM %
    \includegraphics[width=0.6\textwidth]{diagrams/4-chips/cherenkov.png}
    \caption[Diagram of Cherenkov radiation emission.]
    {Diagram of Cherenkov radiation emission and wavefront angles. The angles $\theta_{c}$ and
        $\eta$ are not equal in a dispersive medium. Figure taken from Ref.~\cite{particle2020}}
    \label{fig:cherenkov}
\end{figure}

There is no Cherenkov emission when $\cos\theta_{c} > 1$ in Eq.~\ref{eq:cherenkov_angle}, which is
the case when $\beta n(\lambda)<1$. Therefore, there exists a Cherenkov energy threshold $E_{t}$
for charged particle, which when expressed in terms of the particle mass $m$ is given by:
\begin{equation}
    E_{t} = \gamma m = \frac{m}{\sqrt{1-(1/n)^{2}}}.
    \label{eq:cherenkov_threshold}
\end{equation}
Again using $n\sim 1.33$ this gives a typical threshold energy of $\sim1.5m$.

The number of Cherenkov photons emitted by a particle of charge $ze$ per unit wavelength per unit
path length is given by:
\begin{equation}
    \frac{d^{2}N}{d\lambda dx}=\frac{2\pi\alpha z^{2}}{\lambda^{2}}
    \left(1-\frac{1}{1-\beta^{2}n^{2}(\lambda)}\right),
    \label{eq:cherenkov_emission}
\end{equation}
where $\alpha$ is the fine structure constant~\cite{particle2020}. Integrating over the range of
wavelengths for which PMTs are typically sensitive, \unit{350}{\mathrm{nm}} to
\unit{650}{\mathrm{nm}}, and using $\beta\sim 1$ and $n\sim 1.33$ gives approximately $240$
photons emitted per cm travelled by the charged particle~\cite{perch2017}.

By analysing the Cherenkov ring (or rings) of light recorded by the PMTs on the walls of the
detector, information about the charged particle (particles) within an event can be determined.
The underlying neutrino interaction can then be understood indirectly. The primary challenge for
accelerator beam water Cherenkov detectors is the identification and reconstruction of an electron
or muon ring likely to have been produced initially from a beam neutrino and not a cosmic ray.
This indicates a beam CC $\nu_{e}$ or CC $\nu_{\mu}$ event respectively, rejecting NC and cosmic
events.

The basic shape of a Cherenkov ring can be used to tell which charged particle created it. Muons
are long-lived and typically travel many metres within the detector, producing a clean ring with
sharp edges. Conversely, electrons almost immediately initiate an electromagnetic shower;
therefore, the observed ring is the sum of all the rings produced from the individual electrons
and positrons within the shower. As a consequence, when compared to a muon ring, electron rings
are commonly wider and characteristically fuzzy. A factor of this difference can be seen in
Fig.~\ref{fig:emission distance}, which shows how the total amount of Cherenkov radiation is
emitted for both electrons and muons as a function of distance from the interaction vertex.

\begin{figure} % EMISSION DISTANCE DIAGRAM %
    \includegraphics[width=0.6\textwidth]{diagrams/4-chips/emission_distance.pdf}
    \caption[Fraction of Cherenkov photons emitted as a function of distance.]
    {The fraction of the total number of photons emitted as a function of the distance from the
        interaction vertex for both electrons and muons with an energy of \unit{2.5}{\GeV}. Note
        how the muon travels much further emitted an approximately constant amount of Cherenkov
        radiation as it does so.}
    \label{fig:emission distance}
\end{figure}

The situation becomes complicated when many charged particles above the Cherenkov threshold are
involved, which is common at higher energies. In this case, multiple overlapping rings are
observed, making reconstruction difficult. The worst-case scenario, however, is when two rings
entirely overlap, removing any ability to tell them apart. This is commonly the case for NC
interactions producing a $\pi^{0}$ in the final state, forming the primary background for CC
$\nu_{e}$ appearance analyses after CC $\nu_{\mu}$ rejection.

With a 98.82\% branching ratio $\pi^{0}$ particles decay to a pair of photons~\cite{particle2020}.
Both photons like electrons almost immediately initiate an electromagnetic shower, leading to two
electron like rings. The separation angle between the rings is then given by:
\begin{equation}
    (1-\cos\theta_{ij})=\frac{m_{\pi}^2}{2E_{i}E_{k}},
\end{equation}
where $m_{\pi}$ is the invariant mass of the $\pi^{0}$ and $E_{i}$ and $E_{j}$ are the energies of
the two photons respectively. Therefore, for a $\pi^{0}$ decaying to two \unit{1}{\GeV} photons,
there is just $\sim 8^{\circ}$ of separation between the rings, making them incredibly hard to
tell apart, especially when electron like rings are fuzzy. Conversely, if the two photons have an
unequal energy distribution, such that one is much more energetic than the other, the higher
energy photon ring can dominate, and the other can not be identified, leading to what looks like a
single electron ring.

\section{The \chipsfive detector} %%%%%%%%%%%%%%%%%%%%%%%%%%%%%%%%%%%%%%%%%%%%%%%%%%%%%%%%%%%%%%%%
\label{sec:chips_detector} %%%%%%%%%%%%%%%%%%%%%%%%%%%%%%%%%%%%%%%%%%%%%%%%%%%%%%%%%%%%%%%%%%%%%%%

\chipsfive is the first large scale prototype detector module for the \chips project. A
\unit{25}{\mathrm{m}} wide and \unit{12}{\mathrm{m}} high cylinder once fully deployed, \chipsfive
has an inner surface area of \unit{1924}{\mathrm{m}^2} and a total target mass
\unit{5.9}{\mathrm{kton}}. Via the process of design, construction, deployment, and data taking,
\chipsfive primarily aims to refine the \chips concept for future full-scale
($\sim$\unit{15}{\mathrm{kton}}) modules. With this in mind \chipsfive is designed such that the
details outlined in this section are fully characteristic of what a full-sized \chips module is
envisioned to be. Here, the location, structure, instrumentation, water filtration, deployment,
and current status are presented. The full electronics and DAQ details are instead outlined in
Chapter.~\ref{chap:daq}.

\subsection{Location} %%%%%%%%%%%%%%%%%%%%%%%%%%%%%%%%%%%%%%%%%%%%%%%%%%%%%%%%%%%%%%%%%%%%%%%%%%%
\label{sec:chips_detector_location} %%%%%%%%%%%%%%%%%%%%%%%%%%%%%%%%%%%%%%%%%%%%%%%%%%%%%%%%%%%%%

\chipsfive is located at the Wentworth 2W pit in northern Minnesota, USA, near the small town of
Hoyt Lakes. A disused and flooded surface Taconite ore (a type of iron ore) mine pit, Wentworth 2W
is located \unit{7}{\mathrm{mrad}} off the \numi beam axis at a distance of
\unit{712}{\mathrm{km}} from the beam target. Roughly
\unit{0.8}{\mathrm{km}}$\times$\unit{1.2}{\mathrm{km}} in size with a maximum depth of
\unit{60}{\mathrm{m}} ($\pm3\mathrm{m}$ throughout the year), the pit freezes over during winter,
allowing for work to only take place during the summer months of May to October.

A large earthen ramp on the south side of the Wentworth 2W pit is used for construction on land.
The construction site is easily accessible by road and well connected to power, due to the heavy
infrastructure in place for mining. Additionally, the nearby PolyMet mining administration
building is used as a laboratory environment for construction and testing of individual components
before installation within the detector. A labelled satellite view of Wentworth 2W is given in
Fig.~\ref{fig:pit} for context, with a picture of the construction site shown in
Fig.~\ref{fig:from_the_sky}.

\begin{figure} % PIT DIAGRAM %
    \includegraphics[width=\textwidth]{diagrams/4-chips/pit.pdf}
    \caption[Satellite view of the Wentworth 2W mine pit, with key locations.]
    {Satellite view of the Wentworth 2W flooded mine pit in northern Minnesota, showing key
        \chipsfive locations. The PolyMet building, shore huts, construction site and deployment
        location are shown. For both the construction site and deployment location the red circle
        shows the \chipsfive detector size to scale.}
    \label{fig:pit}
\end{figure}

\begin{figure} % CHIPS FROM THE SKY DIAGRAM %
    \includegraphics[width=\textwidth]{diagrams/4-chips/from_the_sky.jpg}
    \caption[Picture of the \chipsfive construction site from the air.]
    {Picture of the \chipsfive construction site from the air facing south. The Wentworth 2W pit
        is in the lower half of the image, with the part built \chipsfive detector visible at the
        bottom of the earthen construction ramp. The two white shore huts can just be seen halfway
        up the ramp.}
    \label{fig:from_the_sky}
\end{figure}

\subsection{Structure} %%%%%%%%%%%%%%%%%%%%%%%%%%%%%%%%%%%%%%%%%%%%%%%%%%%%%%%%%%%%%%%%%%%%%%%%%%%
\label{sec:chips_detector_structure} %%%%%%%%%%%%%%%%%%%%%%%%%%%%%%%%%%%%%%%%%%%%%%%%%%%%%%%%%%%%%

The structure of the \chipsfive detector module consists primarily of two \unit{26}{\mathrm{m}}
diameter and \unit{1.5}{\mathrm{m}} high lightweight stainless steel circular \emph{endcaps} that
form the top and bottom of the cylinder. During construction on land the conveniently named
\emph{top-cap} is held above the \emph{bottom-cap} by \unit{1.5}{\mathrm{m}} long steel struts as
shown in Fig.~\ref{fig:frame}. This configuration allows for the endcap instrumentation, detailed
in Section.~\ref{sec:chips_detector_instrumentation}, to be easily installed.

\begin{figure} % CHIPS FRAME DIAGRAM %
    \includegraphics[width=\textwidth]{diagrams/4-chips/frame.jpeg}
    \caption[Picture of the \chipsfive structural frame.]
    {Picture of the \chipsfive structural frame, with humans for scale. The top and bottom endcaps
        can be clearly seen separated by steel struts. Rows of stainless steel \emph{stringers}
        are attached to the inside of each endcap to mount the instrumentation.}
    \label{fig:frame}
\end{figure}

The two endcaps are connected by 28 \unit{12}{\mathrm{m}} long Dyneema cables around their
perimeter. Additionally, 48 \unit{16}{\mathrm{inch}} diameter air-filled PVC pipes are attached to
the frame of the top-cap making it buoyant. Therefore, once deployed into the pit, the bottom-cap
sinks while the top-cap floats, this pulls the Dyneema cable taut, forming the final expanded
detector shape, as shown in Fig.~\ref{fig:chips_render}.

\begin{figure} % CHIPS RENDER DIAGRAM %
    \includegraphics[width=0.6\textwidth]{diagrams/4-chips/chips_render_1.png}
    \caption[Graphical rendering of the \chipsfive detector.]
    {Graphical rendering of the fully deployed and expanded \chipsfive detector module with a
        section of the liner cutaway. The bottom endcap and wall planes are visible, as well as
        the top endcap structure and floatation. The green lines indicate the Dyneema cables
        holding the top-cap and bottom-cap together.}
    \label{fig:chips_render}
\end{figure}

\subsection{Instrumentation} %%%%%%%%%%%%%%%%%%%%%%%%%%%%%%%%%%%%%%%%%%%%%%%%%%%%%%%%%%%%%%%%%%%%%
\label{sec:chips_detector_instrumentation} %%%%%%%%%%%%%%%%%%%%%%%%%%%%%%%%%%%%%%%%%%%%%%%%%%%%%%%

The \chipsfive detector is instrumented with PMTs arranged within distinct plane like structures
called Planar Optical Modules (POMs), which take inspiration from the Digital Optical Modules
(DOMs) used by IceCube and KM3NeT~\cite{hanson2006, eijk2015}. Each POM is a roughly
\unit{2}{\mathrm{m}}$\times$\unit{3}{\mathrm{m}} array of watertight PVC tubing equipped with
anywhere between $15$ to $30$ PMTs, in addition to the lowest level of DAQ electronics and power
distribution. Standard commercially available schedule $40$ PVC piping and connectors are used to
form the structure of each plane, bound together with standard PVC primer and cement.

There are two types of POM used within \chipsfive, differentiated by the PMTs and the data
acquisition electronics they use. Firstly, \emph{Nikhef} POMs use \unit{88}{\mathrm{mm}} HZC PMTs
with electronics developed by the KM3NeT experiment~\cite{katz2009, adrian2016}. Secondly,
\emph{Madison} POMs use \unit{3}{\mathrm{inch}} Hamamatsu R6091 PMTs donated from the NEMO3
experiment~\cite{arnold2005} with novel electronics developed by \chips in collaboration with the
Wisconsin IceCube Particle Astrophysics Centre (WIPAC) in Madison, Wisconsin. The
\unit{88}{\mathrm{mm}} HZC PMTs have a high ratio of output electrons to incident photons (quantum
efficiency) of 24.4\% at a wavelength of \unit{400}{\mathrm{nm}}, compared to the low 12.0\% ratio
achieved by the R6091 PMTs.

A total of $6114$ \unit{88}{\mathrm{mm}} HZC and $450$ Hamamatsu R6091 PMTs are arranged into
$226$ Nikhef and $30$ Madison POMs. Every PMT is housed in an assembly as shown in
Fig.~\ref{fig:pmt_assembly} for the Nikhef case, which is glued directly into the POM PVC
structure. Importantly, to increase the level of light collection, each Nikhef PMT is equipped
with a \emph{light-cone} consisting of a circular reflective surface at \unit{45}{^\circ} to the
PMT normal. The Madison assembly is similar but has no PMT cover or light cone. For POMs attached
to either endcap their PMTs are angled at \unit{45}{^\circ} facing the direction of the beam to
furthermore to increase light collection.

All PMTs within a POM are connected to the lowest level of DAQ electronics contained within a
dedicated electronics box. Either an aluminium or PVC cylinder in the Nikhef or Madison case
respectively. A flexible PVC \emph{pigtail} is attached to each POM electronics box containing the
connections to the higher level DAQ system and power supply. A \emph{water-block} within the
pigtail ensures that even if the outside connection is flooded the POM is capable of withstanding
the \unit{6}{\mathrm{atm}} of pressure at the bottom of the pit. A fully assembled and installed
Nikhef POM is shown in Fig.~\ref{fig:single_plane} for reference. The POMs are tiled next to each
other on the detector walls, attached to either the stainless steel \emph{stringers} on the
top-cap and bottom-cap, or clipped to the Dyneema cables on the vertical walls.

\begin{figure} % PMT ASSEMBLY DIAGRAM %
    \centering
    \subcaptionbox{Disassembled}{%
        \includegraphics[height=5.5cm]{diagrams/4-chips/pmt_disassembled.jpg}%
    }
    \quad
    \subcaptionbox{Assembled}{%
        \includegraphics[height=5.5cm]{diagrams/4-chips/pmt_assembled.jpg}%
    }
    \caption[Disassembled and assembled Nikhef PMT housing components.]
    {Disassembled (a) and assembled (b) Nikhef PMT assembly components. The assembly comprises of
        a black PVC insert, a \unit{88}{\mathrm{mm}} HZC PMT, a transparent acrylic cover, and a
        reflective light cone. The PMT is first glued to the inside surface of the cover using a
        silicone based optical gel before a watertight seal is made between the insert and cover
        using an O-ring. The reflective light cone is simply clipped to the front of the cover
        before the whole assembly is glued into the POM PVC structure.}
    \label{fig:pmt_assembly}
\end{figure}

\begin{figure} % NIKHEF POM DIAGRAM %
    \includegraphics[width=\textwidth]{diagrams/4-chips/single_plane.jpg}
    \caption[Picture of a Nikhef POM.]
    {Picture of a single Nikhef full density POM installed on the top-cap of the \chipsfive
        detector. Both the inward facing and veto PMTs are visible as well as the planes pigtail
        and aluminium electronics container.}
    \label{fig:single_plane}
\end{figure}

As mentioned previously, full coverage high-density detector instrumentation is not required for
\chips detector modules. This is due to two main reasons. Firstly, only highly directional
accelerator beam events entering the detector from one side are to be studied. Therefore, the vast
majority of neutrino interaction Cherenkov radiation is deposited on a relatively small downstream
region of the detector walls. Secondly, as beam neutrinos predominantly have multi-\GeV energies,
the Cherenkov light their interactions yield is relatively large when compared to the sub-\GeV
neutrinos studied by other detectors such as Super-Kamiokande. Therefore, a lower number of PMTs
is required to capture adequate Cherenkov radiation from an interaction. Consequently, the
distribution of the percentage of the detector walls covered by sensitive PMT surface area
(\emph{photocathode coverage}) is optimised, to reduce the total number of PMTs used.

The \chipsfive detector is split into three distinct regions of PMT photocathode coverage whose
boundaries are roughly defined by their azimuth angle $\phi$ from the centre of the downstream
wall (at $\phi=0^{\circ}$). Firstly, a \emph{full-density} Nikhef POM region in the most
downstream $\phi=\pm75^{\circ}$ region of the detector (both endcaps and vertical walls) with a
$\sim3\%$ photocathode coverage. Secondly, a \emph{half-density} Nihkef POM region covering the
$\phi=\pm75^{\circ}$ to $\phi=\pm180^{\circ}$ region of the endcaps and the $\phi=\pm75^{\circ}$
to $\phi=\pm140^{\circ}$ region of the vertical walls with a $\sim1.5\%$ photocathode coverage.
Finally, a \emph{half-density} Madison POM region covering the $\phi=\pm140^{\circ}$ to
$\phi=\pm180^{\circ}$ region of the vertical walls with a $\sim0.8\%$ photocathode coverage.

Extensive studies have shown that the \chipsfive layout and photocathode coverage does not
drastically reduce performance while vastly reducing the number of required PMTs~\cite{blake2016}.
The Nikhef light cones go some way to increasing the `sensitive' surface area to an (admittedly
not as effective) $\sim9.5\%$ and $\sim4.7\%$ for full-density and half-density respectively. This
is still a very small coverage when compared to the $\sim40\%$ uniform photocathode coverage of
Super-Kamiokande and highlights how much more expensive general purpose experiments such as this
are because of this.

One possibility for PMT distribution is not to put any in the upstream regions of the detector,
however, these are found to be very useful for rejecting cosmic muon and NC events. Furthermore,
for this purpose \chipsfive is equipped with veto PMTs attached to the top-cap Nikhef planes but
facing directly upwards. The veto contains a total of $324$ \unit{88}{\mathrm{mm}} HZC PMTs for a
photocathode coverage of $\sim0.6\%$. Separated by a liner made from the same geomembrane material
as for the outer liner, the area between the POMs and the top liner within the top-cap structure
becomes the veto region, so about \unit{1.5}{\mathrm{m}}. A graphical rendering of all the top-cap
POMs is shown in Fig.~\ref{fig:top_cap}.

\begin{figure} % TOP CAP RENDER DIAGRAM %
    \includegraphics[width=\textwidth]{diagrams/4-chips/top_cap.png}
    \caption[Graphical rendering of the top-cap POMs.]
    {Graphical rendering of the top-cap POMs. Both the different photocathode coverage regions and
        the veto PMTs are clearly visible.}
    \label{fig:top_cap}
\end{figure}

\subsection{Filtration} %%%%%%%%%%%%%%%%%%%%%%%%%%%%%%%%%%%%%%%%%%%%%%%%%%%%%%%%%%%%%%%%%%%%%%%%%%
\label{sec:chips_detector_water} %%%%%%%%%%%%%%%%%%%%%%%%%%%%%%%%%%%%%%%%%%%%%%%%%%%%%%%%%%%%%%%%%

Also, while still on land, a lightproof and watertight liner is installed. Designed to isolate the
clean internal detector water from the external pit water and to prevent non-Cherenkov light from
reaching the PMTs, the liner is made from geomembrane, a flexible reinforced polymer membrane.
Commercially available in large rolls the liner is welded together during construction to form the
top, bottom, and sides. Note that the liner for the top-cap and vertical walls is not joined until

Also, while still on land, a lightproof and watertight liner is installed. Designed to isolate the
clean internal detector water from the external pit water and to prevent non-Cherenkov light from
reaching the PMTs, the liner is made from geomembrane, a flexible reinforced polymer membrane.
Commercially available in large rolls the liner is welded together during construction to form the
top, bottom, and sides. Note that while on land, the vertical wall liner is very loose, however,
once the detector expands during deployment is held nearly taut (the Dyneema takes the strain).

- Though remarkably clear the Wentworth pit water is not clean enough for the detector volume
where we require ~30m attenuation length. Need to pump water, to prevent algae blooms and
bacterial growth and remove particulates which are present in the water to begin with. An
umbilical resting on the pit floor will contain 2 fibres, 2 power cables and 2 flexible pipes for
filtering the internal water on shore. All connected to two shore huts, one for filtering and one
for data acquisition.
- Any pressure differences on the liner are negligible, but a small positive pressure is kept
inside the detector to prevent particulates coming in via and leaks etc...

\subsection{Construction and deployment} %%%%%%%%%%%%%%%%%%%%%%%%%%%%%%%%%%%%%%%%%%%%%%%%%%%%%%%%%
\label{sec:chips_detector_deployment} %%%%%%%%%%%%%%%%%%%%%%%%%%%%%%%%%%%%%%%%%%%%%%%%%%%%%%%%%%%%

- How it can grow if needed - buoyant top cap anchored to the bottom one, which when fully
deployed will rest on the bottom of the pit.

\subsection{Current status} %%%%%%%%%%%%%%%%%%%%%%%%%%%%%%%%%%%%%%%%%%%%%%%%%%%%%%%%%%%%%%%%%%%%%%
\label{sec:chips_detector_status} %%%%%%%%%%%%%%%%%%%%%%%%%%%%%%%%%%%%%%%%%%%%%%%%%%%%%%%%%%%%%%%%

- No liner between veto and main volume!
- In the summer of 2018 and 2019 work on deploying \chips proceeded. Unforeseen hehe!!

This highlights one of the clear advantages of the \chips concept. No physical structure is
required on the vertical walls of the detector. Alongside the easier deployment discussed in
Section~\ref{sec:chips_detector_deployment} and the significantly simplified engineering, this is
the key reason as to why the \chips concept uses cylindrical rather than spherical detector
modules.

\begin{figure} % WORK DIAGRAM %
    \centering
    \subcaptionbox{gwgw}{%
        \includegraphics[height=5.5cm]{diagrams/4-chips/work1.jpeg}%
    }
    \quad
    \subcaptionbox{asgag}{%
        \includegraphics[height=5.5cm]{diagrams/4-chips/work2.jpeg}%
    }
    \caption[afaga]
    {afag}
    \label{fig:work}
\end{figure}

\begin{figure} % PANORAMA DIAGRAM %
    \includegraphics[width=\textwidth]{diagrams/4-chips/pan_1.jpeg}
    \caption[Panorama of the inside of \chipsfive just before deployment.]
    {Panorama of the inside of \chipsfive just before deployment. The six deployed Madison POMs
        are visible in the foreground on the bottom endcap. Additionally, the flexible tube manifolds
        can be seen connecting each POM to the higher level DAQ electronics.}
    \label{fig:pan_1}
\end{figure}

\section{Monte Carlo event generation and simulation} %%%%%%%%%%%%%%%%%%%%%%%%%%%%%%%%%%%%%%%%%%%%
\label{sec:chips_monte_carlo} %%%%%%%%%%%%%%%%%%%%%%%%%%%%%%%%%%%%%%%%%%%%%%%%%%%%%%%%%%%%%%%%%%%%

- Indispensable tool in particle physics, during the design and data analysis stages. Allows for
optimisation studies, testing of event reconstruction techniques and the study of potential
physics sensitivity.
- Useful MC simulation will provide output matching observables in a real detector. For full
analysis simulations need to be validated fully to make sure they approximate reality well enough.

\subsection{Beam event generation} %%%%%%%%%%%%%%%%%%%%%%%%%%%%%%%%%%%%%%%%%%%%%%%%%%%%%%%%%%%%%%%
\label{sec:chips_monte_carlo_beam} %%%%%%%%%%%%%%%%%%%%%%%%%%%%%%%%%%%%%%%%%%%%%%%%%%%%%%%%%%%%%%%

\begin{figure} % CHIPS FLUX DIAGRAM %
    \includegraphics[width=0.8\textwidth]{diagrams/4-chips/flux.pdf}
    \caption[\numi neutrino flux at CHIPS.]
    {The \numi beam neutrino flux with cross-sections applied at the CHIPS detector location. Shown
        are the seperate contributions from the different neutrino types and sign. No oscillations
        have been applied.}
    \label{fig:flux}
\end{figure}

- Cosmic event rate in \cite{son2013}

- We take full advantage of the MINOS, \nova extensive simulations of the \numi beam for use in
CHIPS.
- The tau neutrino component is negligible and not predicted by the simulation
INFO: expected number of events per year etc...

\subsection{Cosmic event generation} %%%%%%%%%%%%%%%%%%%%%%%%%%%%%%%%%%%%%%%%%%%%%%%%%%%%%%%%%%%%%
\label{sec:chips_monte_carlo_cosmic} %%%%%%%%%%%%%%%%%%%%%%%%%%%%%%%%%%%%%%%%%%%%%%%%%%%%%%%%%%%%%

DIAGRAM: expected cosmic rate at different height plot
DIAGRAM: Cosmic rate given the water overburden diagram
- \unit{50}{\mathrm{m}} of overburden,
- CR muon rate
- In-spill CR occupancy
- CR event dead time

- These short spills are essential for \chips and other experiments in rejecting the massive
cosmic ray background. As all events outside the expected beam spill window at the detector can be
rejected.

\begin{figure} % COSMICS AROUND DETECTOR DIAGRAM %
    \includegraphics[width=0.8\textwidth]{diagrams/4-chips/cosmics.png}
    \caption[Cosmic muon rays around the CHIPS detector]
    {Simulated cosmic muon rays around the CHIPS detector. Note how they are generated within a
        box.}
    \label{fig:cosmics}
\end{figure}

\subsection{Detector simulation} %%%%%%%%%%%%%%%%%%%%%%%%%%%%%%%%%%%%%%%%%%%%%%%%%%%%%%%%%%%%%%%%%
\label{sec:chips_monte_carlo_sim} %%%%%%%%%%%%%%%%%%%%%%%%%%%%%%%%%%%%%%%%%%%%%%%%%%%%%%%%%%%%%%%%

- Geant4 framework, based on WCSim package, designed to allow easy creation of generic water
Cherenkov models in Geant4, extensively modified for \chips, mainly by adding additional features
to simplify the process of creating and loading different geometries to allow for faster testing
of deign ideas without needing to recompile. Does this with XML configuration files, for PMTs,
lightcones, geometries etc... QGSP BIC HP physics list, default for this type of detector in
geant4, 50.0m at 405.0nm attenuation length in water, material properties hardcoded, but can be
scaled by the configuration, come from Super-Kamiokande simulation, configuration loaded at
runtime, layout of PMTs defined in a unit cell containing any number of PMTs of any type, detector
walls are tiled with copies of the unit cell in zones, allowing for different regions of different
PMT density. and type. PMT glass thickness, diameter, protrusion distance, appropriate rayleigh
and mie scattering are included as well as blacksheet, photocathode reflectivity.

- Simulation simulates the passage of a generated events primary particles through the detector
assigning a random vertex atleast 1m away from the detector wall as its interaction vertex. When a
PMT is hit a

\begin{figure} % SIMULATED EVENT DISPLAY DIAGRAM %
    \includegraphics[width=\textwidth]{diagrams/4-chips/sim_event.png}
    \caption[sim event short]
    {$\nu_{\mu}$ CC quasi-elastic event with a single muon final state particle of energy
        1770.24 MeV}
    \label{fig:sim_event}
\end{figure}

\begin{figure} % DIGI DIAGRAM %
    \centering
    \subcaptionbox{\label{fig:digi_method}}{%
        \includegraphics[height=6cm]{diagrams/4-chips/digi_method.pdf}%
    }
    \quad
    \subcaptionbox{\label{fig:digi_likelihood}}{%
        \includegraphics[height=6cm]{diagrams/4-chips/digi_likelihood.pdf}%
    }
    \caption[Simulation PMT digitisaion function.]
    {(a) Digitisation function used within the simulation to convert incident photons to measured
        digitised charge. (b) Likelihood of a measured digitised charge being caused by a number
        of photons incident on a PMT.}
    \label{fig:digitisation}
\end{figure}

- The "signal" in all cases are CC interactions, therefore selection of nuel, anuel, numu and
anumu CC is the goal.
- Main background in CC numu selections are NC with charged pions.
- Main background in CC nuel selections is pi-zero NC, which can mimic the chracteristic EM
shower, due to its near certain decay into two photons.
- You get a small number of nuel intrinsic to the beam, they are just a background as they are
indistinguishabe from the nuel appreaence neutrinos.
- Once you have collected samples in all four cases, a fit is performed to the reconstructed
neutrino energy distributions to extract the four neutrino oscillation parameters.