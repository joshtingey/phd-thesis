\chapter{The \chips R\&D project} %%%%%%%%%%%%%%%%%%%%%%%%%%%%%%%%%%%%%%%%%%%%%%%%%%%%%%%%%%%%%%%%
\label{chap:chips}

In pursuit of answers to the open questions presented in the previous chapter, neutrino
experiments are becoming increasingly, and possibly prohibitively, expensive and impractical. This
trend is particularly true of the next generation of long-baseline experiments, DUNE and
Hyper-Kamiokande, with cost estimates reaching billions of dollars and construction times of
greater than half a decade. It is also telling that the vast majority of global research effort
goes into just these two future projects, such is their complexity, cost, and lead time.

It is clear that for detectors to remain practical and affordable into the future, a novel design
strategy is highly desirable. This approach is especially the case if megaton scale detectors are
ever to become a reality. While instrumentation will continue to improve with time, the statistics
of low event counts will always limit neutrino experiments until vastly larger detectors can be
built. Therefore, R\&D efforts must focus on such detectors now, whilst also attempting to
complement the current and upcoming generation of experiments.

The \chips R\&D project~\cite{adamson2013} aims to develop novel strategies and technologies for
very large yet `cheap as chips' water Cherenkov detectors. Primarily aimed for deployment in long
baseline accelerator beam scenarios, \chips aims to lower the cost per kt of sensitive mass to
between \$200k-\$300k. For comparison, the Super-Kamiokande detector cost approximately \$4
million/kt to build. As physics sensitivity depends on more than just sensitive mass, this
comparison is not entirely rigorous; however, it highlights the scale of possible cost savings.

This chapter aims to describe the fundamental aspects of the \chips R\&D project in detail.
Firstly, the \chips concept will be outlined along with both neutrino beam and Cherenkov detector
physics for context. The design, construction, deployment, and status of the \chipsfive prototype
detector will then follow. Finally, the Monte Carlo event generation and detector simulation
framework will be introduced to aid the discussion of work presented in this thesis.

\section{\chips concept} %%%%%%%%%%%%%%%%%%%%%%%%%%%%%%%%%%%%%%%%%%%%%%%%%%%%%%%%%%%%%%%%%%%%%%%%%
\label{sec:chips_concept} %%%%%%%%%%%%%%%%%%%%%%%%%%%%%%%%%%%%%%%%%%%%%%%%%%%%%%%%%%%%%%%%%%%%%%%%

The \chips concept is to deploy cylindrical water Cherenkov detector modules into deep bodies of
water on the Earth's surface such as lakes, reservoirs and flooded mine pits. Initially
constructed on land \chips detectors can be floated into position before being sunk. The water
above the sunken detector provides a modest overburden from cosmic rays, while the surrounding
water provides support for a lightweight detector structure. By removing the need for underground
excavation and expensive structural support, the cost of construction can be dramatically reduced.

Additionally, the common practice of building majority bespoke components is replaced by using
modern commercially available components wherever possible. The number of expensive elements, such
as photomultiplier tubes are also reduced by only considering multi-$\GeV$ accelerator beam
neutrino events, such that full high density and high coverage detector instrumentation is not
required.

Furthermore, \chips detectors are not only designed to be cheap but practical. Easy to build,
quick to deploy, and upgradable once operational, multiple detector modules can be flexibly
combined depending on available resources and funding. When compared to DUNE and Hyper-Kamiokande
both which require a large upfront budget and many years to construct, cheap \chips detector
modules can be deployed as needed in under a year by a relatively small team.

To date, \chips R\&D efforts have been based in the USA to exploit the NuMI beam before the end of
its lifetime. Plans are focused on the scaling of \chips detectors for the deployment of multiple
modules within the LBNF beam once operational. Collaborators from primarily University College
London, The University of Wisconsin Madison, and Nikhef are focused on multiple R\&D efforts,
outlined below, each aiming to prove the viability of a crucial component of the \chips concept.

\begin{itemize}
    \item \textbf{Detector construction:} Aiming to prove that the construction and deployment of
          \chips concept detector modules are possible. Two prototype detectors have so far been
          deployed. Firstly, the small \chipsm detector shown in Fig.~\ref{fig:chips_m}, deployed
          into a flooded mine pit in northern Minnesota during the summer of 2014~\cite{perch2015,
              pfutznerProto2017, pfutzner2017}. Secondly, the much larger \unit{5}{\mathrm{kt}}
          \chipsfive detector, deployed into the same pit during the summer of 2019 and detailed
          in Section.~\ref{sec:chips_detector}.

    \item \textbf{Water filtration:} Aiming to prove that adequate water purity can be achieved
          using cheap, commercially available filtration. Extensive studies have proven that by
          filtering water directly from bodies of water on the Earth's surface (including flooded
          mine pits), adequate photon attenuation lengths greater than \unit{100}{\mathrm{m}} are
          achievable~\cite{amat2017, campbell2020}.

    \item \textbf{Physics sensitivity:} Aiming to prove that \chips concept detectors (even the
          prototypes) can provide significant physics contributions alone or alongside the current
          and next generation of experiments. Single modules in the current NuMI beam (discussed
          in Section.~\ref{sec:chips_concept_beam}) and multiple modules in the future LBNF beam
          have been studied~\cite{pfutzner2017, adde2016, lang2015}.

    \item \textbf{Data acquisition:} Aiming to prove that a cheap data acquisition (DAQ) system
          using commercially available components and software is viable~\cite{eijk2018}. Outlined
          in Chapter.~\ref{chap:daq}, \chips implements a novel use of cheap single-board
          computers to collect photomultiplier tube data.

    \item \textbf{Event reconstruction and classification:} Aiming to prove that modern machine
          learning techniques can be successfully applied to large water Cherenkov concepts such
          as \chips. The primary contribution of this thesis (detailed in
          Chapter.~\ref{chap:cnn}), this work feeds directly into both the physics sensitivity
          studies mentioned above and design optimisation work.
\end{itemize}

\begin{figure} % CHIPS-M DIAGRAM %
    \includegraphics[width=0.6\textwidth]{diagrams/4-chips/chips_m.png}
    \caption[Picture of the \chipsm detector]
    {Picture of the \unit{3.3}{\mathrm{m}} high \chipsm detector just before deployment. Temporary
        floatation bags are attached to the top rim of the detector, while the umbilical cord
        carrying data, power, and filtered water is attached at the base.}
    \label{fig:chips_m}
\end{figure}

\subsection{The neutrino beam and off-axis alignment} %%%%%%%%%%%%%%%%%%%%%%%%%%%%%%%%%%%%%%%%%%%%
\label{sec:chips_concept_beam} %%%%%%%%%%%%%%%%%%%%%%%%%%%%%%%%%%%%%%%%%%%%%%%%%%%%%%%%%%%%%%%%%%%

\chips detectors will primarily study the appearance of $\nu_{e}$ oscillating from $\nu_{\mu}$
over a long-baseline. To generate a sufficient number of GeV scale $\nu_{\mu}$, a high-intensity
accelerator beam is required. Currently, only two such beams exist, the J-PARC based beam in Japan
used by the T2K experiment and the \numi beam in the USA used by \nova. Here we describe the
\numi~\cite{adamson2016} (Neutrinos at the Main Injection) beam as it is directly relevant to
current \chips efforts. However, it is essential to note that \chips detectors are designed to be
used in any neutrino beam, including the future \numi replacement, LBNF.

The \numi beam is an accelerator muon neutrino beam produced at Fermilab near Chicago in the USA.
Beginning operation in 2005 for the MINOS experiment, \numi was upgraded in 2013 to provide a
higher intensity and energy, principally to achieve a peak in neutrino energy near the
$\sim$\unit{1.5}{\GeV} $\nu_{\mu}\rightarrow\nu_{e}$ oscillation maximum for \nova. Currently the
\numi beam achieves an intensity above \unit{700}{\mathrm{kW}} (\unit{740}{\mathrm{kW}} peak)
making it the most powerful such beam in the world. A schematic of the \numi beamline
configuration is shown in Fig.~\ref{fig:numi_beam}.

\begin{figure} % NUMI BEAM DIAGRAM %
    \includegraphics[width=\textwidth]{diagrams/4-chips/numi_beam.png}
    \caption[Schematic of the main components of the \numi beam]
    {Schematic of the main components of the \numi beamline shown with dimensions. The MINOS and
        \nova near detectors and the MINERvA experiment are located just to the right of what is
        shown. Figure taken from Ref.~\cite{adamson2016}.}
    \label{fig:numi_beam}
\end{figure}

Every \unit{1.33}{\mathrm{seconds}} a \unit{10}{\mu\mathrm{s}} long spill of protons accelerated
to \unit{120}{\GeV} by the Main Injector ring are directed towards a stationary graphite target.
The resulting interactions create a shower of hadrons containing predominantly pions and kaons.
The hadrons are passed through a focusing system of two magnetic horns tuned principally to focus
charged pions along the beamline while rejecting other particles. After focusing, any surviving
hadrons are allowed to decay in flight to a beam of muon neutrinos in a \unit{675}{\mathrm{m}}
long decay pipe via the processes,
\begin{align} % NUMI DECAY EQUATIONS %
    \pi^{+} & \rightarrow\mu^{+}+\nu_{\mu}, \label{eq:pi_decays}   \\
    K^{+}   & \rightarrow\mu^{+}+\nu_{\mu}. \label{eq:kaon_decays}
\end{align}
The resulting muons also decay such that $\mu^{+}\rightarrow e^{+}+\nu_{e}+\bar{\nu}_{\mu}$
producing an intrinsic $\nu_{e}$ component as well as wrong sign $\nu_{\mu}$ contamination.

Alternatively, the polarity of the horns can be used to switch the dominant sign of the hadrons
focused, allowing \numi to operate as either a neutrino or antineutrino beam. These two modes of
operation are called \emph{forward horn current} and \emph{reverse horn current} for primarily a
neutrino or antineutrino beam composition respectively. Any remaining hadrons, alongside
electrons, muons and surviving primary protons are absorbed by rock downstream of the decay pipe,
leaving just the neutrino components of the beam.

Long-baseline neutrino experiments typically consist of a \emph{near} detector to measure the beam
neutrino composition at source and a much larger \emph{far} detector to measure the oscillated
composition at many hundreds of kilometres. The \numi beamline contains three detectors just
downstream of \unit{300}{\mathrm{m}} of rock: The MINERvA spectrometer~\cite{mcfarland2006}, the
near detector for MINOS (now used by MINERvA), and the near detector for \nova. \chips prototypes
within the \numi beam will not have a dedicated near detector; therefore, data from the above
detectors will be crucial for physics analysis in order to constrain the beam composition and
flux.

The \numi neutrino beam passes through the Earth's crust until it finally emerges in northern
Minnesota. This is where the MINOS, \nova, and prototype \chips far detectors are located (used to
be located in the MINOS case), as shown in Fig.~\ref{fig:numi_map}. The Minnesota state nickname
`land of 10,000 lakes' is not an overstatement, with a vast number of potential lakes for \chips
detector deployment. Additionally, intense iron ore mining on the `Iron Range' provides many
suitable disused (and now flooded) mine pits. The exact \chipsfive prototype detector location is
discussed in greater detail within Section.~\ref{sec:chips_detector}.

\begin{figure} % CHIPS LOCATION IN NUMI DIAGRAM %
    \includegraphics[width=0.7\textwidth]{diagrams/4-chips/numi_map.png}
    \caption[Map of detector locations in the \numi beam]
    {Map of the MINOS, \nova and \chips locations in the \numi beam as it surfaces in northern
        Minnesota. Shown is the expected neutrino event rate assuming no oscillations, with lines
        of constant L/E indicated by contours. The western extent of Lake Superior can be seen in
        the lower right of the map. Image taken from Ref.\cite{adamson2013}.}
    \label{fig:numi_map}
\end{figure}

Due to the kinematics of pion decay, whether the far detector is placed directly on the beam axis
or not can have a significant impact on the observed energy spectrum of beam neutrinos as shown in
Fig.~\ref{fig:numi_axis}. For neutrinos directly on the beam axis, there is a strong energy
dependence on the parent pion energy from Eq.~\ref{eq:pi_decays}. However, as the off-axis angle
increases, the neutrino energy becomes less dependent on the parent pion energy and is restricted
to a narrowing range of decreasing energies. Known as the \emph{off-axis effect} this is used by
both \nova and T2K to create a narrower energy spectrum focused on the
$\nu_{\mu}\rightarrow\nu_{e}$ oscillation maximum. Additionally, by reducing the tail of
higher-energy neutrinos, the number of background NC events can be greatly reduced, as is the case
for the \unit{7}{\mathrm{mrad}} off-axis \chipsfive prototype detector.

\begin{figure} % OFF-AXIS FLUX DIAGRAM %
    \includegraphics[width=0.6\textwidth]{diagrams/4-chips/numi_axis.png}
    \caption[Muon neutrino flux for different \numi detectors at different off-axis angles]
    {Muon neutrino flux for different \numi detectors at different off-axis angles. Shown are the
        neutrino energy spectrum for MINOS (on-axis), \nova (\unit{14}{\mathrm{mrad}} off-axis),
        and \chipsfive (\unit{7}{\mathrm{mrad}} off-axis). Figure taken from
        Ref.~\cite{adamson2013}.}
    \label{fig:numi_axis}
\end{figure}

\subsection{Water Cherenkov detectors} %%%%%%%%%%%%%%%%%%%%%%%%%%%%%%%%%%%%%%%%%%%%%%%%%%%%%%%%%%%
\label{sec:chips_concept_cherenkov} %%%%%%%%%%%%%%%%%%%%%%%%%%%%%%%%%%%%%%%%%%%%%%%%%%%%%%%%%%%%%%

The \chips detector concept is based upon the water Cherenkov technique for neutrino detection. A
large body of target water is instrumented with photomultiplier tubes (PMTs) to record the
Cherenkov radiation produced by sufficiently relativistic charged particles resulting from
neutrino interactions. By using readily available water as the target material and only
instrumenting the volume surface, water Cherenkov detectors provide the best detection methodology
for maximising volume and reducing cost.

Cherenkov radiation is emitted by all electrically charged particles travelling faster than the
local phase velocity of light in a dielectric medium. Similar to the sonic boom created by a
supersonic aircraft, Cherenkov radiation forms a shock wave of coherent light, as shown in
Fig.~\ref{fig:cherenkov}. Typically the emitted light has wavelengths in the optical to
ultraviolet range. When projected onto the detector wall, the resulting cone of radiation
generates a distinctive ring shape. The cone opening angle (the angle at which light is emitted)
$\theta_{c}$ is given by:
\begin{equation}
    \cos\theta_{c} = \frac{1}{\beta n(\lambda)},
    \label{eq:cherenkov_angle}
\end{equation}
where $\beta=v/c$ and $n$ is the refractive index of the medium~\cite{particle2020}. Note that $n$
is a function of the wavelength of emission $\lambda$, and so is the opening angle. As the
refractive index of water is $\sim 1.33$ for typical wavelengths of emission, and using the
ultrarelativistic limit $\beta\sim 1$ the opening angle is found to be $\sim41^{\circ}$.

\begin{figure} % CHERENKOV EFFECT DIAGRAM DIAGRAM %
    \includegraphics[width=0.6\textwidth]{diagrams/4-chips/cherenkov.png}
    \caption[Diagram of Cherenkov radiation emission]
    {Diagram of Cherenkov radiation emission and wavefront angles. The angles $\theta_{c}$ and
        $\eta$ are not equal in a dispersive medium. Figure taken from Ref.~\cite{particle2020}}
    \label{fig:cherenkov}
\end{figure}

In Eq.~\ref{eq:cherenkov_angle} there is no Cherenkov emission when $\cos\theta_{c} > 1$, which is
the case when $\beta n(\lambda)<1$. Therefore, a Cherenkov energy threshold $E_{t}$ exists for
charged particle, which when expressed in terms of the particle mass $m$ is given by:
\begin{equation}
    E_{t} = \gamma m = \frac{m}{\sqrt{1-(1/n)^{2}}}.
    \label{eq:cherenkov_threshold}
\end{equation}
Again using $n\sim 1.33$, a threshold energy of $\sim1.5m$ is typical.

The number of Cherenkov photons emitted by a particle of charge $ze$ per unit wavelength per unit
path length is given by:
\begin{equation}
    \frac{d^{2}N}{d\lambda dx}=\frac{2\pi\alpha z^{2}}{\lambda^{2}}
    \left(1-\frac{1}{1-\beta^{2}n^{2}(\lambda)}\right),
    \label{eq:cherenkov_emission}
\end{equation}
where $\alpha$ is the fine structure constant~\cite{particle2020}. Integrating over the range of
wavelengths for which PMTs are typically sensitive, \unit{350}{\mathrm{nm}} to
\unit{650}{\mathrm{nm}}, and using $\beta\sim 1$ and $n\sim 1.33$ gives $\sim240$ photons emitted
per cm travelled by the charged particle~\cite{perch2017}.

By analysing the Cherenkov ring (or rings) of light recorded by the PMTs on the walls of the
detector, information about the charged particle (particles) within an event can be determined.
The underlying neutrino interaction can then be understood indirectly. Primarily the challenge for
accelerator beam water Cherenkov detectors is the identification and reconstruction of an electron
or muon ring likely to have been produced from a beam neutrino and not a cosmic ray. This event
topology indicates a beam CC $\nu_{e}$ or CC $\nu_{\mu}$ event respectively, rejecting NC and
cosmic events.

The basic shape of a Cherenkov ring can be used to tell which charged particle created it. Muons
are long-lived and typically travel many metres within the detector, producing a clean ring with
sharp edges. Conversely, electrons almost immediately initiate an electromagnetic shower;
therefore, the observed ring is the sum of multiple rings produced from the individual electrons
and positrons within the shower. As a consequence, when compared to a muon ring, electron rings
are characteristically fuzzy. A factor of this difference can be seen in Fig.~\ref{fig:emission
    distance}, which shows how the total amount of Cherenkov radiation is emitted for both electrons
and muons as a function of distance from the interaction vertex.

\begin{figure} % EMISSION DISTANCE DIAGRAM %
    \includegraphics[width=0.6\textwidth]{diagrams/4-chips/emission_distance.pdf}
    \caption[Fraction of Cherenkov photons emitted as a function of distance]
    {The fraction of the total number of photons emitted as a function of the distance from the
        interaction vertex for both electrons and muons with energy of \unit{2.5}{\GeV}. Multiple
        particles within the electron-induced electromagnetic shower emit their Cherenkov
        radiation over a short distance and in slightly different directions. Conversely, a muon
        travels relatively much further emitting an approximately constant level of Cherenkov
        radiation as it does so, leading to a clean, sharp-edged ring.}
    \label{fig:emission distance}
\end{figure}

The situation becomes complicated when multiple charged particles above the Cherenkov threshold
are involved, common at multi-$\GeV$ energies. In this case, multiple overlapping rings are
observed, making reconstruction difficult. The worst-case scenario, however, is when two rings
entirely overlap, removing any ability to tell them apart. This topology is commonly the case for
NC interactions producing a $\pi^{0}$ in the final state, forming the primary background for CC
$\nu_{e}$ appearance.

$\pi^{0}$ particles decay to a pair of photons with a 98.82\% branching ratio, both which almost
immediately initiate an electromagnetic shower, just like an electron~\cite{particle2020}. This
process leads to two, electron like rings to be observed with a separation angle given by:
\begin{equation}
    (1-\cos\theta_{ij})=\frac{m_{\pi}^2}{2E_{i}E_{k}},
\end{equation}
where $m_{\pi}$ is the invariant mass of the $\pi^{0}$ and $E_{i}$ and $E_{j}$ are the energies of
the two photons respectively. Therefore, for a $\pi^{0}$ decaying to two \unit{1}{\GeV} photons,
there is just $\sim 8^{\circ}$ of separation between the rings, making them difficult to tell
apart. This distinction is especially hard when electron like rings are also fuzzy. Alternatively,
if the two photons have an unequal energy distribution, such that one is much more energetic than
the other, the higher energy photon ring can dominate, and the other can not be identified,
leading to what looks like a single electron ring, again a misidentification.

\section{The \chipsfive detector} %%%%%%%%%%%%%%%%%%%%%%%%%%%%%%%%%%%%%%%%%%%%%%%%%%%%%%%%%%%%%%%%
\label{sec:chips_detector} %%%%%%%%%%%%%%%%%%%%%%%%%%%%%%%%%%%%%%%%%%%%%%%%%%%%%%%%%%%%%%%%%%%%%%%

\chipsfive is the first large scale prototype detector module for the \chips project. At
\unit{25}{\mathrm{m}} wide and \unit{12}{\mathrm{m}} high, \chipsfive is cylindrical in shape with
an inner surface area of \unit{1924}{\mathrm{m}^2} and a total target mass of
\unit{5.9}{\mathrm{kton}}. Via the process of design, construction, deployment, and data taking,
\chipsfive primarily aims to refine the \chips concept for future full-scale
($\sim$\unit{15}{\mathrm{kton}}) modules. Consequently, \chipsfive is designed such that the
details outlined in this section are characteristic of what a full-sized \chips module is
envisioned to be.

First, the location, structure, instrumentation and water filtration are detailed for the complete
detector. A discussion of the construction and deployment procedure follows before the current
status is presented. The full electronics and DAQ details are instead given in
Chapter.~\ref{chap:daq}.

\subsection{Location} %%%%%%%%%%%%%%%%%%%%%%%%%%%%%%%%%%%%%%%%%%%%%%%%%%%%%%%%%%%%%%%%%%%%%%%%%%%
\label{sec:chips_detector_location} %%%%%%%%%%%%%%%%%%%%%%%%%%%%%%%%%%%%%%%%%%%%%%%%%%%%%%%%%%%%%

\chipsfive is located at the Wentworth 2W pit in northern Minnesota, USA, near the small town of
Hoyt Lakes. A disused and flooded surface Taconite ore (a type of iron ore) mine pit, Wentworth 2W
is located \unit{7}{\mathrm{mrad}} off the \numi axis at a distance of \unit{712}{\mathrm{km}}
from the beam target. Roughly \unit{0.8}{\mathrm{km}}$\times$\unit{1.2}{\mathrm{km}} in size with
a maximum depth of \unit{60}{\mathrm{m}} ($\pm$\unit{3}{\mathrm{m}} throughout the year), the pit
allows for an overburden of approximately \unit{50}{\mathrm{m}} with \chipsfive resting on the
bottom. With an average daily low temperate of $-24^{\circ}\mathrm{C}$ in January, the pit freezes
over during winter, therefore, work is only possible during the summer months of May to October.

A sizeable earthen ramp on the south side of the Wentworth 2W pit is used for detector
construction while on land. The construction site is easily accessible by road and well connected
to power due to the heavy infrastructure in place for mining. Additionally, the nearby PolyMet
mining administration building is used as a laboratory environment for the construction and
testing of individual components before installation within the detector. A labelled satellite
view of Wentworth 2W is given in Fig.~\ref{fig:pit} for context, with a picture of the
construction site shown in Fig.~\ref{fig:from_the_sky}.

\begin{figure} % PIT DIAGRAM %
    \includegraphics[width=\textwidth]{diagrams/4-chips/pit.pdf}
    \caption[Satellite view of the Wentworth 2W mine pit in northern Minnesota]
    {Satellite view of the Wentworth 2W flooded mine pit in northern Minnesota, showing key
        \chipsfive locations. The PolyMet building, shore huts, construction site and deployment
        location are shown. For both the construction site and deployment location, the red circle
        shows the \chipsfive detector size to scale.}
    \label{fig:pit}
\end{figure}

\begin{figure} % CHIPS FROM THE SKY DIAGRAM %
    \includegraphics[width=\textwidth]{diagrams/4-chips/from_the_air.jpeg}
    \caption[Picture of the \chipsfive construction site from the air]
    {Picture of the \chipsfive construction site from the air facing south. The Wentworth 2W pit
        is in the lower half of the image, with the part built \chipsfive detector visible at the
        bottom of the earthen construction ramp. The two white shore huts can just be seen halfway
        up the ramp.}
    \label{fig:from_the_sky}
\end{figure}

\subsection{Structure} %%%%%%%%%%%%%%%%%%%%%%%%%%%%%%%%%%%%%%%%%%%%%%%%%%%%%%%%%%%%%%%%%%%%%%%%%%%
\label{sec:chips_detector_structure} %%%%%%%%%%%%%%%%%%%%%%%%%%%%%%%%%%%%%%%%%%%%%%%%%%%%%%%%%%%%%

The structure of the \chipsfive detector module consists primarily of two \unit{26}{\mathrm{m}}
diameter and \unit{1.3}{\mathrm{m}} high lightweight stainless steel circular \emph{endcaps} that
form the top and bottom of its cylindrical shape. During construction on land the conveniently
named \emph{top-cap} is held above the \emph{bottom-cap} by \unit{1.5}{\mathrm{m}} long steel
struts as shown in Fig.~\ref{fig:frame}. This configuration allows for the endcap instrumentation,
detailed in Section.~\ref{sec:chips_detector_instrumentation}, to be easily installed.

\begin{figure} % CHIPS FRAME DIAGRAM %
    \includegraphics[width=\textwidth]{diagrams/4-chips/frame.jpeg}
    \caption[Picture of the \chipsfive structural frame]
    {Picture of the \chipsfive structural frame with humans for scale. The endcaps can be seen
        separated by steel struts. Rows of stainless steel \emph{stringers} are attached to the
        inside of each endcap to mount the instrumentation too.}
    \label{fig:frame}
\end{figure}

The two endcaps are connected by 28 \unit{12}{\mathrm{m}} long Dyneema cables around their
perimeter. Additionally, 48 \unit{16}{\mathrm{inch}} diameter air-filled PVC pipes are attached to
the frame of the top-cap, making it buoyant. Therefore, once deployed into the pit, the bottom-cap
sinks while the top-cap floats, this pulls the Dyneema cable until taut, forming the final
expanded detector shape, as shown in Fig.~\ref{fig:chips_render}.

\begin{figure} % CHIPS RENDER DIAGRAM %
    \includegraphics[width=0.6\textwidth]{diagrams/4-chips/chips_render_1.png}
    \caption[Graphical rendering of the \chipsfive detector]
    {Graphical rendering of the fully deployed and expanded \chipsfive detector module with a
        section of the liner cutaway. The bottom endcap and wall POMs are visible, as well as the
        top endcap structure and floatation. The green lines indicate the Dyneema cables holding
        the top-cap and bottom-cap together.}
    \label{fig:chips_render}
\end{figure}

A lightproof and watertight liner is also installed to surround the fully expanded structure.
Designed to isolate the clean internal water from the external pit water and to prevent
non-Cherenkov light from reaching the PMTs, the liner is made from geomembrane, a flexible
reinforced polymer membrane. Commercially available in large rolls the liner is welded together
during construction to form the top, bottom, and sides. Note that when fully deployed, the liner
does not take any of the structural strain.

\subsection{Instrumentation} %%%%%%%%%%%%%%%%%%%%%%%%%%%%%%%%%%%%%%%%%%%%%%%%%%%%%%%%%%%%%%%%%%%%%
\label{sec:chips_detector_instrumentation} %%%%%%%%%%%%%%%%%%%%%%%%%%%%%%%%%%%%%%%%%%%%%%%%%%%%%%%

The \chipsfive detector is instrumented with PMTs arranged within distinct plane like structures
called Planar Optical Modules (POMs), which take inspiration from the Digital Optical Modules
(DOMs) used by IceCube and KM3NeT~\cite{hanson2006, eijk2015}. Each POM is a roughly
\unit{2}{\mathrm{m}}$\times$\unit{3}{\mathrm{m}} array of watertight PVC tubing equipped with
anywhere between $15$ to $30$ PMTs, in addition to the lowest level of DAQ electronics and power
distribution. Standard commercially available schedule $40$ PVC piping and connectors are used to
form the structure of each plane, bound together with PVC primer and cement.

There are two types of POM used within \chipsfive, differentiated by the PMTs and the data
acquisition electronics they use. Both are named after the institution at which they were
primarily developed. Firstly, \emph{Nikhef} POMs use \unit{88}{\mathrm{mm}} HZC PMTs with
electronics developed by the KM3NeT experiment~\cite{katz2009, adrian2016}. Secondly,
\emph{Madison} POMs use \unit{3}{\mathrm{inch}} Hamamatsu R6091 PMTs donated from the NEMO3
experiment~\cite{arnold2005} with novel electronics developed by \chips in collaboration with the
Wisconsin IceCube Particle Astrophysics Centre (WIPAC) in Madison, Wisconsin. The
\unit{88}{\mathrm{mm}} HZC PMTs have a high ratio of output electrons to incident photons (quantum
efficiency) of 24.4\% at a wavelength of \unit{400}{\mathrm{nm}}, compared to the low 12.0\% ratio
achieved by the R6091 PMTs. Furthermore, the photon hit time resolution is $\sim2\mathrm{ns}$ and
$\sim$\unit{5}{\mathrm{ns}} for the \unit{88}{\mathrm{mm}} HZC and R6091 PMTs respectively.

In total $6114$ \unit{88}{\mathrm{mm}} HZC and $450$ Hamamatsu R6091 PMTs are arranged into $226$
Nikhef and $30$ Madison POMs. Every PMT is housed in an assembly as shown in
Fig.~\ref{fig:nikhef_pmt_assembly} for the Nikhef case and Fig.~\ref{fig:madison_pmt_assembly} for
the Madison case. Importantly, to increase the level of light collection, each Nikhef PMT is
equipped with a \emph{light-cone} consisting of a circular reflective surface at \unit{45}{^\circ}
to the PMT normal. The Madison PMT assembly is similar but has no cover or light cone. For POMs
attached to either endcap, their PMTs are angled at \unit{45}{^\circ} facing the direction of the
beam to increase light collection furthermore.

All PMTs within a POM are connected to the lowest level of DAQ electronics contained within a
dedicated onboard electronics box. Either an aluminium or PVC cylinder in the Nikhef or Madison
case respectively. A flexible PVC \emph{pigtail} is attached to each POM electronics box
containing connections to the higher level DAQ and power supply. A \emph{water-block} within each
pigtail ensures that even if the external connection is flooded, every POM is capable of
withstanding the \unit{6}{\mathrm{atm}} of water pressure at the bottom of the pit. A fully
assembled and installed Nikhef POM is shown in Fig.~\ref{fig:single_plane} for reference.

\begin{figure} % NIKHEF PMT ASSEMBLY DIAGRAM %
    \centering
    \subcaptionbox{Disassembled}{%
        \includegraphics[height=5.5cm]{diagrams/4-chips/pmt_disassembled.jpg}%
    }
    \quad
    \subcaptionbox{Assembled}{%
        \includegraphics[height=5.5cm]{diagrams/4-chips/pmt_assembled.jpg}%
    }
    \caption[Disassembled and assembled Nikhef PMT housing components]
    {Disassembled (a) and assembled (b) Nikhef PMT assembly components. The assembly comprises of
        a black PVC insert, a \unit{88}{\mathrm{mm}} HZC PMT, a transparent acrylic cover, and a
        reflective light cone. The PMT is glued to the inside surface of the cover using a
        silicone-based optical gel and a watertight seal is made between the insert and cover
        using an O-ring. The reflective light cone is clipped to the front of the cover and the
        whole assembly is glued into the POM PVC structure.}
    \label{fig:nikhef_pmt_assembly}
\end{figure}

\begin{figure} % MADISON PMT ASSEMBLY DIAGRAM %
    \centering
    \subcaptionbox{Outside}{%
        \includegraphics[angle=270,origin=c,height=4.3cm]{diagrams/4-chips/madison_pmt.jpeg}%
    }
    \quad
    \subcaptionbox{Inside}{%
        \includegraphics[height=6cm]{diagrams/4-chips/madison_assembly.jpg}%
    }
    \caption[Pictures of the Madison PMT assembly]
    {A Hamamatsu R6091 Madison POM PMT outside (a) and inside its insert (b). The PMT is
        \emph{potted} inside its black PVC insert creating a watertight seal that can withstand
        the \unit{6}{\mathrm{atm}} of water pressure at the bottom of the pit. A microDAQ detailed
        in Chapter.~\ref{chap:daq}, is seen attached to the base of the PMT in (a).}
    \label{fig:madison_pmt_assembly}
\end{figure}

\begin{figure} % NIKHEF POM DIAGRAM %
    \includegraphics[width=\textwidth]{diagrams/4-chips/single_plane.jpg}
    \caption[Picture of a Nikhef Planar Optical Module]
    {Picture of a single Nikhef full-density POM installed on the top-cap of the \chipsfive
        detector. Both the inward-facing and veto PMTs are visible as well as the aluminium
        electronics container and pigtail whose end is covered in green tape.}
    \label{fig:single_plane}
\end{figure}

TODO: Too much description without a plot to back it up. Jenny says we need a visual way to see
how we chose the are of high and low density POMs!

The POMs are tiled next to each other on the detector walls, attached to either the stainless
steel \emph{stringers} on the top-cap and bottom-cap, or clipped to the Dyneema cables on the
vertical walls of the \emph{barrel}. As mentioned previously, full high density and high coverage
detector instrumentation is not required for \chips detector modules for two main reasons.
Firstly, only highly directional accelerator beam events are to be studied. Therefore, the vast
majority of neutrino interaction Cherenkov radiation is deposited on a relatively small downstream
region of the detector walls. Secondly, beam neutrinos predominantly have multi-$\GeV$ energies,
yielding a relatively large amount of Cherenkov radiation. Therefore, a lower number of PMTs is
required to capture adequate Cherenkov radiation from each interaction.

Consequently, the distribution of the percentage of the detector walls covered by sensitive PMT
surface area (\emph{photocathode coverage}) is optimised to reduce the total number of PMTs. The
detector is split into three distinct regions of PMT photocathode coverage whose boundaries are
roughly defined by their azimuth angle $\phi$ from the centre of the downstream wall where
$\phi=0^{\circ}$. Firstly, a \emph{full-density} Nikhef POM region in the most downstream
$\phi=\pm75^{\circ}$ region of the detector with a $\sim3\%$ photocathode coverage. Secondly, a
\emph{half-density} Nihkef POM region covering the $\phi=\pm75^{\circ}$ to $\phi=\pm180^{\circ}$
region of the endcaps and the $\phi=\pm75^{\circ}$ to $\phi=\pm140^{\circ}$ region of the barrel
with a $\sim1.5\%$ photocathode coverage. Finally, a \emph{half-density} Madison POM region
covering the $\phi=\pm140^{\circ}$ to $\phi=\pm180^{\circ}$ upstream region of the barrel with a
$\sim0.8\%$ photocathode coverage. Studies have shown that this configuration results in a
negligible reduction in performance while vastly reducing the number of required
PMTs~\cite{blake2016}.

Compared to the $\sim40\%$ uniform photocathode coverage of Super-Kamiokande, the \chipsfive
instrumentation configuration highlights just how significantly different detector design can be
when only studying accelerator beam neutrinos. However, for cosmic muon and NC event rejection,
photocathode coverage in the upstream regions of the detector is still required.

To further help with cosmic muon rejection, the \chipsfive detector module is equipped with a veto
region within the top-cap frame structure. Separated from the main detector volume by a
geomembrane liner, the \unit{1.3}{\mathrm{m}} high region rejects predominantly downward going
cosmic muons by detecting the Cherenkov radiation they produce. A total of $324$ upward-facing
\unit{88}{\mathrm{mm}} HZC veto PMTs within the top-cap Nikhef POMs give a veto photocathode
coverage of $\sim0.6\%$. A graphical rendering of the full top-cap instrumentation, including the
veto PMTs, is shown in Fig.~\ref{fig:top_cap}.

\begin{figure} % TOP CAP RENDER DIAGRAM %
    \includegraphics[width=\textwidth]{diagrams/4-chips/top_cap.png}
    \caption[Graphical rendering of the top-cap Planar Optical Modules]
    {Graphical rendering of the top-cap POMs. Both the different photocathode coverage regions and
        the veto PMTs are visible.}
    \label{fig:top_cap}
\end{figure}

All \chipsfive instrumentation receives power and is connected to the highest level DAQ systems
through a \unit{500}{\mathrm{m}} long flexible PVC umbilical resting on the pit floor. The
umbilical contains two optical fibres for data and two shielded gauge 10 cables for power. One end
of the umbilical is attached to the bottom-cap of the detector while the other enters a hut on
shore containing the master power supply and highest level DAQ equipment.

\subsection{Filtration} %%%%%%%%%%%%%%%%%%%%%%%%%%%%%%%%%%%%%%%%%%%%%%%%%%%%%%%%%%%%%%%%%%%%%%%%%%
\label{sec:chips_detector_water} %%%%%%%%%%%%%%%%%%%%%%%%%%%%%%%%%%%%%%%%%%%%%%%%%%%%%%%%%%%%%%%%%

Though surprisingly clear, the Wentworth 2W pit water requires filtration in order to reach the
necessary \unit{30}{\mathrm{m}} attenuation length of light. Therefore, a high volume pump is
installed to pull detector water through a \unit{500}{\mathrm{m}} long flexible HDPE pipe
constantly to a filtration hut on the shore. After filtration, the water is returned to the
detector through a second identical pipe.

Within the hut, ten parallel sets of filters are installed in order to achieve a high flow rate.
This set up allows the full detector volume to be filtered every ten days. Each filter set
consists of a 20 inch \unit{10}{\mu\mathrm{m}} carbon block filter followed by a 20 inch
\unit{0.5}{\mu\mathrm{m}} polypropylene filter, as shown in fig.~\ref{fig:filtration}. This
configuration is found to achieve a light attenuation length of $133\pm2~m$ after a few months of
constant filtering~\cite{campbell2020}.

\begin{figure} % SIMULATION GEOM DIAGRAM %
    \includegraphics[width=0.6\textwidth]{diagrams/4-chips/filtration.jpg}
    \caption[Picture of the \chipsfive filtration system]
    {Picture of the \chipsfive filtration system within one of the shore huts. The pipes to and
        from the detector can be seen entering the hit on the back wall.}
    \label{fig:filtration}
\end{figure}

\subsection{Construction and deployment} %%%%%%%%%%%%%%%%%%%%%%%%%%%%%%%%%%%%%%%%%%%%%%%%%%%%%%%%%
\label{sec:chips_detector_deployment} %%%%%%%%%%%%%%%%%%%%%%%%%%%%%%%%%%%%%%%%%%%%%%%%%%%%%%%%%%%%

The construction and deployment of any \chips detector module such as \chipsfive is a novel and
relatively complex process when compared to other experiments. This complexity is primarily due to
a body of water being used for these activities rather than solid ground. Below a simplified
version of the \chipsfive construction and deployment procedure is outlined.

\begin{enumerate}
    \item An earthen barrier is built separating the main body of Wentworth 2W water from the
          construction area. As the pit water level rises during the summer, this acts as a dam
          preventing the construction area from flooding.
    \item The bottom-cap liner is welded together before the endcap structural frames are
          constructed above, resulting in that shown by Fig.~\ref{fig:frame}.
    \item The pre-assembled and tested endcap POMs are installed along with their associated power
          and data connections as well as other high-level DAQ components.
    \item The barrel liner is partly constructed starting from the detector base. Using a vertical
          liner roll for each of the 28 sides, the barrel liner is welded to the height of the
          top-cap, as is shown in Fig.~\ref{fig:chips_with_liner}. The top-cap liner is also
          installed but not welded to the barrel liner.
    \item A strongly buoyant and circular \emph{floating-dock}, made from steel, is constructed
          surrounding the detector. The bottom-cap is attached with metal chains to winches on the
          floating-dock and with Dyneema cables to winches on the top-cap.
    \item The earthen barrier is removed, and the construction area flooded, causing the detector
          and floating-dock structure to float. The detector is towed by a boat to its deployment
          location.
    \item The detector is slowly filled with water at the same time as the detector is lowered
          using the winches on the floating-dock. This process continues until the top-cap reaches
          the surface of the water and begins the float. At this point, the steel struts
          separating the endcaps are removed.
    \item The barrel POMs are installed in layers, brought to the detector location by boat. The
          gap between the barrel and top-cap liner allows this to happen. After a complete layer
          has been installed the bottom-cap is lowered using the winches on the floating-dock and
          top-cap and the barrel liner is correspondingly welded to a greater height before the
          procedure repeats.
    \item After all the POMs have been installed, the barrel and top-cap liner are welded
          together, the detector umbilical attached, and the whole detector lowered until it rests
          of the bottom of the pit using the floating-dock winches.
\end{enumerate}

\begin{figure} % CHIPS WITH LINER DIAGRAM %
    \includegraphics[width=\textwidth]{diagrams/4-chips/chips_with_liner_sun.jpeg}
    \caption[Picture of the \chipsfive detector module with liner]
    {Picture of the \chipsfive detector module with liner welded up to the height of the top-cap.
        A part of the floating-dock can also be seen in the foreground.}
    \label{fig:chips_with_liner}
\end{figure}

\subsection{Current status} %%%%%%%%%%%%%%%%%%%%%%%%%%%%%%%%%%%%%%%%%%%%%%%%%%%%%%%%%%%%%%%%%%%%%%
\label{sec:chips_detector_status} %%%%%%%%%%%%%%%%%%%%%%%%%%%%%%%%%%%%%%%%%%%%%%%%%%%%%%%%%%%%%%%%

During the summer of 2018 and 2019, extensive \chipsfive construction work was carried out by a
team of 10 to 15 collaborators at any one time. Given the novel nature of the project, most tasks,
some of which are pictured in Fig.~\ref{fig:work1} and Fig.~\ref{fig:work2}, proved challenging.
Interestingly, an ability to derive pleasure from permanently wet feet was found to be invaluable
for personal morale.

\begin{figure} % WORK DIAGRAM 1 %
    \centering
    \subcaptionbox{Cornerstone placement.}{%
        \includegraphics[height=6.5cm]{diagrams/4-chips/cornerstone.jpg}%
    }
    \quad
    \subcaptionbox{Floatation production.}{%
        \includegraphics[height=6.5cm]{diagrams/4-chips/work3.jpg}%
    }
    \caption[Some \chipsfive construction work]
    {Some \chipsfive construction work.}
    \label{fig:work1}
\end{figure}

\begin{figure} % WORK DIAGRAM 2 %
    \centering
    \subcaptionbox{Connection testing.}{%
        \includegraphics[height=5.6cm]{diagrams/4-chips/work1.jpeg}%
    }
    \quad
    \subcaptionbox{POM installation.}{%
        \includegraphics[height=5.6cm]{diagrams/4-chips/work4.jpg}%
    }
    \caption[More \chipsfive construction work]
    {More \chipsfive construction work.}
    \label{fig:work2}
\end{figure}

By late summer 2019, it became apparent that deployment of a fully instrumented detector would be
impossible before the pit started to freeze over in October. Therefore, the decision was taken to
only partially instrument the endcaps, leave the barrel walls bare, and reduce the module height
to \unit{8}{\mathrm{m}}. In October 2019 this version of \chipsfive (shown in
Fig.~\ref{fig:pan_1}) was deployed into Wentworth 2W with DAQ tests commencing shortly after.
Given the outbreak of the worldwide SARS-CoV-2 epidemic, all plans for work over the summer of
2020 were suspended, including essential repairs and instrumentation of the barrel walls. Plans
for summer 2021 are currently under review.

TODO: Maybe replace with a clearer picture before deployment
\begin{figure} % PANORAMA DIAGRAM %
    \includegraphics[width=\textwidth]{diagrams/4-chips/pan_1.jpeg}
    \caption[Panorama of the inside of \chipsfive just before deployment]
    {Panorama of the inside of \chipsfive just before deployment. Six total Madison POMs were
        deployed for testing visible in the foreground on the bottom-cap. Additionally, the
        flexible tube \emph{manifolds} can be seen connecting each POM to the higher level DAQ
        electronics and power supply.}
    \label{fig:pan_1}
\end{figure}

\section{Event generation and detector simulation} %%%%%%%%%%%%%%%%%%%%%%%%%%%%%%%%%%%%%%%%%%%%%%%
\label{sec:chips_monte_carlo} %%%%%%%%%%%%%%%%%%%%%%%%%%%%%%%%%%%%%%%%%%%%%%%%%%%%%%%%%%%%%%%%%%%%

TODO: Maybe put the simulation first and make it tie nicely into the previous detector
description. Jenny says the switch from detector description to event generation is a bit jarring.

Monte Carlo event generation and detector simulation methods are an indispensable tool within high
energy particle physics. This is particularly true during the design and prototyping phase of an
experimental project when no real-world data is available, as is the case with \chips. By matching
the observables of a real-world detector as close as possible, these methods allow for the
optimisation of designs, the testing of reconstruction techniques and the study of physics
sensitivities. Here, the beam and cosmic event generation procedures and a description of the
detector simulation used by \chips are described. All are employed extensively for the work
presented in Chapter.~\ref{chap:cnn} of this thesis.

\subsection{Beam event generation} %%%%%%%%%%%%%%%%%%%%%%%%%%%%%%%%%%%%%%%%%%%%%%%%%%%%%%%%%%%%%%%
\label{sec:chips_monte_carlo_beam} %%%%%%%%%%%%%%%%%%%%%%%%%%%%%%%%%%%%%%%%%%%%%%%%%%%%%%%%%%%%%%%

The expected flux of beam neutrinos at the \chipsfive detector location, shown in
Fig.~\ref{fig:flux}, is generated using the existing beam simulation written for the \numi
experiments. As the $\nu_{\tau}$ component is negligible, it is not predicted by the beam
simulation and ignored. Using the generated fluxes as input the GENIE neutrino event generator
(version 3.0.6)~\cite{andreopoulos2009, andreopoulos2015} is used to generate beam neutrino
events. Default cross-sections on water provided by GENIE are used. All initial, intermediate, and
final state particle tracks for each event are stored as output in a NUANCE formatted file for use
in the detector simulation. Note that unoscillated input fluxes are used such that analyses
samples must be later weighted to match the desired oscillated neutrino composition.

\begin{figure} % CHIPS FLUX DIAGRAM %
    \includegraphics[width=0.8\textwidth]{diagrams/4-chips/flux.pdf}
    \caption[\numi neutrino flux at at the \chipsfive detector location]
    {The neutrino mode (forward horn current) \numi beam neutrino energy spectrum at the
        \chipsfive detector module location. Shown are the individual contributions from the
        different neutrino types and signs. No cross-sections or oscillations have been applied.}
    \label{fig:flux}
\end{figure}

\subsection{Cosmic event generation} %%%%%%%%%%%%%%%%%%%%%%%%%%%%%%%%%%%%%%%%%%%%%%%%%%%%%%%%%%%%%
\label{sec:chips_monte_carlo_cosmic} %%%%%%%%%%%%%%%%%%%%%%%%%%%%%%%%%%%%%%%%%%%%%%%%%%%%%%%%%%%%%

The Cosmic-Ray Shower Library (CRY)~\cite{hagmann2012_1, hagmann2012_2} is used for cosmic ray
event generation. Both the solar cycle and Earth's geomagnetic field are taken into account, with
the \chipsfive latitude ($47.56^{\circ}$ N) and deployment date (1st November 2019) used as input.
Single muons are generated at sea level by CRY within a
\unit{1}{\mathrm{km}}$\times$\unit{1}{\mathrm{km}} area, with the detector at it's centre. Note
that only single muon events (the dominant cosmic component) are considered for simplicity.

Assuming a \chipsfive overburden of \unit{50}{\mathrm{m}} and a \unit{2.2}{\MeV/\mathrm{cm}^{2}}
muon energy loss in water as suggested by Ref.~\cite{klimushin2001} the muon parameters are
updated to estimate their values \unit{1}{\mathrm{m}} above the top of the detector. All muons
whose path does not cross the detector volume or do not have sufficient energy to reach the
detector are discarded~\cite{chipsgen2020}. All accepted muon tracks are stored as output in a
NUANCE formatted file for use in the detector simulation.

In order to use generated cosmic events in analysis, studies have looked at the likely cosmic rate
for \chips detector modules at different water overburden depths~\cite{son2013}. In this work, the
fits shown in Fig.~\ref{fig:cosmic_rate} for a cylindrical detector of both height and diameter
\unit{24}{\mathrm{m}} are used to estimate a \chipsfive cosmic muon rate of
\unit{11.8}{\mathrm{KHz}} with \unit{50}{\mathrm{m}} of overburden.

Given the \unit{10}{\mu\mathrm{s}} long \numi beam spill occurring every \unit{1.33}{\mathrm{s}}
an in spill cosmic rate of $\sim2.1$ million events per year is calculated with an in spill
occupancy of 9\%. Considering a typical event takes $\sim$\unit{100}{\mathrm{ns}} to unfold, there
is approximately a 0.3\% chance that any beam event overlaps with a cosmic muon. This low
coincidence shows just how powerful a short beam spill can be at reducing the significant cosmic
background.

\begin{figure} % COSMIC RATE DIAGRAM %
    \includegraphics[width=0.6\textwidth]{diagrams/4-chips/cosmic_rate.png}
    \caption[Expected \chipsfive cosmic muon rate as a function of water overburden depth]
    {Expected cosmic muon rate as a function of water overburden depth for a \unit{24}{\mathrm{m}}
        high and \unit{24}{\mathrm{m}} wide \chips detector module. Shown are fits made to the
        work originally conducted in Ref.~\cite{bugaev1998}. Figure taken from
        Ref.~\cite{son2013}.}
    \label{fig:cosmic_rate}
\end{figure}

\subsection{Detector simulation} %%%%%%%%%%%%%%%%%%%%%%%%%%%%%%%%%%%%%%%%%%%%%%%%%%%%%%%%%%%%%%%%%
\label{sec:chips_monte_carlo_sim} %%%%%%%%%%%%%%%%%%%%%%%%%%%%%%%%%%%%%%%%%%%%%%%%%%%%%%%%%%%%%%%%

The detector simulation uses the WCSim water Cherenkov simulation package~\cite{wcsim2020} built
on top of the Geant4 simulation framework~\cite{agostinelli2003, allison2006, Allison2016}.
Developed initially to simulate possible water Cherenkov detectors in the LBNE beam (now LBNF),
WCSim is now used more widely in the field. Heavily modified for the \chips project, WCSim allows
for generic water Cherenkov detector geometries to be easily loaded at runtime via a series of
simple XML configuration files. These changes allow for a broad range of detector geometries to be
quickly considered without recompilation of the code.

The simulation builds an n-sided, regular polygonal prism consisting of two endcaps and a barrel,
filled with water and lined with a low reflectivity \emph{blacksheet}. The geometry is separated
into \emph{regions} within both the barrel and endcaps, defined either by a list of barrel sides
or an opening angle respectively. Each region is filled with a unique base unit of geometry known
as the \emph{unit cell}, as shown in Fig.~\ref{fig:sim_geom}.

The unit cell defines a pattern of any number of PMTs, as well as their relative positions and in
which direction they face. The final geometry is built by tiling each of the defined regions with
their respective unit cell scaled to match the required regional photocathode coverage. Note that
although exact PMT positions are not used in this procedure, a given configuration will always
generate the same geometry (it is deterministic). In this work, the \chipsfive geometry is
generated with 28 sides and regions matching the angles and photocathode coverage detailed in
Section.~\ref{sec:chips_detector_instrumentation}.

\begin{figure} % SIMULATION GEOM DIAGRAM %
    \includegraphics[width=0.5\textwidth]{diagrams/4-chips/sim_geom.png}
    \caption[Illustrative diagram of a WCSim detector geometry]
    {Illustrative diagram of a WCSim detector geometry, showing the shape, endcap and barrel
        regions, tiled unit cells and PMTs within a unit cell. Figure taken from
        Ref.~\cite{blake2016}.}
    \label{fig:sim_geom}
\end{figure}

The geometry shape, regions, and unit cells are defined in a configuration file. Additionally, a
file for PMT definitions containing their shape, time resolution, and quantum efficiency is
defined. Light cones are described by a list of radial profile points in a further file. Although
the underlying Geant4 material properties are mostly hardcoded (taken from the Super-Kamiokande
simulation) values within yet another configuration file can scale them. This scaling controls the
water absorption and scattering (Rayleigh and Mie) lengths, and both the blacksheet and PMT glass
reflectivity. In this work an attenuation length of \unit{50}{\mathrm{m}} at
\unit{405}{\mathrm{nm}} is used with negligible scattering, the blacksheet reflectivity is set to
be 4\% and the PMT glass reflectivity 24\%.

A veto volume can also be defined. The veto is built as either a concentric shell around the whole
inner volume with a given thickness or solely above the top-cap with a given height. Any PMTs
defined as facing outwards within a unit cell look into the veto volume instead of the inner
volume. In this work, the \chipsfive geometry is given a top-cap veto of height
\unit{1.3}{\mathrm{m}}.

Once the full generation of the runtime Geant4 geometry is complete, the final state tracks for
each successive event to be simulated are loaded from either the beam or cosmic event generator
NUANCE files. Beam event vertices are randomly placed within the inner detector volume, while
cosmic vertices are kept at \unit{1}{\mathrm{m}} above the detector volume. WCSim then simulates
the passage of all particles through the detector materials, with interactions, decays, and
Cherenkov emission also considered.

Whenever a photon is calculated to have hit the photocathode of a PMT, an angular dependent
acceptance efficiency is applied to see if it is recorded. If accepted all hits within
\unit{200}{\mathrm{ns}} windows are grouped to form a single recorded hit, with the smeared first
hit time used as the recorded time. A method similar to that employed in the Super-Kamiokande
simulation determines the total output charged of the hit given the number of incident photons. A
single photoelectron charge distribution is repeatedly probed for each photon, and the combined
sum returned. By sampling this procedure multiple times, the output charge probability
distribution given the number of incident photons is shown in Fig.~\ref{fig:digitisation}. The
figure also shows the reverse \emph{likelihood} function (a lower value is more likely) of an
actual number of incident photons given a measured digitised charge.

\begin{figure} % DIGI DIAGRAM %
    \centering
    \subcaptionbox{\label{fig:digi_method}}{%
        \includegraphics[height=6cm]{diagrams/4-chips/digi_method.pdf}%
    }
    \quad
    \subcaptionbox{\label{fig:digi_likelihood}}{%
        \includegraphics[height=6cm]{diagrams/4-chips/digi_likelihood.pdf}%
    }
    \caption[Detector simulation PMT digitisation function]
    {(a) The probability function for the PMT photon digitisation process used in the simulation
        converting the number of incident photons to a measured charge. (b) Likelihood of a
        measured digitised charge being caused by a number of photons incident on a PMT.}
    \label{fig:digitisation}
\end{figure}

All fully simulated PMT hits for each event are stored in an output ROOT file along with important
truth information and track descriptions. A full event takes approximately
\unit{3}{\mathrm{seconds}} to simulate on a standard batch farm computing node. An example output
event display for a CC $\nu_{\mu}$ event is shown in Fig.~\ref{fig:sim_event}.

\begin{figure} % SIMULATED EVENT DISPLAY DIAGRAM %
    \includegraphics[width=\textwidth]{diagrams/4-chips/sim_event.png}
    \caption[Event display of a simulated beam CC $\nu_{\mu}$ \chipsfive event]
    {Event display of a simulated CC $\nu_{\mu}$ quasi-elastic event with a single muon in the
        final state of energy \unit{1.77}{\GeV}. The display shows both the unrolled barrel of the
        \chipsfive detector as well the two endcaps. Every coloured entry represents a hit PMT
        with the color indicating the total photoelectrons (charge) collected. The pink ring is a
        projection of the true Cherenkov light cone associated with the muon track from the
        interaction vertex.}
    \label{fig:sim_event}
\end{figure}