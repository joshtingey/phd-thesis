\chapter{Summary and conclusion} %%%%%%%%%%%%%%%%%%%%%%%%%%%%%%%%%%%%%%%%%%%%%%%%%%%%%%%%%%%%%%%%%
\label{chap:conclusion} %%%%%%%%%%%%%%%%%%%%%%%%%%%%%%%%%%%%%%%%%%%%%%%%%%%%%%%%%%%%%%%%%%%%%%%%%%

This thesis has presented a range of work for the \chips neutrino detector R\&D project. \chips
puts forward a novel water Cherenkov based concept to counter the vast expense, increased
complexity, and long construction time expected from future long-baseline neutrino oscillation
experiments, allowing for affordable and practical megaton scale detectors. As detailed, this feat
is possible via a series of steps: deploying detector modules in bodies of water on the Earth's
surface rather than deep underground; using commercially available rather than bespoke components
wherever possible; and optimising photocathode coverage to study accelerator beam neutrinos
exclusively. These steps reduce the total cost per kt of target mass to between \$200k-\$300k. 

Moreover, \chips detector modules are expected to be relatively easy to build, quick to deploy,
and can be upgraded once operational, making them a much more attractive proposition when
resources are constrained. It is hoped that future \chips detectors will be able to help answer
some of the key unsolved questions in neutrino physics, such as the neutrino hierarchy ambiguity
and the search for CP violation in neutrino oscillations.

During the summer of 2019 a \chips prototype, the \chipsfive detector was deployed into the
Wentworth 2W disused mine pit in northern Minnesota, USA. Although \chipsfive is not yet fully
proven and plans are in flux, the amount of knowledge gained during its construction, deployment,
and initial commissioning can not be understated. Alongside proving that a large \chips detector
module can be constructed and deployed, \chipsfive acted as a brilliant testbed for the
development of the \chips data acquisition system. 

Notably, the use of commercially available single-board Beaglebone machines and the open-source
Elasticsearch monitoring solution were found to be highly successful. Within both future \chips
detectors and the broader experimental particle physics field, it is clear that bespoke components
should continue to be phased out, with commercially available or open-source components used
instead. Not only does this drastically reduce the implementation effort required by
collaborators, but leads to a much improved final result due to the pooling of resources. 

As is the case within the world around us, future \chips concept detectors will make
ever-increasing use of modern deep learning techniques. This comes as a direct result of the
dramatic improvement in \chipsfive reconstruction and classification performance brought about by
the principal work presented in this thesis. Three forms of a Convolutional Neural Network have
been trained to reject cosmic muon events, classify beam events, and estimate neutrino energies,
all using only lightly modified versions of the raw detector events as input. This new approach
replaces the standard likelihood-based reconstruction and simple neural network classification,
greatly increasing generalisability and processing speed, whilst reducing human influence (bias).

With the primary goal of selecting an efficient and pure appeared CC $\nu_{e}$ signal, the new
approach is found to provide excellent performance. The vast cosmic muon background of \chipsfive
is found to be rejected (without the help of a veto) by a factor of $\sim1.5\times10^{-8}$,
leading to negligible contamination of the beam selection. Beam classification performance is
found to be surprisingly comparable to the much more complex \nova and T2K experiments, with a
$70.9\pm0.4\%$ efficiency and $71.0\pm0.6\%$ purity of the selected CC $\nu_{e}$ sample when both
the appeared and intrinsic CC $\nu_{e}$ beam components are considered signal. Similarly, neutrino
energy estimation is relatively comparable to other experiments. A 10.3\% and 12.6\% neutrino
energy resolution on signal CC $\nu_{e}$ and CC $\nu_{\mu}$ events respectively, is achieved,
falling to 7.1\% and 6.2\% when just quasi-elastic events are considered. 

Not only are the new networks found to provide excellent performance, but their inner workings are
found to be explainable and their outputs robust to distributional changes in the input. These
efforts rebut the common and justified concern that they are too often just used as a black box
(input in, outputs out). Cherenkov ring and Hough peak features are clearly extracted from the
input images when looking at the output feature maps. This results in a learnt representation of
the inputs that is seen to have strong discriminating power between classification categories when
visualised using the t-SNE technique. Additionally, realistic modifications to the input hit time
and charges, as well as the addition of random noise, are all found to have a negligible effect on
output performance.

------------------

It is sincerely hoped that other water Cherenkov neutrino experiments will take inspiration from
the work presented in this thesis for their own Convolutional Neural Network implementations.
Although the results presented in this work are compelling, there are still many avenues for
exploration and improvement in any future work. These are principally related to the critical
performance drivers outlined within the thesis. Firstly, any efforts to remove detector induced
distortions in the input event maps and improve the estimation of the interaction vertex position
should increase performance. Secondly, comprehensively study the training samples used, different
tasks are probably optimised using different samples, here we found they had to match, such as a
training sample of only contained events for beam classification. Alternatively, a loss function
that focuses on the physics sensitivity rather than simple accuracy may allow for different
training samples to be used. Finally, multi-task learning clearly shows promise, additional tasks
should be used to separate exclusive final states, for example, increasing performance. Only by
further and more widespread use will the full potential of such methods become truly apparent. It
is very likely we do not know yet! Real power not understood. Time will tell what additional
impact they may have within neutrino physics.