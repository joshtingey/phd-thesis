\chapter{Introduction} %%%%%%%%%%%%%%%%%%%%%%%%%%%%%%%%%%%%%%%%%%%%%%%%%%%%%%%%%%%%%%%%%%%%%%%%%%%
\label{chap:introduction} %%%%%%%%%%%%%%%%%%%%%%%%%%%%%%%%%%%%%%%%%%%%%%%%%%%%%%%%%%%%%%%%%%%%%%%%
\setcounter{page}{17}  % Need to explicitly set page number due to hepthesis class!! CHECK THIS!!!

It is abundantly evident that machine learning and the broader field of artificial intelligence
will play an ever-increasing role in the world around us over the coming years. Virtually every
industry and human on the planet is promised to be touched by this proliferation. Whether this
change has a positive or detrimental effect on society as a whole is questionable, however, within
the field of physics, especially experimental particle physics, machine learning applications can
only be seen in a positive light.

Already, a widespread revolution is underway across the field. From the way particle interactions
are simulated to the way recorded events are analysed, machine learning techniques are producing
dramatic improvements. This is especially true of the study of the vastly abundant yet incredibly
difficult to detect neutrino. For multiple years now, a large number of neutrino experiments have
been using deep learning methods for primarily event analysis. This is principally driven by the
fact that the raw output from neutrino detectors is effectively a simple image which is well
suited to the methods of the computer vision field.

However, there has currently been no fully-fledged implementation of modern deep learning
techniques to the full event reconstruction and classification of cylindrical water Cherenkov
detectors. This is particularly important as long-baseline accelerator beam experiments with a
water Cherenkov detector are an incredibly promising (and cheap) channel for solving some of the
key unsolved problems in neutrino physics.

This thesis presents a broad range of work conducted for the CHerenkov detectors In mine PitS
(\chips) neutrino detector R\&D project. The author's principal contribution comes from a novel
application of Convolutional Neural Networks to the reconstruction and classification of neutrino
events within the \chipsfive prototype detector module. This is the first fully-fledged
implementation of such methods to a long-baseline accelerator beam water Cherenkov neutrino
detector, covering the full range of event reconstruction and classification analysis. 

To begin, Chapter.~\ref{chap:theory} introduces the history of neutrino physics, as well as the
theoretical background and current status of the field as motivation for the broader \chips
project. A full description of the \chips concept follows in Chapter.~\ref{chap:chips}, with a
principle focus on the \chipsfive prototype detector module deployed into the \numi neutrino beam
during the autumn of 2019. A particularly detailed overview of the \chipsfive data acquisition and
monitoring systems is then given in Chapter.~\ref{chap:daq}, for which the author also played a
large role.

Chapter.~\ref{chap:cnn} firstly introduces the standard neutrino event reconstruction and
classification methods to be replaced alongside a review of the relevant neural network theory and
other deep learning applications within the neutrino field. The chapter concludes by detailing the
three Convolutional Neural Networks developed for \chipsfive, including that for cosmic muon
rejection, beam event classification, and neutrino energy estimation.

The thesis concludes with a comprehensive evaluation of the train Convolutional Neural Networks in
Chapter.~\ref{chap:results}. Firstly, the final combined performance is determined and compared
with similar experiments. Secondly, the inner workings of the trained networks are explored.
Thirdly, the robustness of the network outputs to distributional input changes is studied.
Finally, alternative implementations are discussed to highlight the key factors driving
performance.

It is hoped that the work presented in this thesis will inform the development of similar
Convolutional Neural Network applications for other water Cherenkov detectors. The scope of
application has not fully been explored, with a considerable amount of possibilities for the
future. 
