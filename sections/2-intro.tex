\chapter{Introduction} %%%%%%%%%%%%%%%%%%%%%%%%%%%%%%%%%%%%%%%%%%%%%%%%%%%%%%%%%%%%%%%%%%%%%%%%%%%
\label{chap:introduction} %%%%%%%%%%%%%%%%%%%%%%%%%%%%%%%%%%%%%%%%%%%%%%%%%%%%%%%%%%%%%%%%%%%%%%%%
\setcounter{page}{17}  % Need to explicitly set page number due to hepthesis class!! CHECK THIS!!!

Consider a two-dimensional plane onto which thousands of individual points are randomly placed.
Now suppose that the points are instead arranged so that combined (on a macro scale) they form the
shape of a ring (an annulus). Not a perfect ring, but a ring with a thickness and whose
constituent points are smeared to make it noticeably `fuzzy'. If you were to see a picture of this
ring for the first time, how would your process of understanding unfold?

Initially, you may notice edges between areas with few points and those with many. After
considering how these edges interconnect, you may begin to understand that they form a circular
shape. This shape is then probably refined once you notice the central hole and realise it is, in
fact, a ring. You would then start to consider its size, its diameter, where is the plane it is
located, and whether it is elliptical in nature. Upon detailed inspection, you may observe an
internal structure to the ring with a changing density of points. Given enough time you could
start to produce quantitative properties: the number of constituent points, the location of its
centre, even a density profile.

This is fundamentally the primary event analysis task for water Cherenkov neutrino detectors.
Neutrino interactions within the detector create rings of light that are recorded on the walls of
the detector, these are used to infer the properties on the underlying neutrino interaction. The
methodology of starting from the fundamental base features of an image and slowly building up more
and more detailed and complex features of an image is the is what Convolutional Neural Networks
are a deep learning algorithm that uses a process of understanding very similar to the one above,
just like your brain. 

This thesis presents a broad range of work conducted for the CHerenkov detectors In mine PitS
(CHIPS) neutrino detector R\&D project. The main contribution comes from the application of
Convolutional Neural Networks, a type of "deep learning" machine learning algorithm, to the
reconstruction and classification of neutrino events within the CHIPS-5 prototype detector.

The neutrino physics history, theoretical background, and current status are detailed as
motivation for the CHIPS project. A full description of the CHIPS detector concept follows, with a
principle focus on the CHIPS-5 prototype detector module deployed within the NuMI neutrino beam in
the autumn of 2019. A particularly detailed overview of the data acquisition and monitoring
systems developed for the CHIPS-5 detector is also given.

The standard neutrino event reconstruction and classification methods to be replaced are discussed
alongside a review of the relevant neural network theory and other "deep learning" applications
within the field. The Convolutional Neural Network implementations for CHIPS are then outlined,
including that for cosmic muon rejection, beam event classification and energy estimation.

A comprehensive evaluation of the trained Convolutional Neural Networks is then presented.
Firstly, the final combined performance is determined and compared with similar experiments.
Secondly, the inner workings of the trained networks are explored. Thirdly, the robustness of the
network outputs to distributional input changes is studied. Finally, alternative implementations
are discussed to highlight the key factors driving performance.

This is the first fully-fledged implementation of Convolutional Neural Networks to long-baseline
accelerator beam water Cherenkov neutrino detectors. Covering the full range of event
reconstruction and classification analysis. Hopefully, this work will inform the development of
similar applications and help with the optimisation of future \chips detector designs. 