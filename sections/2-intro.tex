\chapter{Introduction} %%%%%%%%%%%%%%%%%%%%%%%%%%%%%%%%%%%%%%%%%%%%%%%%%%%%%%%%%%%%%%%%%%%%%%%%%%%
\label{chap:introduction} %%%%%%%%%%%%%%%%%%%%%%%%%%%%%%%%%%%%%%%%%%%%%%%%%%%%%%%%%%%%%%%%%%%%%%%%
\setcounter{page}{17}  % Need to explicitly set page number due to hepthesis class!! CHECK THIS!!!

It is abundantly evident that machine learning and the broader field of artificial intelligence
will play an ever-increasing role in the world around us over the coming years. Virtually every
industry and human on the planet is promised to be touched by their proliferation. Whether this
change has a positive or detrimental effect on society as a whole is still questionable, however,
within the field of physics, especially experimental particle physics, machine learning
applications can only be seen in a positive light.

Already, a widespread revolution is underway across the field. From the way particle interactions
are simulated to the way recorded events are analysed, machine learning techniques are producing
dramatic improvements. This is particularly true of the study of the vastly abundant yet
incredibly difficult to detect neutrino. For a few years now, a range of neutrino experiments have
successfully been employing modern machine learning methods, usually deep learning algorithms,
primarily for event analysis. This application has principally been driven by the fact that the
raw output from neutrino detectors is well suited to the algorithms at the forefront of computer
vision research.

However, to date, a fully-fledged implementation of such techniques has not been made to the
reconstruction, classification, and energy estimation of neutrino events within long-baseline
water Cherenkov neutrino detectors. This is particularly important as the field moves to a new
generation of experiments, where long-baseline accelerator beam experiments using a water
Cherenkov detector are an incredibly promising (and surprisingly cheap as will be seen) channel
for solving some of the critical unsolved problems within neutrino physics.

This thesis presents a broad range of work conducted for the CHerenkov detectors In mine PitS
(\chips) neutrino detector R\&D project. The author's principal contribution comes from the novel
application of Convolutional Neural Networks to the reconstruction, classification, and energy
estimation of neutrino events within the \chipsfive prototype detector module.

To begin, Chapter.~\ref{chap:theory} introduces the history of neutrino physics, as well as the
theoretical background and current status of the field as motivation for the broader \chips
project. A full description of the \chips detector concept follows in Chapter.~\ref{chap:chips},
with a principle focus on the \chipsfive prototype detector module deployed into the \numi
neutrino beam during the autumn of 2019. A particularly detailed overview of the \chipsfive data
acquisition and monitoring systems is then given in Chapter.~\ref{chap:daq}, for which the author
played a primary role.

Chapter.~\ref{chap:cnn} begins by introducing the standard reconstruction and classification
methods to be replaced alongside a review of the relevant neural network theory and current
machine learning applications within the field of neutrino physics. The chapter concludes by
detailing the three Convolutional Neural Networks developed for \chipsfive, including that for
cosmic muon rejection, beam event classification, and neutrino energy estimation.

The thesis closes in Chapter.~\ref{chap:results} with a comprehensive evaluation of the new
Convolutional Neural Network approach. Firstly, the final combined performance is determined and
compared with similar experiments. Secondly, the inner workings of the trained networks are
explored. Thirdly, the robustness of the network outputs to distributional changes in the input is
studied. Finally, alternative implementations are discussed to highlight the key factors driving
performance.

It is hoped that the work presented in this thesis will achieve two principle goals. Firstly,
motivate the need for a detector concept such as \chips for the long term success of the field of
neutrino physics. Secondly, inform the development of similar Convolutional Neural Network
applications for other water Cherenkov neutrino detectors. The real power of such techniques is
still not fully understood, and only time will tell what uses they may have for neutrino physics.