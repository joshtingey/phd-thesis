\chapter{Introduction} %%%%%%%%%%%%%%%%%%%%%%%%%%%%%%%%%%%%%%%%%%%%%%%%%%%%%%%%%%%%%%%%%%%%%%%%%%%
\label{chap:introduction} %%%%%%%%%%%%%%%%%%%%%%%%%%%%%%%%%%%%%%%%%%%%%%%%%%%%%%%%%%%%%%%%%%%%%%%%
\setcounter{page}{17}  % Need to explicitly set page number due to hepthesis class!! CHECK THIS!!!

It is abundantly evident that machine learning and the broader field of artificial intelligence
will play an ever-increasing role in the world around us over the coming years. Virtually every
industry and human on the planet is promised to be touched by their proliferation. Whether this
change has a positive or detrimental effect on society as a whole is still questionable, however,
within the field of physics, especially experimental particle physics, machine learning
applications can only be seen in a positive light.

Already, a widespread revolution is underway across the field. From the way particle interactions
are simulated to the way recorded events are analysed, machine learning techniques, usually deep
learning algorithms, are yielding dramatic improvements. This is particularly true of the study of
the vastly abundant yet incredibly difficult to detect neutrino. Principally driven by the fact
that the raw output from neutrino detectors is well suited to the algorithms at the forefront of
computer vision research, many neutrino experiments now routinely use deep learning methods for
event analysis.

However, to date, a thorough end-to-end implementation of such techniques is not in use for the
reconstruction, classification, and energy estimation of neutrino events within long-baseline
water Cherenkov detectors studying accelerator beam neutrinos. Without redress, this lack of
progress could have significant implications for the future of neutrino physics, especially when
such experiments are deemed a highly promising (and potentially cheap) channel for answering some
of the critical unsolved problems of the field.

This thesis presents a broad range of work conducted for one such experiment, the CHerenkov
detectors In mine PitS (\chips) neutrino detector R\&D project. \chips aims to deploy very large
yet `cheap as chips' water Cherenkov detectors into deep bodies of water on the Earth's surface.
By vastly reducing costs, \chips aims to make practical and affordable megaton scale detectors a
reality, crucial for the long-term future of neutrino physics.

The author's principal contribution comes from the novel application of Convolutional Neural
Networks, a type of deep learning algorithm, to the reconstruction, classification, and energy
estimation of neutrino events within the \chipsfive prototype detector module. This work is the
solely that of the authors. 

At the same time as being explainable and robust, which are of course a key concern of
any new methodology or approach. 

It is hoped that the work presented in this thesis will achieve three principal goals. Firstly,
motivate the need for the \chips project and detail how it accomplishes its aims. Secondly, show
that Convolutional Neural Networks can be used to fully reconstruct and classify neutrino events
within water Cherenkov detectors whilst providing significant performance improvements in the
process. Finally, inform the development of similar Convolutional Neural Network applications for
other water Cherenkov neutrino detectors.

To begin, Chapter.~\ref{chap:theory} introduces the history of neutrino physics, as well as the
theoretical background and current status of the field as motivation for the broader \chips
project. A full description of the \chips detector concept follows in Chapter.~\ref{chap:chips},
with a principal focus on the \chipsfive prototype detector module deployed into the \numi
neutrino beam during the autumn of 2019. A particularly detailed overview of the \chipsfive data
acquisition and monitoring systems is then given in Chapter.~\ref{chap:daq}, for which the author
played a primary role.

Chapter.~\ref{chap:cnn} begins by introducing the standard reconstruction and classification
methods to be replaced alongside a review of the relevant neural network theory and current
machine learning applications within the field of neutrino physics. The chapter concludes by
detailing the three Convolutional Neural Networks developed for \chipsfive, including that for
cosmic muon rejection, beam event classification, and neutrino energy estimation.

The thesis closes in Chapter.~\ref{chap:results} with a comprehensive evaluation of the new
Convolutional Neural Network approach. Firstly, the final combined performance is determined and
compared with similar experiments. Secondly, the inner workings of the trained networks are
explored. Thirdly, the robustness of the network outputs to distributional changes in the input is
studied. Finally, alternative implementations are discussed to highlight the key factors driving
performance.