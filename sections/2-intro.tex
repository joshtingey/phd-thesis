\chapter{Introduction} %%%%%%%%%%%%%%%%%%%%%%%%%%%%%%%%%%%%%%%%%%%%%%%%%%%%%%%%%%%%%%%%%%%%%%%%%%%
\label{chap:introduction} %%%%%%%%%%%%%%%%%%%%%%%%%%%%%%%%%%%%%%%%%%%%%%%%%%%%%%%%%%%%%%%%%%%%%%%%
\setcounter{page}{17}  % Need to explicitly set page number due to hepthesis class!! CHECK THIS!!!

It is abundantly evident that artificial intelligence and machine learning will play an every
increasing role in the world around is over the years to come. 

It is abundantly evident that artificial intelligence and machine learning will play an ever
increasing role in the world around us over the years to come. 


It is equally capable of 


Allowing for vastly increased speed with which analysis can be conducted and less reliance on
heavily human influenced bespoke software frameworks.

As the lives of the majority of human beings 


As with virtually every industry
and the life of the majority of human beings, it is capable of revolutionising the way we conduct
experimental particle physics.


its application revolutionises virtually every industry
and the life of the majority of human beings on the planet, it also equally capable of
revolutionsiing 

It is abundantly evident that the 

As it revolutionises virtually every industry and the life of every human being on the planet it
is also capable of revolutionising the way we conduct experimental particle physics. 

It is abundantly clear that the use of artificial intelligence and machine learning 

Whether for the good or bad

- AI is impacting the future of virtually every industry and every human being on the planet. 
- Game-changer is every aspect of life, revoluntionse how idustries work.

every aspect of the world around us

As the prevalence of artificial intelligence and machine learning continues to dominate every
aspect of the world around us, it is clear that this trend will continue 

Machine learning in particular deep learning is clearly going to become essential 

Consider a two-dimensional plane onto which thousands of individual points are randomly placed.
Now suppose that the points are instead arranged so that on a macro scale they form the shape of a
ring. Not a perfect ring, but a ring with a thickness whose constituent points are smeared to make
it noticeably `fuzzy'. If you were to see a picture of this ring for the first time, how would
your process of understanding unfold?

Initially, you notice rough edges between those regions with few points and those with many. Next,
you consider how these edges interconnect and understand that they form a circular shape. This
shape is then refined once you notice the central hole and realise that, in fact, you are viewing
a ring. Knowing this you start to consider its size, its diameter, where in the plane it is
located, and whether it is elliptical in nature. Upon enough inspection you could start to
quantify the rings properties: the number of constituent points, the exact location of its centre,
even a density profile of the its internal structure.

Fundamentally, this process of understanding is not dissimilar from the way Convolutional Neural
Networks, a type of deep learning algorithm, learn to make sense of images. 

The thought experiment above relates to a Cherenkov ring, created by the cone of light emitted by
charged particles as they travel through a medium such as water.

Fundamentally, this process of understanding is the same for Convolutional Neural Networks

This is exactly the form 

multiple rings

This is fundamentally the primary event analysis task for water Cherenkov neutrino detectors.
Neutrino interactions within the detector create rings of light that are recorded on the walls of
the detector, these are used to infer the properties on the underlying neutrino interaction. The
methodology of starting from the fundamental base features of an image and slowly building up more
and more detailed and complex features of an image is the is what Convolutional Neural Networks
are a deep learning algorithm that uses a process of understanding very similar to the one above,
just like your brain. 

This thesis presents a broad range of work conducted for the CHerenkov detectors In mine PitS
(\chips) neutrino detector R\&D project. The main contribution comes from the novel application of
Convolutional Neural Networks, to the reconstruction and classification of neutrino events within
the \chipsfive prototype detector module. This is the first full-fledged implementation of
Convolutional Neural Networks to long-baseline accelerator beam water Cherenkov neutrino detectors
covering 

This thesis begins with the history of discovery in neutrino physics.

This thesis begins with the history of neutrino physics, as well as the theoretical background
underpinning the field and its current status. 


and
current status of the field.


The history of neutrino physics, as well as the theoretical background and current status of the
field are detailed as motivation for the \chips project in Chapter.~\ref{chap:theory}. A
full description of the \chips concept follows in Chapter.~\ref{chap:chips}, with a principle
focus on the \chipsfive prototype detector module deployed into the \numi neutrino beam in the
autumn of 2019. A particularly detailed overview of the data acquisition and monitoring systems
developed for the \chipsfive detector is then given in Chapter.~\ref{chap:daq}.

The standard neutrino event reconstruction and classification methods to be replaced are discussed
alongside a review of the relevant neural network theory and other deep learning applications
within the field. The Convolutional Neural Network implementations for CHIPS are then outlined,
including that for cosmic muon rejection, beam event classification and energy estimation.

A comprehensive evaluation of the trained Convolutional Neural Networks is then presented.
Firstly, the final combined performance is determined and compared with similar experiments.
Secondly, the inner workings of the trained networks are explored. Thirdly, the robustness of the
network outputs to distributional input changes is studied. Finally, alternative implementations
are discussed to highlight the key factors driving performance.

This is the first fully-fledged implementation of Convolutional Neural Networks to long-baseline
accelerator beam water Cherenkov neutrino detectors. Covering the full range of event
reconstruction and classification analysis. 

It is hoped that the work presented in this thesis will inform the development of similar
Convolutional Neural Network applications to other water Cherenkov detectors.

