\chapter{Convolutional neural networks for CHIPS} %%%%%%%%%%%%%%%%%%%%%%%%%%%%%%%%%%%%%%%%%%%%%%%%
\label{chap:cnn} %%%%%%%%%%%%%%%%%%%%%%%%%%%%%%%%%%%%%%%%%%%%%%%%%%%%%%%%%%%%%%%%%%%%%%%%%%%%%%%%%

For the majority of HEP experiments, event analysis entails the separation of signal from
background, the identification of particle types, the discovery of kinematic properties, and the
estimation of energies. The same is true for \chips detectors, with the primary aims being the
selection of CC $\nu_{e}$ signal events from a sizeable background and estimation of the
associated neutrino energy.

For this purpose, the \chips project has so far relied on a human implemented reconstruction
algorithm and a classification neural network driven by hand-engineered features. Both are prone
to human error and restricted in scope to what has been implemented in software.

The work outlined in this chapter presents a replacement event analysis methodology for \chips
concept detector modules. Three \emph{Convolutional Neural Networks} (CNNs)~\cite{fukushima1982},
a type of \emph{deep learning}~\cite{goodfellow2016} neural network have been implemented to
achieve the primary aims outlined above. One for cosmic rejection, one for beam classification,
and one for neutrino energy estimation. For evaluation purposes, only the implementation as
applied to the \chipsfive detector module is considered in this work.

After covering previous deep learning implementations for neutrino experiments, a brief
description of the current (standard) techniques is made. The theoretical background of CNNs is
then outlined before the baseline implementation for \chips is described. The specific
implementations for each of the three networks are then discussed with their combined performance
presented in the following chapter (Chapter.~\ref{chap:results}).

\section{Previous applications of deep learning for neutrino experiments} %%%%%%%%%%%%%%%%%%%%%%%%
\label{sec:cnn_previous} %%%%%%%%%%%%%%%%%%%%%%%%%%%%%%%%%%%%%%%%%%%%%%%%%%%%%%%%%%%%%%%%%%%%%%%%%

Over the last few years, neutrino experiments have started to adopt deep learning techniques for a
range of event analysis tasks. This trend has closely followed the general explosion of interest
in the field amongst the global research community, especially within computer vision, as can be
seen in Fig.~\ref{fig:papers}.

\begin{figure} % NUMBER OF PAPERS DIAGRAM DIAGRAM %
    \includegraphics[width=0.8\textwidth]{diagrams/6-cnn/papers.png}
    \caption[The number of artificial intelligence papers submitted to arXiv]
    {The number of artificial intelligence papers submitted to arXiv, broken down by sub-category.
        Note the particularly large increase in Computer Vision (CV) and pattern recognition
        papers. Figure taken from Ref.~\cite{perrault2019}.}
    \label{fig:papers}
\end{figure}

In 2016 the \nova experiment applied a CNN to the task of classifying the interaction type of
events within their sampling calorimeter detector~\cite{aurisano2016}. Two views of raw detector
events were used as input to train a network based on the popular GoogLeNet
architecture~\cite{szegedy2015} (discussed in Section.~\ref{sec:cnn_theory_conv}). Further
iterations have since been applied to both the classification of individual energy deposit
clusters~\cite{psihas2019} and $\nu_{e}$ and $e^{-}$ energy reconstruction~\cite{baldi2019}.

CNNs have also been applied to liquid argon time-projection chambers. The MicroBooNE
experiment~\cite{acciarri2017_ref} has shown that in addition to classification tasks, the
localisation of single particles within events is possible~\cite{acciarri2017}. Furthermore, the
DUNE collaboration has designed a network to output both the interaction class and counts of
different particle types within an event~\cite{collaboration2020, abi2020}. This approach take by
DUNE is called \emph{multi-task} learning and is discussed in detail within
Section.~\ref{sec:cnn_baseline_outputs}.

Applications to water Cherenkov detectors have also been made by both the Daya Bay reactor
experiment~\cite{racah2016} and the KM3NeT/ORCA collaboration~\cite{aiello2020}. Furthermore, a
type of CNN known as a \emph{variational autoencoder} has been shown to approximate the
distribution of simulated water Cherenkov data~\cite{abhishek2019}. If further studies prove
successful, this could allow for training on real data to reduce experimental uncertainties and
vastly increase the speed of simulated data generation.

\section{Standard event reconstruction and classification} %%%%%%%%%%%%%%%%%%%%%%%%%%%%%%%%%%%%%%%
\label{sec:cnn_old} %%%%%%%%%%%%%%%%%%%%%%%%%%%%%%%%%%%%%%%%%%%%%%%%%%%%%%%%%%%%%%%%%%%%%%%%%%%%%%

It is essential to outline the standard event reconstruction and classification methods that have
been used by the \chips project until now, for two main reasons. Firstly, to add context when
comparing their performance with the new CNN approach. Secondly, to highlight their main
weaknesses, motivating the new technique and making its advantages clear.

A \emph{maximum likelihood} method based on that implemented by MiniBooNE~\cite{patterson2009} is
used for event reconstruction, with a simple neural network built using the TMVA
package~\cite{hocker2007} used for event classification. Both methods are typical of the
mainstream approach used by the majority of water Cherenkov neutrino experiments. A prime example
is the fiTQun algorithm developed for the Super-Kamiokande detector, which is used for both
atmospheric~\cite{jiang2019} and T2K~\cite{missert2017} analyses.

\subsection{Likelihood based reconstruction} %%%%%%%%%%%%%%%%%%%%%%%%%%%%%%%%%%%%%%%%%%%%%%%%%%%%%
\label{sec:cnn_old_reco} %%%%%%%%%%%%%%%%%%%%%%%%%%%%%%%%%%%%%%%%%%%%%%%%%%%%%%%%%%%%%%%%%%%%%%%%%

The event reconstruction methodology is simple in theory: for a given set of hypothesised charged
particle tracks, the number of photoelectrons and the time at which the first of these is recorded
for each PMT in the detector is predicted. By comparing this prediction with the measured hit
charges and times the likelihood that the given track hypothesis produced the measured signals can
be calculated. The parameters that describe the hypothesised tracks are then varied until the
negative logarithm of the likelihood is minimised, identifying the best-fit parameters. A brief
description of the full procedure is given below, however, for a detailed description see
Ref.~\cite{blake2016} and Ref.~\cite{perch2017}. The full C++ software implementation can also be
found at Ref.~\cite{chipsreco2020}.

\subsubsection*{Seeding} %%%%%%%%%%%%%%%%%%%%%%%%%%%%%%%%%%%%%%%%%%%%%%%%%%%%%%%%%%%%%%%%%%%%%%%%%

The first stage of event reconstruction is the effective \emph{seeding} of tracks that are then
used in the full likelihood fit. The seeding methods aim to provide a good starting point for the
minimisation, both to increase the efficiency of finding the optimal track parameters and also to
avoid a false local minimum from being returned.

Firstly, the PMT hits are sliced in both space and time. Gaps in the time ordering of hits are
used to separate the event into time slices. Each of these slices then undergoes basic filtering
and clustering to remove outlying hits and ensure only the dominant collections of hits are
considered. Each cleaned slice is then run through simple vertex finding algorithms to estimate
the interaction position and time, as well and the initial track direction.

A circular \emph{Hough transform} algorithm, traditionally used for water Cherenkov ring finding
is then applied. The voting-based transformation produces an output space within which rings of
PMT hits exist as single peaks. The track direction values are then further refined using this
space, and a search for smaller peaks is carried out to indicate if multiple particles are likely
to be involved. This process results in a list of seeds with a corresponding score related
directly to the height of the associated peak in Hough transform space.

\subsubsection*{Likelihood fit} %%%%%%%%%%%%%%%%%%%%%%%%%%%%%%%%%%%%%%%%%%%%%%%%%%%%%%%%%%%%%%%%%%

Each track in a fit hypothesis comprises a vector of parameters $\vec{x}$, containing the
following:
\begin{itemize}
    \item The track vertex position ($x_{0}$, $y_{0}$, $z_{0}$) and interaction time $t_{0}$.
    \item The initial track direction ($d_{\theta}$, $d_{\phi}$).
    \item The initial kinetic energy of the particle.
    \item The particle type (muon, electron or photon).
\end{itemize}
For a photon hypothesis (identical to an electron hypothesis in reality) the distance between the
interaction vertex and the beginning of the electromagnetic shower is also included as a
parameter.

The hypothesis tracks are then initialised using the list of seeds found in the seeding procedure
in descending order of Hough peak height score. As the seeding algorithms do not estimate the
particle energy, a default value equal to the average particle energy observed in the Monte Carlo
simulation is assigned. Additionally, constraints can be placed on a multi-track hypothesis before
the minimisation process to reduce the number of free parameters.

As an example, considering the NC $\pi^{0}$ case, a multi-track two-photon hypothesis is used.
Firstly, the initial parameters for the two photons are assigned from the two highest-scoring
seeds from the seeding procedure. Secondly, the vertex position for both tracks is constrained to
remain the same, and the directions and energies are set to be constrained by the invariant mass
of the $\pi^{0}$.

In it's simplest form the likelihood is a simple product of two terms:
\begin{equation} % LIKELIHOOD EQUATION %
    \mathcal{L}(\vec{x})=\mathcal{L}_{unhit}(\vec{x})\mathcal{L}_{hit}(\vec{x})=
    \prod_{unhit}P_{unhit}(\vec{x})\prod_{hit}P_{charge}(\vec{x})P_{time}(\vec{x}),
    \label{eq:likelihood}
\end{equation}
where the first ($unhit$) term gives the likelihood that the hypothesis $\vec{x}$ will not predict
a hit on the PMTs that do not have a measured hit, and the second ($hit$) term gives the
likelihood that $\vec{x}$ produces the observed photoelectrons and hit times on the hit PMTs. By
considering the negative logarithm of the likelihood the computation can be simplified into a sum
of logarithms over the PMTs, such that
\begin{equation} % LIKELIHOOD SUM PMTS EQUATION %
    -\log\mathcal{L}(\vec{x})=
    -\sum_{unhit}\log(P_{unhit}(\vec{x}))
    -\sum_{hit}\log(P_{charge}(\vec{x}))
    -\sum_{hit}\log(P_{time}(\vec{x})).
    \label{eq:likelihood_sum}
\end{equation}
This form also has the effect of separating the charge (number of photoelectrons) and time
prediction components, which can then be dealt with separately computationally. In the actual
likelihood calculation the $P_{unhit}(\vec{x})$ and $P_{charge}(\vec{x})$ components are combined,
where the probability of an unhit PMT is treated as a PMT with an observed charge equal to zero.

The Minuit2 algorithm contained within the ROOT software package~\cite{brun1997} is used for the
minimisation process. At each iteration, the charge and hit time predictions are made, and the
negative logarithm of the likelihood is calculated. The track parameters are then varied to
minimise the likelihood before the next iteration. Through a series of stages, each fixing and
freeing specific parameters, the minimisation process converges to the best-fit parameters for the
given hypothesis. This procedure typically takes two minutes on a standard batch farm computing
node.

\subsubsection*{Downsides} %%%%%%%%%%%%%%%%%%%%%%%%%%%%%%%%%%%%%%%%%%%%%%%%%%%%%%%%%%%%%%%%%%%%%%%

The charge and hit time predictions and their associated likelihood contributions depend on many
low-level inputs to the reconstruction. These inputs generally describe how Cherenkov light is
emitted from specific particles and then how it propagates through the detector to be detected by
the PMTs. Examples of these inputs include:
\begin{itemize}
    \item The number of Cherenkov photons emitted by a particle of a specific type and energy.
    \item The fraction of Cherenkov light emitted at each step along a specific particle's track
          length.
    \item The angular distribution of Cherenkov photon emission for each type of particle.
    \item The survival probability of photons within the detector medium as a function of
          distance.
    \item A detailed description of the PMT positions and directions within the detector.
    \item The angular efficiency of each PMT relative to the incident photon angle.
    \item The probability of a measured charge given the predicted number of photoelectrons
          (derived from a reversal of the simulation digitisation methodology).
\end{itemize}
The first three are combined into \emph{emission profiles} generated from a large Monte Carlo
simulation sample, while the last is comparable to that shown in Fig.~\ref{fig:digitisation}.

The above list demonstrates a fundamental problem with the likelihood-based approach. It is
heavily reliant on the accuracy of low-level inputs and their associated use in human implemented
software. If a physical process is not dealt with appropriately or overlooked (such as the
hadronic component of neutrino events), then the prediction accuracy of PMT charges and hit times
is affected, impacting the performance of finding accurate best-fit parameters.

Moreover, the likelihood-based approach requires a predefined track hypothesis. Real events
expected within \chips detector modules are rarely simple single particle or even two-particle
events. In reality, the majority of events contain multiple final state particles of various types
and in various topologies. The predefinition of a track hypothesis makes the implementation of a
generalised approach to event reconstruction where all possible event types are considered
incredibly challenging.

\subsection{Event classification}%%%%%%%%%%%%%%%%%%%%%%%%%%%%%%%%%%%%%%%%%%%%%%%%%%%%%%%%%%%%%%%%%
\label{sec:cnn_old_pid} %%%%%%%%%%%%%%%%%%%%%%%%%%%%%%%%%%%%%%%%%%%%%%%%%%%%%%%%%%%%%%%%%%%%%%%%%%

As the standard event reconstruction is based on the calculation of a likelihood (analogous to a
\emph{goodness-of-fit}), the likelihood ratio between different hypotheses can be used for event
classification tasks. It is also found that additional hand-engineered features derived from the
reconstruction outputs have power in classifying the event type.

Two simple neural networks are used, the first for CC $\nu_{e}$ - CC $\nu_{\mu}$ separation and
the second for CC $\nu_{e}$ - NC separation. Both contain a single hidden layer with the number of
nodes equal to the number of input parameters plus five. Variables from both a single electron
track and single muon track hypothesis fit to each event are used for both networks, with a full
list of inputs as follows:
\begin{itemize}
    \item The $\Delta\log\mathcal{L}$ between $e$ and $\mu$ hypothesis for both time and charge
          components.
    \item The total number of hit PMTs ($N_{hits}$) and total collected charge.
    \item $\frac{\Delta\log\mathcal{L}_{charge}}{N_{hits}}$.
    \item The fraction of hits inside, within, and outside the ring for both the $e$ and $\mu$
          hypotheses.
    \item The fraction of predicted charge outside the ring for both the $e$ and $\mu$ hypotheses.
    \item The ratio of the total predicted charged to the total measured charge for both the $e$
          and $\mu$ hypothesis.
    \item The ratio of the reconstructed energy to the total measured charge.
    \item The reconstructed track direction under the $e$ hypothesis.
    \item The fraction of hits in the downstream half of the detector.
    \item The number of seeds generated by the Hough transform seeding algorithm.
    \item The peak height score of the first and last seeds found by the Hough transform seeding
          algorithm.
\end{itemize}

A sample of CC $\nu_{e}$ and CC $\nu_{\mu}$ beam events characteristic of those expected to be
seen within \chipsfive are used to train the first classifier, and a corresponding sample of CC
$\nu_{e}$ and NC events for the second. Both network output values can then be used to select CC
$\nu_{e}$ events from the background. Note that only the selection of CC $\nu_{e}$ events has been
implemented, no CC $\nu_{\mu}$ selection has been developed.

The main limitation of this approach is that the input features are restricted to those that have
been imagined (requiring extensive domain knowledge) and then implemented in software. The current
list is undoubtedly non-exhaustive of all the possible variables and combinations of variables
that can, in theory, be used for discrimination between events. Additionally, any mistakes in the
likelihood-based reconstruction and, therefore, input variables to the neural networks, can lead
to incorrect categorisation of events.

\section{The theory of neural networks} %%%%%%%%%%%%%%%%%%%%%%%%%%%%%%%%%%%%%%%%%%%%%%%%%%%%%%%%%%
\label{sec:cnn_theory} %%%%%%%%%%%%%%%%%%%%%%%%%%%%%%%%%%%%%%%%%%%%%%%%%%%%%%%%%%%%%%%%%%%%%%%%%%%

There are many machine learning techniques: linear regression, logistic regression, k-nearest
neighbours, decision trees, random forests, support vector machines, and others, all of which
learn to make predictions about data. However, none have been as successful, especially in recent
years, as the deep neural network. As both the size of datasets and the amount of available
computing power has increased, deep neural networks have proved incredibly powerful for many
tasks, as they are well suited to this paradigm.

Here we discuss the application of neural networks for \emph{supervised learning}, one of two
broad machine learning categories and concerned with using labelled example data to train
algorithms. The other broad category of \emph{unsupervised learning}, where the properties of the
dataset are inferred without example data is not discussed, however, will be used for network
\emph{explainability} in Section.~\ref{sec:results_explain}.

\subsection{Neural network basics} %%%%%%%%%%%%%%%%%%%%%%%%%%%%%%%%%%%%%%%%%%%%%%%%%%%%%%%%%%%%%%%
\label{sec:cnn_theory_basics} %%%%%%%%%%%%%%%%%%%%%%%%%%%%%%%%%%%%%%%%%%%%%%%%%%%%%%%%%%%%%%%%%%%%

A neural network is a type of algorithm inspired by the repeating cell structure of neurons within
our brains. The basic building block of a neural network is a \emph{neuron}, which takes a vector
of $k$ inputs $\vec{x} = (x_{1}, x_{2},\dots,x_{k})$ and outputs a scalar $a(\vec{x})$. The
neurons are arranged into layers, with the input of one layer being the output from the previous
layer. The first layer is commonly referred to as the \emph{input layer}, the middle layers as
\emph{hidden layers}, and the final layer as the \emph{output layer}, as illustrated in
Fig.~\ref{fig:network}. In general, this simple neural network structure is referred to as
\emph{fully-connected}, as all the neurons in each layer, have connections to all the neurons in
the previous and following layers.

\begin{figure} % BASIC NETWORK DIAGRAM %
    \includegraphics[width=0.6\textwidth]{diagrams/6-cnn/network.pdf}
    \caption[Illustration of a simple neural network]
    {Illustration of a simple neural network. There is a single input layer (yellow), two hidden
        layers (blue), and an output layer (green). Each node corresponds to a \emph{neuron}
        except for the input layer.}
    \label{fig:network}
\end{figure}

Input variables (traditionally hand-engineered features extracted from raw data) are passed into
the network via the input layer. Any number of hidden layers containing any number of neurons can
then follow. The neurons contained within these layers are trained so that collectively their
$a(\vec{x})$ functions solve the task at hand. For a \emph{regression} task, the output layer
returns a continuous decimal value. Otherwise, for a \emph{classification} task, a probability
value between zero and one is output for each class. The forward passing of information from one
layer to the next is why neural networks can also be referred to as \emph{feed-forward graphs}.

For a neuron $i$, $a_{i}(\vec{x})$ can be decomposed into a linear operation, specific to the
neuron, followed by a non-linear operation, which is the same across all neurons. The linear
operation consists of the dot product of the input vector $\vec{x}$ with a vector of weights
$\vec{w}^{(i)} = (w_{1}^{(i)}, w_{2}^{(i)},\dots,w_{k}^{(i)})$, plus a bias term $b^{(i)}$:
\begin{equation} % NETWORK BASIC EQUATION %
    z^{(i)}=\vec{w}^{(i)}\cdot\vec{x}+b^{(i)}.
    \label{eq:network}
\end{equation}
After applying the non-linear operation $\sigma_{i}$, commonly referred to as the \emph{activation
    function} the final output from the neuron can be written as
\begin{equation} % NETWORK ACTIVATION EQUATION %
    a_{i}(\vec{x})=\sigma_{i}(z^{(i)}).
    \label{eq:activation}
\end{equation}

Traditionally, a step-function (for networks called \emph{perceptrons}) was used for the
activation function. However, as is shown in Section.~\ref{sec:cnn_theory_training}, a non-zero
gradient (only valid at $x=0$ for a step-function) is required for the practical training of
neural networks. In fact, the choice of activation function can greatly affect how the network
trains and performs.

Therefore, common activation function choices have been the \emph{hyperbolic tangent} and the
\emph{sigmoid} function, primarily because they are bounded and differentiable at all points.
Recently, the \emph{ReLU} and other similar functions have become popular,  mainly due to them
avoiding the problem of \emph{vanishing} gradients caused by the saturation of the tanh and
sigmoid functions at large values of $x$. All of these functions are shown in
Fig.~\ref{fig:activations} for reference.

\begin{figure} % ACTIVATIONS DIAGRAM %
    \includegraphics[width=\textwidth]{diagrams/6-cnn/activations.pdf}
    \caption[Common non-linear activation functions]
    {Common non-linear activation functions used for the neurons within neural networks.}
    \label{fig:activations}
\end{figure}

\subsection{Training neural networks} %%%%%%%%%%%%%%%%%%%%%%%%%%%%%%%%%%%%%%%%%%%%%%%%%%%%%%%%%%%%
\label{sec:cnn_theory_training} %%%%%%%%%%%%%%%%%%%%%%%%%%%%%%%%%%%%%%%%%%%%%%%%%%%%%%%%%%%%%%%%%%

The process of supervised training of a neural network uses labelled data to iteratively find the
optimal weights and biases (network parameters) that maximise the network performance. In order to
quantify the performance, we define a \emph{loss function} $E(\vec{w})$, describing the difference
between the network output and the true label, where $\vec{w}$ is the vector of network
parameters. For a given input data point $(\vec{x}_{i}, y_{i})$, with $\vec{x}_{i}$ being the
input parameters and $y_{i}$ the known truth label, the network generates an output
$\hat{y}_{i}(\vec{w})$. Using this notation, we can construct loss functions suitable for
different tasks.

In the case of a simple binary classification task the most commonly used function is the
\emph{binary cross-entropy}:
\begin{equation} % BINARY CROSS-ENTROPY EQUATION %
    E(\vec{w})=
    -\displaystyle\sum_{i=1}^{n}y_{i}\log\hat{y}_{i}(\vec{w})+
    (1-y_{i})\log[1-\hat{y}_{i}(\vec{w})],
    \label{eq:binary_cross_entropy}
\end{equation}
where the number of data points is given by $n$. For a classification task where the number of
classes is greater than two $y$ can take on $M$ values. In this case we redefine each data point
so that $y$ is instead a vector $y_{im}$ such that
\begin{equation} % ONE-HOT EQUATION %
    y_{im}=
    \begin{cases}
        1 & \text{if $y_{i}=m$} \\
        0 & \text{otherwise.}   \\
    \end{cases}
\end{equation}
This is commonly named a \emph{one-hot} vector. The cross-entropy then becomes the
\emph{categorical cross-entropy}:
\begin{equation} % CATEGORICAL CROSS-ENTROPY EQUATION %
    E(\vec{w})=
    -\displaystyle\sum_{i=1}^{n}\displaystyle\sum_{m=0}^{M-1}y_{im}\log\hat{y}_{im}
    (\vec{w})+(1-y_{im})\log[1-\hat{y}_{im}(\vec{w})].
    \label{eq:categorical_cross_entropy}
\end{equation}
For a regression task predicting a continuous output variable, the \emph{mean-squared error} is
most often used as the loss function:
\begin{equation} % MEAN-SQUARED ERROR LOSS EQUATION %
    E(\vec{w})=
    \frac{1}{n}\displaystyle\sum_{i=1}^{n}(y_{i}-
    \hat{y}_{i}(\vec{w}))^{2}.
    \label{eq:mse}
\end{equation}

To find the optimal network parameters for the given task we can iteratively minimise the loss
function until it converges to a minimum (or in reality a local minimum that performs well). This
is done by updating the network parameters at each iteration $t$ to move in the direction of the
los function gradient, using the update rule
\begin{equation} % UPDATE EQUATION %
    \vec{w}_{t+1}=\vec{w}_{t}-\eta_{t}\nabla_{\vec{w}}E(\vec{w}),
    \label{eq:update_rule}
\end{equation}
where $\eta_{t}$ is the \emph{learning rate}, determining the size of the step taken at each
iteration. This methodology is known as \emph{gradient descent} and is illustrated in
Fig.~\ref{fig:gradient_descent}.

\begin{figure} % GRADIENT DESCENT DIAGRAM %
    \includegraphics[width=0.5\textwidth]{diagrams/6-cnn/gradient_descent.pdf}
    \caption[Illustration of the gradient descent process]
    {Simplified illustration of the gradient descent procedure. Shown is the case for a loss
        function dependent on a single weight.}
    \label{fig:gradient_descent}
\end{figure}

Therefore, in order to use gradient descent, we require that the gradient of the loss function
with respect to the parameters of the network can be calculated. Doing this for each parameter at
every iteration would render neural networks impossible to train due to the vast computational
requirements. Instead, an innovative application of the chain rule, in an algorithm called
\emph{backpropagation} is used~\cite{werbos1974}. Here we follow the derivation of the four main
equations of backpropagation given in Ref.~\cite{mehta2019}.

For a network containing $L$ layers, we can index the individual layers using $l=1,\dots,L$. The
weight associated with the connection between the $k$-th neuron in layer $l-1$ and the $j$-th
neuron in layer $l$ can be denoted as $w^{l}_{jk}$. The bias of the layer $l$ neuron is written as
$b^{l}_{j}$. The activation of the $j$-th neuron in layer $l$ is then related to the outputs from
the previous layer by
\begin{equation} % PREVIOUS LAYER EQUATION %
    a^{l}_{j}=\sigma(z^{l}_{j})=\sigma\left(\sum_{k}w^{l}_{jk}a^{l-1}_{k}+b^{l}_{j}\right).
    \label{eq:feedforward}
\end{equation}

The change in the loss function with respect to the linear weighted sum $z^{l}_{j}$ of the $j$-th
neuron in the last layer $L$ can be used to define the error
\begin{equation} % PREVIOUS LAYER EQUATION %
    \Delta^{L}_{j}=\frac{\partial E}{\partial z^{L}_{j}}.
\end{equation}
Similarly the error on any neuron $j$ in any layer $l$ is given by
\begin{equation} % BACKPROP EQUATION 1 %
    \Delta^{l}_{j}=\frac{\partial E}{\partial z^{l}_{j}}=\frac{\partial E}{\partial a^{l}_{j}}
    \sigma '(z^{l}_{j}),
    \label{eq:backprop_1}
\end{equation}
with $\sigma '$ denoting the derivative of the non-linear activation function. As $\partial
    b^{l}_{j}/\partial z^{l}_{j}=1$, the error can also be viewed as the partial derivative of the
loss function with respect to the bias, such that
\begin{equation} % BACKPROP EQUATION 2 %
    \Delta^{l}_{j}=\frac{\partial E}{\partial z^{l}_{j}}
    =\frac{\partial E}{\partial b^{l}_{j}}\frac{\partial b^{l}_{j}}{\partial z^{l}_{j}}
    =\frac{\partial E}{\partial b^{l}_{j}}.
    \label{eq:backprop_2}
\end{equation}

Using the chain rule and the fact that the error on neurons in layer $l$ only depends on the
activation of neurons in the following layer $l+1$, we can write
\begin{align} % BACKPROP EQUATION 3 %
    \begin{split}
        \Delta^{l}_{j} &=\frac{\partial E}{\partial z^{l}_{j}}
        =\sum_{k}\frac{\partial E}{\partial z^{l+1}_{k}}
        \frac{\partial z^{l+1}_{k}}{\partial z^{l}_{j}} \\
        &=\sum_{k}\Delta^{l+1}_{k}\frac{\partial z^{l+1}_{k}}{\partial z^{l}_{j}} \\
        &=\left(\sum_{k}\Delta^{l+1}_{k}w^{l+1}_{kj}\right)\sigma '(z^{l}_{j}).
    \end{split}
    \label{eq:backprop_3}
\end{align}
Additionally, the differential of the cost function with respect to the weight $w^{l}_{jk}$ can be
written as
\begin{equation} % BACKPROP EQUATION 4 %
    \frac{\partial E}{\partial w^{l}_{jk}}
    =\frac{\partial E}{\partial z^{l}_{j}}\frac{\partial z^{l}_{j}}{\partial w^{l}_{jk}}
    =\Delta^{l}_{j}a^{l-1}_{k}.
    \label{eq:backprop_4}
\end{equation}

The full backpropagation algorithm then proceeds as follows:
\begin{enumerate}
    \item After calculating the activations $a^{1}_{j}$ for all neurons in the input layer, use
          the feed-forward architecture of the network to calculate all activations at every layer
          using Eq.~\ref{eq:feedforward}.
    \item Use Eq.~\ref{eq:backprop_1} to calculate the error of the top layer neurons. Both the
          derivative of the loss function and the activation function is required for this step.
    \item Use Eq.~\ref{eq:backprop_2} to `backpropagate' the error through the network from the
          top layer to the input layer to find all $\Delta^{l}_{j}$ values.
    \item Calculate the gradients for all the weights and biases using Eq.~\ref{eq:backprop_3} and
          Eq.~\ref{eq:backprop_4}.
\end{enumerate}
As can be seen, a single activation finding \emph{forward pass} followed by a single error
propagating \emph{backward pass}  is all that's required to calculate the gradients for all
weights and biases within the network. This incredibly efficient procedure allows gradient descent
to be used for the training of neural networks.

\subsection{Convolutional neural networks} %%%%%%%%%%%%%%%%%%%%%%%%%%%%%%%%%%%%%%%%%%%%%%%%%%%%%%%
\label{sec:cnn_theory_conv} %%%%%%%%%%%%%%%%%%%%%%%%%%%%%%%%%%%%%%%%%%%%%%%%%%%%%%%%%%%%%%%%%%%%%%

The broad category of deep learning covers multiple neural network techniques spanning a range of
application fields such as computer vision, speech recognition and natural language processing. By
stacking many layers on top of each other into a `deep' network, these methods offer increased
problem-solving capacity by allowing higher-order non-linear functions to form. As a direct
consequence, instead of requiring hand-engineered features as input, deep networks can learn to
extract the most powerful features for a given task from raw data. Here we outline just the
specific technique used in this work; the CNN.

At their core, a CNN makes use of a mathematical operation called \emph{convolution}, which either
entirely or in part replaces the simple vector multiplication seen in the fully-connected networks
introduced in Section.~\ref{sec:cnn_theory_basics}. This difference makes CNNs incredibly powerful
for applications using grid-like input data such as computer vision tasks.

Using standard CNN terminology, the discrete convolution between the \emph{input}, $x$ and the
\emph{kernel}, $w$ is given by
\begin{equation}
    f_{i}=(x*w)_{i}=\sum^{\infty}_{j=-\infty}x_{j}w_{i-j},
    \label{eq:convolution}
\end{equation}
where $f$ is commonly referred to as the \emph{feature map}. In typical applications, the input is
a two-dimensional array $X$. Therefore, both the kernel $W$, and the resulting feature map $F$,
also become two-dimensional. In this case the convolution operation becomes
\begin{equation}
    F_{i}=(X*W)_{i,j}=\sum_{m}\sum_{n}X_{i+m,j+n}W_{m,n},
    \label{eq:conv}
\end{equation}
where the infinite sum in Eq.~\ref{eq:convolution} has been replaced with a discrete sum over
two-dimensional elements. Analogous to the simple neural network weights $\vec{w}$ first described
in Eq.~\ref{eq:network}, the elements of $W_{m,n}$ can be trained to maximise the network
performance (minimise the loss). Also like the simple case, the output feature maps are commonly
passed through a non-linear activation function to produce the final output.

To illustrate this operation Fig.~\ref{fig:conv_input} gives examples of a $4 \times 4$ input grid
and a $3 \times 3$ kernel. The output feature map is generated by sliding the kernel across both
dimensions of the input grid, summing the products of all associated elements at each step
according to Eq.~\ref{eq:conv}, as is shown in Fig.~\ref{fig:conv_operation}.

\begin{figure} % CONV INPUTS DIAGRAM %
    \includegraphics[width=0.7\textwidth]{diagrams/6-cnn/conv_input.pdf}
    \caption[Example of a Convolutional Neural Network input grid and kernel]
    {Example of an input grid (left) and kernel (right). The specific kernel shown is sensitive to
        x-shaped features}
    \label{fig:conv_input}
\end{figure}

\begin{figure} % CONV OPERATION DIAGRAM %
    \includegraphics[width=0.7\textwidth]{diagrams/6-cnn/conv_operation.pdf}
    \caption[Example of a convolutional operation]
    {Example of a $\mathrm{stride}=1$ convolution operation involving the input grid and kernel
        from Fig.~\ref{fig:conv_input}. Both the operation for the case of \emph{valid} (top) and
        \emph{same} (bottom) padding is shown. The blue square in the top-left of the output
        feature maps indicates the output generated from the specific operation shown on the left
        also in blue.}
    \label{fig:conv_operation}
\end{figure}

Two additional parameters that impact the output size of the feature map are also introduced in
Fig.~\ref{fig:conv_operation}. The \emph{stride} and the \emph{padding}. The stride governs how
far the kernel moves at each step while the padding decides how the input grid is padded with
zeros around its border. If $L$ is the size of the input (both height and width) and $K$ is the
kernel size, the output feature map size $O$, is given by
\begin{equation}
    O=\frac{(L-K+2P)}{S}+1,
    \label{eq:conv_size}
\end{equation}
where $S$ is the stride, and $P$ is the amount of zero padding.

The other key operation used in CNNs is pooling. Pooling layers coarse-grain the spatial
information of the input via down-sampling to reduce the number of network parameters. \emph{Max
    pooling} or \emph{average pooling} are the two common ways this is achieved. In both cases, the
input is first divided into rectangular regions, and then either the maximum or average value of
the region is used as the output, for max or average pooling respectively, as illustrated in
Fig.~\ref{fig:pooling}.

\begin{figure} % POOLING DIAGRAM %
    \includegraphics[width=0.7\textwidth]{diagrams/6-cnn/pooling.pdf}
    \caption[Example of pooling operation]
    {Example of both a max and average $2 \times 2$ pooling operation with $\mathrm{stride}=2$.}
    \label{fig:pooling}
\end{figure}

Taking inspiration from how neurons behave in the visual cortex of animals~\cite{lecun2015}, small
kernels are generally used that only scan over a small spatial patch of the input at any one time.
When combined with the loss of absolute position information from pooling, a key feature of CNNs
is highlighted. They exhibit translational invariance and respect the local structure contained
within the input. In simpler terms, they do not care wherein the input image a particular feature
exists, just that it exists.

\subsubsection*{CNN architectures}

In 2012 the AlexNet CNN lowered the error rate on the ubiquitous ImageNet classification
task~\cite{deng2009} from 28\% to 16\%~\cite{krizhevsky2012}. Since this breakthrough, the
standard CNN has adopted a similar architecture. Multiple convolutional layers are stacked on top
of each other, periodically interspersed with pooling layers. Once the output feature map size no
longer allows for additional pooling, one or more fully-connected layers are appended before the
final output layer, as shown in Fig.~\ref{fig:conv_diagram}.

\begin{figure} % CONV DIAGRAM %
    \includegraphics[width=\textwidth]{diagrams/6-cnn/conv_diagram.jpeg}
    \caption[Typical Convolutional Neural Network architecture]
    {Illustration of a typical CNN architecture, containing convolutional, pooling and
        fully-connected layers before the final output layer.}
    \label{fig:conv_diagram}
\end{figure}

Led primarily by research teams at the technology giants, improvements upon this standard
architecture have been made. Initially, this involved the addition of extra convolutional layers
to form a deeper network as VGG did in 2014~\cite{simonyan2014}. Separately, the \emph{inception
    module} introduced by GoogLeNet~\cite{szegedy2015} the same year allowed for different scales of
features to be explored using the concept of a \emph{network-within-a-network}.

As the size of CNNs increased, it was found that the gradients on the lower layers of the network
were also found to decrease, sometimes even vanishing, preventing further learning. To
counter this problem ResNet~\cite{he2016_original, he2016_improved} introduced residual
connections, skipping specific layers and allowing for a larger gradient to reach the affected
lower layers during backpropagation. Recently, the inception module and ResNet ideas have been
combined~\cite{szegedy2016}, and there has been a significant push for efficient rather than just
deeper networks~\cite{sandler2018, tan2019}. The repeating layout of layers that form the above
networks, often referred to as \emph{blocks}, are shown in Fig.~\ref{fig:blocks}.

\begin{figure} % BLOCKS DIAGRAM %
    \includegraphics[width=\textwidth]{diagrams/6-cnn/blocks.pdf}
    \caption[Common Convolutional Neural Network architecture blocks]
    {Common blocks used within CNN architectures, taking $x$ as input and producing $\tilde{x}$ as
        output through operations whose forward pass is described with the arrows. The blue and
        red boxes represent convolutional (conv) and max-pooling (m-pool) layers respectively,
        with the size of the operation shown. The circular yellow $R$ indicates the use of the
        ReLU activation function.}
    \label{fig:blocks}
\end{figure}

\subsection{Regularisation} %%%%%%%%%%%%%%%%%%%%%%%%%%%%%%%%%%%%%%%%%%%%%%%%%%%%%%%%%%%%%%%%%%%%%%
\label{sec:cnn_theory_reg} %%%%%%%%%%%%%%%%%%%%%%%%%%%%%%%%%%%%%%%%%%%%%%%%%%%%%%%%%%%%%%%%%%%%%%%

A key challenge for supervised machine learning techniques is ensuring the algorithm can
generalise to new, previously unseen data, not within the training dataset. This generalisation
can be difficult when for a deep neural network containing millions of trainable parameters, it
can become easy for the network to learn the specific features and noise of the training data.
This unwanted learning is named \emph{overfitting} and is very common when training CNNs. Methods
used within this work to prevent overfitting are outlined below, all of which are commonly
referred to as \emph{regularisation} techniques.

\subsubsection*{Stochastic gradient descent} %%%%%%%%%%%%%%%%%%%%%%%%%%%%%%%%%%%%%%%%%%%%%%%%%%%%%

The procedure outlined so far of updating the network weights at each training iteration using the
gradient calculated over the full dataset is called \emph{batch training}. It is instead much more
common to calculate an approximation to the complete gradient at each iteration using a
\emph{minibatch} of the full dataset. This is done by considering just a subset of the training
data with a size commonly referred to as the \emph{batch size} and typically equal to a power of
two for computational reasons.

This modification to standard batch training gradient descent is called \emph{stochastic gradient
    descent} as it introduces stochasticity to the training process, and provides two main advantages.
Firstly the computational speed of each iteration is significantly reduced, and crucially the
memory requirements lowered. Secondly, the addition of noise decreases the chance that the
minimisation will get stuck in a local minimum suited to overfitting the training dataset.

\subsubsection*{Early stopping} %%%%%%%%%%%%%%%%%%%%%%%%%%%%%%%%%%%%%%%%%%%%%%%%%%%%%%%%%%%%%%%%%%

\emph{Early stopping} is a simple procedure which prevents overfitting from affecting the final
generalised performance of the network. During training, the training dataset is commonly iterated
over multiple times, with each iteration called an \emph{epoch}. By evaluating the error on an
independent \emph{validation} dataset at the end of each epoch, the point at which overfitting
begins can be determined, as illustrated in Fig.~\ref{fig:early_stopping}. At the determined
epoch, the training is stopped to return the best possible generalised model. In practice, it is
common only to stop training after $n$ number of epochs have passed with no validation error
improvement.

\begin{figure} % EARLY STOPPING DIAGRAM %
    \includegraphics[width=0.5\textwidth]{diagrams/6-cnn/early_stopping.pdf}
    \caption[Illustration of the early stopping procedure]
    {Illustration of the early stopping procedure. Initially both the training and test error
        decreases, but, at some point the test error will start to increase due to overfitting, at
        this point the training is stopped.}
    \label{fig:early_stopping}
\end{figure}

\subsubsection*{Batch normalisation} %%%%%%%%%%%%%%%%%%%%%%%%%%%%%%%%%%%%%%%%%%%%%%%%%%%%%%%%%%%%%

The training of a neural network is found to work best when the inputs of each neuron are centred
on zero with respect to the bias of the neuron. This is because large input values can cause
saturation of the activation function and subsequent vanishing of the associated gradient,
reducing the ability of the network to learn. To counter this, \emph{batch normalisation}
introduces layers that standardise the inputs using the mean and variance of each
minibatch~\cite{ioffe2015}. This change not only speeds up training by preventing the vanishing of
gradients but also reduces overfitting by using the stochasticity of the minibatch.

\subsubsection*{Dropout} %%%%%%%%%%%%%%%%%%%%%%%%%%%%%%%%%%%%%%%%%%%%%%%%%%%%%%%%%%%%%%%%%%%%%%%%%

\emph{Dropout} is another simple technique to reduce overfitting~\cite{hinton2012}. At each
training iteration, each neuron has a probability $p_{d}$ to be \emph{dropped out} and ignored for
that iterations calculation. This is illustrated in Fig.~\ref{fig:dropout}. By ignoring a subset
of the neurons at each iteration, it is difficult for the network to form the particularly strong
connections between neurons that are usually responsible for overfitting, leading to greater
generalisation.

\begin{figure} % DROPOUT DIAGRAM %
    \includegraphics[width=0.6\textwidth]{diagrams/6-cnn/dropout.pdf}
    \caption[Illustration of dropout]
    {Example of dropout applied to the network shown in Fig.~\ref{fig:network}. Neurons are
        randomly \emph{dropped out} and not considered at each training step. This reduces the
        strong correlations between neurons that can lead to overfitting.}
    \label{fig:dropout}
\end{figure}

\section{A baseline implementation} %%%%%%%%%%%%%%%%%%%%%%%%%%%%%%%%%%%%%%%%%%%%%%%%%%%%%%%%%%%%%%
\label{sec:cnn_baseline} %%%%%%%%%%%%%%%%%%%%%%%%%%%%%%%%%%%%%%%%%%%%%%%%%%%%%%%%%%%%%%%%%%%%%%%%%

The raw output from a water Cherenkov detector, such as those envisioned by the \chips concept, is
a simple image of each event where two channels of information are known for each PMT: the number
of collected photoelectrons, and the associated hit times. Therefore, it is a natural fit to use
CNNs primarily developed for image-based computer vision tasks for \chips detector event analysis.

For this purpose, a Python-based software package named \emph{chipsnet}~\cite{chipsnet2020} has
been built. By using the high-level \emph{Application Programming Interfaces}, (APIs) provided by
the Tensorflow framework (initially developed by Google)~\cite{tf2015}, a full pipeline including
data preparation, network training, and performance evaluation is implemented. In this section,
the baseline CNN implementation built into chipsnet is outlined. The specific network
implementations described afterwards in Section.~\ref{sec:cnn_specific} for cosmic rejection, beam
classification, and energy estimation, share this common baseline with a few specific differences.

\subsection{Baseline inputs} %%%%%%%%%%%%%%%%%%%%%%%%%%%%%%%%%%%%%%%%%%%%%%%%%%%%%%%%%%%%%%%%%%%%%
\label{sec:cnn_baseline_inputs} %%%%%%%%%%%%%%%%%%%%%%%%%%%%%%%%%%%%%%%%%%%%%%%%%%%%%%%%%%%%%%%%%%

The primary difficulty in the application of CNNs to \chips detectors is determining how to map an
event captured on a cylindrical surface to a rectangular two-dimensional grid. Furthermore, this
must be done in such a way as not to distort the underlying event topology which could inhibit
network learning. As a solution, this work takes inspiration and then builds upon the ideas
outlined in Ref.~\cite{theodore2016}. Simply put, an event is mapped onto a two-dimensional grid
as though it is viewed from its estimated interaction vertex position. The primary motivation
behind this is to remove any detector shape effects and to focus on the underlying event topology
and Cherenkov emission profiles.

To estimate the interaction vertex position for each event, the top-scoring seed from the seeding
procedure introduced in Section.~\ref{sec:cnn_old_reco} is used. This process, unlike the full
likelihood fit, requires no track hypothesis and typically takes under
\unit{0.1}{\mathrm{seconds}} per event on a standard batch farm computing node. The seed estimated
vertex position resolutions for a sample of expected beam events are shown in
Fig.~\ref{fig:explore_true_reco_vtx}. The $x$ component is commonly estimated closer to the
downstream wall of the detector than reality, however, as the $y$ and $z$ components perpendicular
to the beam primarily drive event distortions, the impact on event topology is small.

\begin{figure} % HOUGH VTX RES DIAGRAM %
    \includegraphics[width=\textwidth]{diagrams/7-results/explore_true_reco_vtx.pdf}
    \caption[Seed interaction vertex resolutions]
    {Seed interaction vertex position resolution, split by component. The large negative tail for
        the $x$ component shows the tendency of the seeding procedure to estimate the $x$
        component closer to the downstream detector wall than reality.}
    \label{fig:explore_true_reco_vtx}
\end{figure}

Using $\theta$ and $\phi$ components calculated as viewed from the estimated interaction vertex
position facing along the x-axis (downstream), hit PMTs are mapped onto a $64 \times 64$ grid.
This procedure is used to generate two event \emph{maps}. Firstly, a \emph{hit-charge} map where
each grid bin is given by the total collected photoelectrons from all PMTs mapped to that bin.
Secondly, a \emph{hit-time} map where each grid bin is given by the first hit time (in
nanoseconds) across all PMTs mapped to that bin. Each hit-time map bin is also corrected so that
the first hit time across all bins is at zero.

By design, the Hough transform within the seeding procedure uses the estimated interaction vertex
position to generate the transform space. Therefore, by re-binning the transform space to a $64
    \times 64$ grid a third \emph{hough-height} map is generated for each event. This event map aims
to provide a complementary but different representation of the event where Cherenkov rings are
instead represented as peaks, allowing for additional discriminating features to be learnt.

All three event maps: hit-charge, hit-time and hough-height, are down-sampled (from 32-bit floats)
using 8-bit encoding. The encoding not only significantly reduces storage requirements but also
dramatically increases the speed with which data can be loaded during training (which turns out to
be the primary training bottleneck). For each map type, a range over which to encode from zero up
to a \emph{cap-point} is chosen to minimise the number of bin values that are capped at the
maximum encoded value of 255. Table.~\ref{tab:encoding} shows the cap-points, and associated
percentage of bin values capped for each map type, while Fig.~\ref{fig:explore_8_bit_range} shows
the distribution of bin values for each map across the encoded range.

\begin{table}
    \begin{tabular}{lcc}
        Event map    & Cap-point & Capped percentage \\
        \midrule
        hit-charge   & 25 p.e    & 0.10\%            \\
        hit-time     & 120 ns    & 0.15\%            \\
        hough-height & 3500 p.e  & 0.23\%            \\
    \end{tabular}
    \caption[Table of input event map 8-bit cap-points]
    {Table showing the input event map cap-points (maximum value of the encoding range) and the
        associated percentage of bin values that are capped at the maximum 8-bit value of 255 as a
        consequence. The cap-points are specifically chosen so that the \emph{capped percentage}
        is approximately 0.1\%, keeping the information loss as small as possible.}
    \label{tab:encoding}
\end{table}

\begin{figure} % 8-BIT DIAGRAM %
    \includegraphics[width=0.7\textwidth]{diagrams/7-results/explore_8_bit_range.pdf}
    \caption[Event map encoded 8-bit distributions]
    {The distribution of encoded 8-bit values for the hit-charge, hit-time and hough-height event
        maps. Note the maximum (\emph{capped}) bin at an 8-bit value of 255.}
    \label{fig:explore_8_bit_range}
\end{figure}

Event maps for example events generated using the above procedure are shown in
Fig.~\ref{fig:explore_nuel_ccres_event} for a CC resonant $\nu_{e}$ event, in
Fig.~\ref{fig:explore_numu_ccdis_event} for a CC DIS $\nu_{\mu}$ event, in
Fig.~\ref{fig:explore_nuel_ncdis_event} for a NC DIS event, and in
Fig.~\ref{fig:explore_cosmic_event} for a cosmic muon event.

\begin{figure} % NUEL EVENT DIAGRAM %
    \includegraphics[width=\textwidth]{diagrams/7-results/explore_nuel_ccres_event.pdf}
    \caption[Example of a CC resonant $\nu_{e}$ event]
    {Three map representation of a CC resonant $\nu_{e}$ event. Initiated by a $\nu_{e}$ of
        energy \unit{3.3}{\GeV} the final state particles above the Cherenkov threshold include a
        $e^{-}$ of energy \unit{2.8}{\GeV} and a \unit{0.3}{\GeV} $\pi^{0}$.}
    \label{fig:explore_nuel_ccres_event}
\end{figure}

\begin{figure} % NUMU EVENT DIAGRAM %
    \includegraphics[width=\textwidth]{diagrams/7-results/explore_numu_ccdis_event.pdf}
    \caption[Example of a CC DIS $\nu_{\mu}$ event]
    {Three map representation of a CC DIS $\nu_{\mu}$ event. Initiated by a $\nu_{\mu}$ of
        energy \unit{3.5}{\GeV} the final state particles above the Cherenkov threshold include a
        $\mu^{-}$ of energy \unit{1.9}{\GeV}, a proton of energy\unit{2.0}{\GeV}, and a
        \unit{0.2}{\GeV} $\pi^{-}$.}
    \label{fig:explore_numu_ccdis_event}
\end{figure}

\begin{figure} % NC EVENT DIAGRAM %
    \includegraphics[width=\textwidth]{diagrams/7-results/explore_nuel_ncdis_event.pdf}
    \caption[Example of a NC DIS event]
    {Three map representation of a NC DIS event. Initiated by a $\nu_{e}$ of energy
        \unit{9.3}{GeV} the final state particles above the Cherenkov threshold include a proton
        of energy \unit{2.6}{\GeV} and a \unit{2.5}{\GeV} $\pi^{-}$.}
    \label{fig:explore_nuel_ncdis_event}
\end{figure}

\begin{figure} % COSMIC MUON EVENT DIAGRAM %
    \includegraphics[width=\textwidth]{diagrams/7-results/explore_cosmic_event.pdf}
    \caption[Example of a cosmic muon event]
    {Three channel representation of a cosmic muon event, containing a $\mu^{-}$ of energy
        \unit{2.9}{\GeV}.}
    \label{fig:explore_cosmic_event}
\end{figure}

\subsection{Baseline architecture} %%%%%%%%%%%%%%%%%%%%%%%%%%%%%%%%%%%%%%%%%%%%%%%%%%%%%%%%%%%%%%%
\label{sec:cnn_baseline_arch} %%%%%%%%%%%%%%%%%%%%%%%%%%%%%%%%%%%%%%%%%%%%%%%%%%%%%%%%%%%%%%%%%%%%

An illustrative diagram of the baseline chipsnet architecture is shown in Fig.~\ref{fig:chipsnet}.
Based on the VGG network (16 layer variant) previously mentioned in
Section.~\ref{sec:cnn_previous} and detailed in Ref.~\cite{simonyan2014} there are a few key
differences from the literature defined network, discussed below.

\begin{figure} % CHIPSNET DIAGRAM %
    \includegraphics[width=0.7\textwidth]{diagrams/6-cnn/chipsnet.pdf}
    \caption[Illustrative diagram of the baseline chipsnet architecture]
    {Illustrative diagram of the baseline chipsnet architecture. The three input event maps are
        separately passed through two VGG blocks each before their outputs are combined and passed
        through a further three VGG blocks together. The flattened VGG blocks outputs are then
        concatenated with five seed parameters (seed pars) and passed through two fully-connected
        (FC) layers of 512 neurons each before the output layer. Both the number of convolutional
        units and kernels is shown for each block. The detailed VGG block structure is shown
        within the grey box. The circular yellow $R$ and $Bn$ indicate the use of the ReLU
        activation function and batch normalisation, respectively.}
    \label{fig:chipsnet}
\end{figure}

\begin{itemize}
    \item Each of the three event maps: hit-charge, hit-time, and hough-height are initially fed
          into three separate branches. Each branch contains two VGG blocks with two convolutional
          layers each (four convolutional layers in total). The outputs from each branch are
          merged together using a concatenation layer before being fed to the rest of the network.
          This configuration allows for event map specific features to be learnt independently
          before combined features are learnt by the rest of the network.

    \item Batch normalisation as described in Section.~\ref{sec:cnn_theory_reg} is included before
          the activation (ReLU) function within every convolutional layer.

    \item Squeeze-and-excitation units, as detailed in Ref.~\cite{hu2018} are included after the
          max-pooling operation in all VGG blocks. These units introduce extra parameters to model
          the interdependencies between output feature maps, allowing the network to learn how to
          weight each feature map effectively.

    \item Dropout is included at the end of each VGG block as well as after the final
          fully-connected layer. Instead of dropping individual kernel elements, the dropout
          within the VGG blocks drops entire kernels at each training iteration, this is commonly
          called \emph{two-dimensional spatial dropout}. The dropout after the fully-connected
          layers is standard, in that it drops out individual fully-connected neurons.

    \item Five seed parameters are concatenated with the flattened layer before the
          fully-connected layers. Included are the three components of the estimated interaction
          vertex position and the two components of the estimated track direction ($s_{x},s_{y},
              s_{z},s_{\theta},s_{\phi}$). These parameters provide the network with spatial context
          as to where in the detector the input event maps have been generated and the dominant
          direction of PMT activity.
\end{itemize}

The chipsnet network architecture is implemented using the Keras~\cite{chollet2015} API built into
Tensorflow. This specification allows for predefined common layers such as a two-dimensional
convolution or a max-pooling layer to be easily structured into a full network definition.

\subsection{Baseline outputs} %%%%%%%%%%%%%%%%%%%%%%%%%%%%%%%%%%%%%%%%%%%%%%%%%%%%%%%%%%%%%%%%%%%%
\label{sec:cnn_baseline_outputs} %%%%%%%%%%%%%%%%%%%%%%%%%%%%%%%%%%%%%%%%%%%%%%%%%%%%%%%%%%%%%%%%%

Many CNN applications are found to benefit from learning multiple tasks at the same time, using
the same inputs. This is believed to be the case as training with multiple tasks tends to return a
network with an improved generalised representation of the inputs. Thus, features learnt for one
task can improve the performance of another. Additionally, multiple tasks work to regularise any
one output from overfitting. Commonly named \emph{multi-task} learning, this methodology is used
extensively in this work.

In order to train a network with multiple tasks (outputs), a loss function $E_{tot}$ must be
defined to combine the individual loss functions for each task $E_{i}$. The simplest way to do
this is via a linear weighted sum, such that
\begin{equation}
    E_{tot} = \sum_{i=1}^{i=N}w_{i}E_{i},
    \label{eq:multi_simple}
\end{equation}
where $N$ is the number of tasks and $w_{i}$ are the associated weights. In this work this is
referred to as the \emph{simple} multi-task loss.

The final network performance can strongly depend on the relative weighting between loss
functions, especially when the values returned from the different loss function differ by many
order of magnitude (common when combining regression and classification tasks). Therefore, finding
the optimal $w_{i}$ weights can be both difficult and time-consuming. Another approach outlined in
Ref.~\cite{kendall2018} aims to remedy this problem by learning the optimal weighting between loss
functions. This is done by introducing an additional trainable parameter $\sigma_{i}$, for each
loss function, such that
\begin{equation}
    E_{tot}= \sum_{i=1}^{i=N}\frac{1}{2\sigma_{i}^2}E_{i}+ \log\sigma_{i}.
    \label{eq:multi_learnt}
\end{equation}
In this work we refer to this as the \emph{learnt} multi-task loss.

The specific number and nature of outputs for the different specific networks are detailed in
Section.~\ref{sec:cnn_specific}. Although physically motivated to some degree, the exact set of
tasks and the loss combination technique used is mainly driven by trial-and-error. The chipsnet
software is specifically designed to enable this process by making it easy to configure the
network outputs via a simple configuration file.

\subsection{Baseline training} %%%%%%%%%%%%%%%%%%%%%%%%%%%%%%%%%%%%%%%%%%%%%%%%%%%%%%%%%%%%%%%%%%%
\label{sec:cnn_baseline_training} %%%%%%%%%%%%%%%%%%%%%%%%%%%%%%%%%%%%%%%%%%%%%%%%%%%%%%%%%%%%%%%%

All three networks are trained using Tensorflow (version 2.3.0) on an 18 core CPU (36 thread)
machine equipped with four NVIDIA GeForce RTX 2080 graphics processing units (GPUs). The
Tensorflow dataset API is used to create an efficient input data pipeline where data is loaded
on-the-fly at training time. This procedure ensures all GPU threads are utilised loading,
decoding, and preprocessing data for the primary GPU based network calculations before being
needed, maximising computational efficiency.

During preprocessing, all 8-bit input map values are converted to 32-bit float values bounded
between zero and one. Furthermore, a random factor scaling is applied to each map bin. Generated
from a Gaussian centred on one with a standard deviation of $\sigma_{r}$, by fluctuating the bin
values in this way the network is forced to focus less on the absolute bin values and more on the
underlying event topology. Not only does this process provide valuable regularisation to reduce
overfitting, but also makes the trained networks robust to small changes within the input
(explored within Section.~\ref{sec:results_robust}).

A minibatch training strategy of minibatch size of $n_{b}$, using the Adam
optimiser~\cite{kingma2014} with $\beta_{1}=0.9$, $\beta_{2}=0.999$, and $\epsilon = 1e-7$ is
used. The exact training sample size and composition for each specific network are given in
Section.~\ref{sec:cnn_specific}, but for all networks a 95\% training to 5\% validation data split
is employed early stopping used. The learning rate for each epoch $\eta_{e}$ is set to decrease
throughout training according to
\begin{equation}
    \eta_{e}=\frac{\eta_{0}}{1+c_{d}(e-1)},
\end{equation}
where $\eta_{0}$ is the initial learning rate, $e$ is the epoch number (starting at one), and
$c_{d}$ is the learning rate decay coefficient.

The list below details all the possible \emph{hyperparameters} that are tunable when training each
network. All are optimised for each specific network using the SHERPA hyperparameter tuning
framework~\cite{hertel2020}. Random configurations selected from either a range or selection of
choices for each hyperparameter (also outlined below) are tested to maximise performance.

\begin{itemize}
    \item \textbf{Initial learning rate, $\eta_{0}$:} in a range from 0.00005 to 0.001.
    \item \textbf{Learning rate decay coefficient, $c_{d}$:} in a range from 0.2 to 0.8.
    \item \textbf{Dropout probability, $p_{d}$:} in a range from 0.0 to 0.5.
    \item \textbf{Random scaling size, $\sigma_{r}$:} in a range from 0.0 to 0.1.
    \item \textbf{Minibatch size, $n_{b}$:} choosing from 32, 64, 128, or 256.
    \item \textbf{Multi-task loss combination strategy:} choosing from \emph{simple} or
          \emph{learnt}.
\end{itemize}

\section{Specific implementations} %%%%%%%%%%%%%%%%%%%%%%%%%%%%%%%%%%%%%%%%%%%%%%%%%%%%%%%%%%%%%%%
\label{sec:cnn_specific} %%%%%%%%%%%%%%%%%%%%%%%%%%%%%%%%%%%%%%%%%%%%%%%%%%%%%%%%%%%%%%%%%%%%%%%%%

The specific CNN implementations for cosmic rejection, beam classification and energy estimation
are outlined below. It is important to note that the exact configuration of networks outlined here
is the result of extensive testing designed to maximise the selection of a pure and efficient
sample of CC $\nu_{e}$ beam events whose neutrino energy can also be accurately determined.

As an example of an alternative implementation, if cosmic and beam classification are combined
into a single network, both objectives see a reduction in performance. The same is also true if
either classification task are combined with neutrino energy estimation. However, specific
secondary outputs such as counting the number of primary particles in conjunction with beam
classification are seen to improve performance. It is clear, therefore, that the multi-task
approach only works for tasks that require a similar learnt representation of the inputs. Put
simply; the tasks must be similar.

\subsection{Cosmic rejection} %%%%%%%%%%%%%%%%%%%%%%%%%%%%%%%%%%%%%%%%%%%%%%%%%%%%%%%%%%%%%%%%%%%%
\label{sec:cnn_specific_cosmic} %%%%%%%%%%%%%%%%%%%%%%%%%%%%%%%%%%%%%%%%%%%%%%%%%%%%%%%%%%%%%%%%%%

The cosmic rejection network aims to prevent the vast cosmic muon background from contaminating
the final selected sample of beam events. Therefore, the primary task is a simple binary
classification between beam and cosmic events. Additionally, training the network to also separate
events where the primary charged lepton escapes the detector volume or not, is found to improve
cosmic rejection performance. As a large proportion of cosmic muons are relatively high in energy
and, therefore, escape the detector in this fashion, there is motivation as to why this additional
task is helpful.

The network is trained on a sample of 3.15 million simulated events produced using the detector
simulation and event generation methods outlined in Section.~\ref{sec:chips_monte_carlo} with a
\chipsfive geometry. Roughly $1/3^{rd}$ are $\nu_{\mu}$ beam events, $1/3^{rd}$ $\nu_{e}$ beam
events, and $1/3^{rd}$ cosmic muon events, the counts of which are shown in
Fig.~\ref{fig:cosmic_training_sample}. All beam events (both $\nu_{\mu}$ and $\nu_{e}$) are
generated using the expected \chipsfive $\nu_{\mu}$ energy spectrum to closely mimic the dominant
$\nu_{\mu}$ beam component and appeared $\nu_{e}$ signal. All events are used for training with no
preselection as this is found to be the best for cosmic rejection performance. The full sample is
split 95\% to 5\% for training and validation, respectively.

\begin{figure} % COSMIC TRAINING SAMPLE DIAGRAM %
    \includegraphics[width=0.7\textwidth]{diagrams/7-results/explore_cosmic_training_sample.pdf}
    \caption[Number of training events per category for the cosmic rejection network]
    {Number of training events per category for the cosmic rejection network. All beam event
        interaction types are shown, however, all are classed as beam events (blue) in training
        against cosmic events (red).}
    \label{fig:cosmic_training_sample}
\end{figure}

There are two outputs to the cosmic rejection network:
\begin{enumerate}
    \item \textbf{Cosmic score (1 neuron):} Returns a score between zero and one corresponding to
          whether the event is beam or cosmic like. The binary cross-entropy loss function in
          Eq.~\ref{eq:binary_cross_entropy} is used for training with a simple multi-task weight
          of $1$.
    \item \textbf{Escapes score (1 neuron):} Returns a score between zero and one corresponding to
          whether the charged lepton in an event is contained or escapes the detector. The binary
          cross-entropy loss function in Eq.~\ref{eq:binary_cross_entropy} is used for training
          with a simple multi-task weight of $1$. NC beam events without a charged lepton are
          masked (do not contribute to the loss) during training for this output.
\end{enumerate}

The network is allowed to train for up to 30 epochs using the SHERPA optimised hyperparameters:
$\eta_{0}=0.00005$, $c_{d}=0.7$, $p_{d}=0.1$, $\sigma_{r}=0.02$, and $n_{b}=128$, with a simple
multi-task loss combination as given in Eq.~\ref{eq:multi_simple}. Typically, only 6 epochs are
required to reach the maximum validation sample \emph{cosmic score} accuracy, with early stopping
halting training after 11 epochs (taking 15 hours), as can be seen in
Fig.~\ref{fig:final_cosmic_history}.

\begin{figure} % COSMIC HISTORY DIAGRAM %
    \includegraphics[width=0.7\textwidth]{diagrams/7-results/final_cosmic_history.pdf}
    \caption[Loss and accuracy throughout training for the cosmic rejection network]
    {Total loss and \emph{cosmic score} accuracy for both the training sample (solid) and
        validation sample (dashed) throughout training for the cosmic rejection network. The final
        network weights are taken at epoch 6 as shown by the vertical black line.}
    \label{fig:final_cosmic_history}
\end{figure}

\subsection{Beam classification}%%%%%%%%%%%%%%%%%%%%%%%%%%%%%%%%%%%%%%%%%%%%%%%%%%%%%%%%%%%%%%%%%%
\label{sec:cnn_specific_beam} %%%%%%%%%%%%%%%%%%%%%%%%%%%%%%%%%%%%%%%%%%%%%%%%%%%%%%%%%%%%%%%%%%%%

The beam classification network aims to separate beam events by their interaction type to
primarily select a pure and efficient sample of appeared $\nu_{e}$ events, but also a sample of
survived CC $\nu_{\mu}$ events. Therefore, the principle task is a categorical classification
between CC $\nu_{e}$, CC $\nu_{\mu}$, and NC events. No attempt is made to separate the appeared
CC $\nu_{e}$ and intrinsic beam CC $\nu_{e}$ components as they are impossible to tell apart,
except for their distribution in energy.

Similar to the implementation used by DUNE~\cite{collaboration2020}, alongside the core
classification additional classification and particle counting tasks outlined below are included
to improve performance. Note that the particle counting tasks are not used in this work for
anything but increasing the primary classification performance. Future work, however, could
exploit any ability to separate exclusive final states deduced from these particle counts, to
reduce both energy resolution and systematic errors. As an example of a method already in use,
\nova use their ability to accurately determine the hadronic energy of an event to split their CC
$\nu_{\mu}$ sample into populations of different energy resolution. Each population can then be
treated separately in the analysis before being combined, which increases
performance~\cite{acero2018}.

The network is trained on a sample of 1.67 million simulated events produced using the detector
simulation and event generation methods outlined in Section.~\ref{sec:chips_monte_carlo} with a
\chipsfive geometry. Roughly half are $\nu_{\mu}$ beam events, with the other half being $\nu_{e}$
beam events, as shown in Fig.~\ref{fig:beam_training_sample}. All events (both $\nu_{\mu}$ and
$\nu_{e}$) are generated using the expected \chipsfive $\nu_{\mu}$ energy spectrum to closely
mimic the dominant $\nu_{\mu}$ beam component and appeared $\nu_{e}$ signal. All events are used
for training with no preselection as this is found to be the best for beam classification
performance, especially NC rejection. The full sample is split 95\% to 5\% for training and
validation, respectively.

\begin{figure} % BEAM TRAINING SAMPLE DIAGRAM %
    \includegraphics[width=0.7\textwidth]{diagrams/7-results/explore_beam_training_sample.pdf}
    \caption[Number of training events per category for the beam classification network]
    {Number of training events per category for the beam classification network. All beam event
        interaction types are shown.}
    \label{fig:beam_training_sample}
\end{figure}

There are nine outputs to the beam classification network:
\begin{enumerate}
    \item \textbf{Combined category (3 neurons):} Returns a classification probability score
          between zero and one for each of CC $\nu_{e}$, CC $\nu_{\mu}$, and NC (summing to one).
          The categorical cross-entropy loss function in Eq.~\ref{eq:categorical_cross_entropy} is
          used for training with a simple multi-task weight of $1$.
    \item \textbf{CC category (6 neurons):} Returns a classification probability score between
          zero and one for each of CC-QEL, CC-Res, CC-DIS, CC-Coh, CC-MEC and CC-other (summing to
          one). The categorical cross-entropy loss function in
          Eq.~\ref{eq:categorical_cross_entropy} is used for training with a simple multi-task
          weight of $1$. NC events are masked (do not contribute to the loss) during training for
          this output.
    \item \textbf{NC category (4 neurons):} Returns a classification probability score between
          zero and one for each of NC-Res, NC-DIS, NC-Coh and NC-other (summing to one). The
          categorical cross-entropy loss function in Eq.~\ref{eq:categorical_cross_entropy} is
          used for training with a simple multi-task weight of $1$. CC events are masked (do not
          contribute to the loss) during training for this output.
    \item \textbf{Electron count (4 neurons):} Returns a classification probability score between
          zero and one for each of 0, 1, 2, and 3+ electrons in the final state (summing to one).
          The categorical cross-entropy loss function in Eq.~\ref{eq:categorical_cross_entropy} is
          used for training with a simple multi-task weight of $1$.
    \item \textbf{Muon count (4 neurons):} Returns a classification probability score between zero
          and one for each of 0, 1, 2, and 3+ muons in the final state (summing to one). The
          categorical cross-entropy loss function in Eq.~\ref{eq:categorical_cross_entropy} is
          used for training with a simple multi-task weight of $1$.
    \item \textbf{Proton count (4 neurons):} Returns a classification probability score between
          zero and one for each of 0, 1, 2, and 3+ protons in the final state (summing to one).
          The categorical cross-entropy loss function in Eq.~\ref{eq:categorical_cross_entropy} is
          used for training with a simple multi-task weight of $1$.
    \item \textbf{$\pi^{\pm}$ count (4 neurons):} Returns a classification probability score
          between zero and one for each of 0, 1, 2, and 3+ charged pions in the final state
          (summing to one). The categorical cross-entropy loss function in
          Eq.~\ref{eq:categorical_cross_entropy} is used for training with a simple multi-task
          weight of $1$.
    \item \textbf{$\pi^{0}$ count (4 neurons):} Returns a classification probability score between
          zero and one for each of 0, 1, 2, and 3+ neutral pions in the final state (summing to
          one). The categorical cross-entropy loss function in
          Eq.~\ref{eq:categorical_cross_entropy} is used for training with a simple multi-task
          weight of $1$.
    \item \textbf{Photon count (4 neurons):} Returns a classification probability score between
          zero and one for each of 0, 1, 2, and 3+ photons in the final state (summing to one).
          The categorical cross-entropy loss function in Eq.~\ref{eq:categorical_cross_entropy} is
          used for training with a simple multi-task weight of $1$.
\end{enumerate}

The network is allowed to train for up to 30 epochs using the SHERPA optimised hyperparameters:
$\eta_{0}=0.0002$, $c_{d}=0.5$, $p_{d}=0.1$, $\sigma_{r}=0.02$, and $n_{b}=128$, with a simple
multi-task loss combination as given in Eq.~\ref{eq:multi_simple}. Typically, only 7 epochs are
required to reach the maximum validation sample \emph{combined category} accuracy, with early
stopping halting training after 12 epochs (taking 15 hours), as can be seen in
Fig.~\ref{fig:final_beam_history}.

\begin{figure} % BEAM HISTORY DIAGRAM %
    \includegraphics[width=0.7\textwidth]{diagrams/7-results/final_beam_history.pdf}
    \caption[Loss and accuracy throughout training for the beam classification network]
    {Total loss and \emph{combined category} accuracy for both the training dataset (solid) and
        validation dataset (dashed) throughout training for the beam classification network. The
        final network weights are taken at epoch 7 as shown by the vertical black line.}
    \label{fig:final_beam_history}
\end{figure}

\subsection{Energy estimation} %%%%%%%%%%%%%%%%%%%%%%%%%%%%%%%%%%%%%%%%%%%%%%%%%%%%%%%%%%%%%%%%%%%
\label{sec:cnn_specific_energy} %%%%%%%%%%%%%%%%%%%%%%%%%%%%%%%%%%%%%%%%%%%%%%%%%%%%%%%%%%%%%%%%%%

Accurate neutrino energy estimation is accomplished using multiple networks trained on separate
samples of $\nu_{e}$ and $\nu_{\mu}$ events across multiple CC interaction types. It is found that
separation such as this results in greater performance than if a single energy estimation network
or even separate $\nu_{e}$ and $\nu_{\mu}$ networks are trained. This is expected as a single set
of network weights is unlikely to be able to capture the specific topological features that
contribute to the energy for all types of event.

Alongside the primary neutrino energy regression task, additionally learning the primary charged
lepton energy and the interaction vertex position and time is found to improve performance.
Although this improvement is relatively small for neutrino energy estimation, it dramatically
improves primary charged lepton energy estimation when compared to being predicted alone. With two
energy tasks, the network is encouraged to learn how the primary charged lepton and neutrino
energies are related. As the interaction vertex position within the detector and hence distance
from the instrumented wall can impact the number of deposited photoelectrons, there is motivation
as to why these additional tasks are also helpful.

Separate networks are trained for each of CC-QEL(CC-MEC), CC-Res, and CC-DIS for both $\nu_{e}$
and $\nu_{\mu}$ events (8 in total) using 250000 corresponding simulated events each. All events
(both $\nu_{\mu}$ and $\nu_{e}$) are produced using the detector simulation and event generation
methods outlined in Section.~\ref{sec:chips_monte_carlo} with a \chipsfive geometry. The expected
\chipsfive $\nu_{\mu}$ energy spectrum is used to generate all events to closely mimic the
dominant $\nu_{\mu}$ beam component and appeared $\nu_{e}$ signal. Only events for which the
primary charged lepton is fully contained within the detector volume are used for training. Each
sample is split 95\% to 5\% for training and validation, respectively. Note that CC-QEL and CC-MEC
neutrino energy estimation is combined as both have incredibly similar final state topologies.

There are six outputs to each of the energy estimation networks:
\begin{enumerate}
    \item \textbf{Neutrino energy (1 neuron):} Returns the estimated neutrino energy. The
          mean-squared error loss function in Eq.~\ref{eq:mse} is used for training.
    \item \textbf{Charged lepton energy (1 neuron):} Returns the estimated primary charged lepton
          energy. The mean-squared error loss function in Eq.~\ref{eq:mse} is used for training.
    \item \textbf{Interaction vertex x-position (1 neuron):} Returns the estimated interaction
          vertex x-position. The mean-squared error loss function in Eq.~\ref{eq:mse} is used for
          training.
    \item \textbf{Interaction vertex y-position (1 neuron):} Returns the estimated interaction
          vertex y-position. The mean-squared error loss function in Eq.~\ref{eq:mse} is used for
          training.
    \item \textbf{Interaction vertex z-position (1 neuron):} Returns the estimated interaction
          vertex z-position. The mean-squared error loss function in Eq.~\ref{eq:mse} is used for
          training.
    \item \textbf{Interaction time (1 neuron):} Returns the estimated interaction time relative to
          the first PMT hit for each event. The mean-squared error loss function in
          Eq.~\ref{eq:mse} is used for training.
\end{enumerate}

Each network is allowed to train for up to 30 epochs using the SHERPA optimised hyperparameters:
$\eta_{0}=0.0002$, $c_{d}=0.5$, $p_{d}=0.1$, $\sigma_{r}=0.0$, and $n_{b}=128$, with a learnt
multi-task loss combination as given in Eq.~\ref{eq:multi_learnt}. Typically, only approximately
16 epochs are required to reach the minimum mean absolute error on the validation sample
\emph{neutrino energy}. Early stopping typically halts training after approximately 20 epochs
(taking 2 hours). An example of how an energy estimation networks training typically proceeds is
given in Fig.~\ref{fig:final_energy_history}.

\begin{figure} % ENERGY HISTORY DIAGRAM %
    \includegraphics[width=0.7\textwidth]{diagrams/7-results/final_energy_history.pdf}
    \caption[Loss and mean absolute error throughout training for the beam classification network]
    {Total loss and \emph{neutrino energy} mean absolute error for both the training dataset
        (solid) and validation dataset (dashed) throughout training for the energy estimation
        network trained on CC-QEL and CC-MEC $\nu_{e}$ events. The final network weights are taken
        at epoch 16 as shown by the vertical black line.}
    \label{fig:final_energy_history}
\end{figure}